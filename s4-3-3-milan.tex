% BEGIN SECTION 4 - STUART
% Last updated by Milan Ilnyckyj 2013-07-30



		\singlespacing
		\section {Divestment is compatible with the university's fiduciary duties}
		\label{sec:Fiduciary}
		\doublespacing
		


% Has incorporated comments from Alena Prazak



		\subsection {From the U of T divestment policy}
		
		
		
\begin{itquote}    
(i) prudent investment. The University has a fiduciary duty to manage investments responsibly to maximize return on its investments within a policy risk tolerance as approved by Business Board from time to time.

...

The committee will consider the following guidelines in considering the appropriate response to any request:
\begin{itemize}
  \item the extent and significance of the University's investment in a particular entity. Determination of whether investments are considered significant will depend on the committee's judgment of the relative magnitude of the University's holdings both as a fraction of all University investments and in relation to the market capitalization of the entity under review.
  \item the degree to which the entity itself is involved in the undesirable activity.
\end{itemize}
Normally, activity is considered significant if more than ten percent of the entity's revenues are derived from the undesirable activity.
\end{itquote}



	\subsection{Fossil fuel divestment is financially responsible}

	

Any fiduciary has two main factors to consider in investments: risk and return.
Fossil fuel divestment offers considerable potential to mitigate important risks, while creating only negligible new ones.
In addition, the historical returns of a portfolio that excludes fossil fuel stocks are comparable to those with no such exclusion, and there are good reasons to believe that the future returns of non-fossil-fuel investments will be competitive.
This section will consider both the financial case for divestment and questions about the practicality of divesting from a financial perspective, including in terms of the fiduciary duties borne by the University of Toronto.



In advice provided to the United Nations Environment Programme Finance Initiative, Freshfields Bruchhaus Deringer considered the relationship between fiduciary duty and environmental, social and governance (ESG) issues within common law jurisdictions.
They explain that: ``[t]he modern prudent investor rule, which incorporates both a duty of care and a duty of loyalty, emphasises modern portfolio theory and provides that: ... there is no duty to `maximise' the return of individual investments, but instead a duty to implement an overall investment strategy that is rational and appropriate to the fund''.\footcite[][p. 6]{UNEPFinanceInit}
They go on to explain that: ``[t]here is accordingly no reason why investment strategies should not include investments with positive ESG characteristics. The important limiting requirement is that imposed by the duty of loyalty: all investment decisions must be motivated by the interests of the fund's beneficiaries and / or the purposes of the fund.''\footcite[][p. 6]{UNEPFinanceInit}
Discussing Canada specifically, they explain: ``the power of investment is undertaken in a prudent manner when adequate processes (including completion of studies of the nature and quality of a proposed investment in light of the plan's total assets and obligations) have been followed and salient information (including expert opinions) has been considered''.\footcite[][p. 51]{UNEPFinanceInit}
The report lists a number of major Canadian institutional investors that have incorporated ESG issues into their processes, including the Ontario Municipal Employees Retirement System, the Ontario Teachers' Pension Plan Board, and the Canada Pension Plan Investment Board.\footcite[][p. 52]{UNEPFinanceInit}
In the same report, they claim: ``Climate change is an obvious example of an environmental consideration that is recognised as affecting value.''\footcite[][p. 11]{UNEPFinanceInit}
As this brief explains in detail, the beneficiaries and purposes of the University of Toronto's investments will be well-served by fossil fuel divestment.
Such divestment will not be financially harmful, will help the university reduce exposure to important risks, and will be in keeping with the values and reputation of the institution.



The Freshfields advice accords with advice that U of T itself sought, in the case of tobacco divestment in 2007.
Mr. Timothy Youdan of Davies Ward Phillips \& Vineberg was hired to produce a legal opinion on the University's investment policy with respect to a proposed prohibition on investment in the tobacco 
industry.
Mr. Youdan found that: ``the position taken in England by the Pension Law Review Committee 
is applicable in Ontario and applicable to trustees of a charity as well as pension trustees''.
He found further that this allows: ``pension trustees to properly exclude investments for nonfinancial reasons if doing so will not have a detrimental effect on the financial performance of the fund''.\footcite[][p. 6]{TobaccoReport_2007}



The International Energy Agency argues that: ``the deployment of a low-carbon energy system... delivers wide benefits by enhancing energy security, environmental protection and economic growth'', that: ``a low-carbon energy system increases energy security, particularly for energy importing countries, through reduced energy dependence and greater diversity of energy sources and technologies'', and that: ``the pathway to [stabilizing global temperatures at less than 2˚C above pre-industrial levels] is not just environmentally necessary but economically sound''.\footcite[][]{IEAOnTwoDegrees}
They argue that the net benefit of decarbonization amounts to US\$61 trillion if not discounted and US\$5 trillion if using a 10\% discount rate.
Furthermore, they argue that: ``low-carbon technologies often also reduce local air pollution, providing other environmental benefits and improve quality of life''.



\subsection {There is no evidence of a divestment penalty for investors}
\label{NoDivestPenalty}



Several studies have attempted to quantify the financial consequences of taking environmental factors into account in the investment management process.
In aggregate, these studies found no significant impact on investment risk in predictive models, nor a performance penalty in tests using historical data.
\begin{description}
  \item[Historical] The UN Environment Program Finance Initiative's analysis of twenty academic studies on the effect of incorporating Environmental, Social and Governance factors in the investment management process found there to be no evidence of a resulting performance penalty. The two reviewed studies that focused specifically on environmental factors found a positive relationship between consideration of those factors and performance.\footcite{UNEPFI2007}
  \item[Risk Based Assessment] The Aperio Group found that divesting from the ``Filthy Fifteen'' ``increases absolute portfolio risk by only 0.0006\%, or about a half of one one-thousandth of a percent.'' Even divesting from the entire Fossil Fuel sector only results in a 0.0034\% return penalty.\footnote{For the purposes of this study, the ``Filthy Fifteen'' was defined as the group of 15 U.S. companies judged by As You Sow and the Responsible Endowment Coalition to be the most harmful based on the amount of coal mined and coal burned along with other metrics.''} In other words, the portfolio does become riskier, but by such a trivial amount that the impact is statistically insignificant.\footcite{Aperio2013}
  \item[Forward Looking] Carbon Tracker and Standard \& Poors together conducted a study on the implications of carbon constraints for credit ratings of the oil and gas sector. Their scenario assumes reducing demand for \ce{CO2}-intensive fuels, in line with the internationally recognized limit of a 2˚C rise in global temperatures, and is ``not materially different from the current price deck assumptions.'' The study concludes with the statement:
\begin{quote} 
[A]s the price declines persist in our stress scenario of weaker oil demand, meaningful pressure could build on ratings. First to be affected would be the relatively focused, higher cost producers, and then the more diversified integrated players. In both cases, according to our study, the causes would be a decline in operating cash flows, weakening free cash flow and credit measures, along with less certain returns on investment and less robust reserve replacement.\footcite{SandPConstrained}
\end{quote}
  \item[Meta Analysis] It is frequently assumed that excluding the fossil-fuel sector from a portfolio will inevitably lead to a reduced performance, due to the reduction in potential investment opportunities.
  However, empirical research has repeatedly shown this assumption is fallacious.
  Deutsche Bank and Mercer have conducted major meta-studies that discovered the vast majority of academic studies of ESG investment performance found the incorporation of ESG factors into portfolio management to be either neutral or positive.\footcite{DeutscheBankSI} \footcite{MercerRI}
  \item[Case Study] Portfolio 21, based in Portland, Oregon, created one of the first sustainability-themed global equity mutual funds, known as Portfolio 21 Global Equity Fund (PORTX). The institutional share class has outperformed its benchmark by 105 basis points annualized over the past five years and by 93 basis points annualized over the last decade. Portfolio 21 has therefore demonstrated for more than a decade that a global investment strategy that avoids fossil fuels --- and many other unsustainable industries --- need not come at the cost of financial performance or increased portfolio risk.\footcite{FossilFreeInvesting}
\end{description}
Canadian socially-responsible investment funds like the NEI Ethical Canadian Dividend A fund --- which tries to balance social concerns with returns --- have outperformed the S\&P / TSX total return in 2013.\footcite[][]{HoldTheirOwn}



In July 2013, Impax Asset Management published a study examining the last seven years worth of data on international equity markets.
They compared a portfolio consisting of the MSCI World Index with another in which fossil fuel stocks were excluded and determined:
\begin{quote}
Excluding the fossil energy stocks from the MSCI World Index over the last seven years (to the end of April 2013) would have had a small positive impact on returns (0.5\% annually) and only a modest increase in tracking error of 1.6\% a year. For the five years to the end of April 2013, which excludes the dramatic run up in energy prices ahead of the 2008 financial crash, excluding the fossil energy sector would have improved returns by almost 0.5 percentage points annually, to 2.3\% a year from 1.8\%. Again, tracking error is low at 1.6\%.\footcite[][p. 5]{ImpaxBeyondFF} \footcite[See also: ][]{ImpaxEnergyCollective}
\end{quote}
These conclusions are echoed in recent analysis from MCSI ESG Research:
\begin{quote}
Over the period from January 2008 through March 2013, the market capitalization of the 247 fossil fuel reserve-owning companies described above ranged from approximately 7 percent to 8 percent of the MSCI ACWI IMI. Hence, excluding these stocks left between 93 percent and 94 percent of the MSCI ACWI IMI intact over the time series in terms of market capitalization. This meant that for each 10 percent active return differential in the carbon reserve stocks relative to the MSCI ACWI IMI, the effect of removing these stocks from the index ranged from 0.7 percent to 0.8 percent (70 to 80 basis points) in changes to active returns. Nearly all of this effect was due to industry factors, as opposed to country exposure and other style factors. As shown in the chart below, the performance of the MSCI ACWI IMI excluding the carbon reserve stocks closely tracked the MSCI ACWI IMI over the time series. Slight underperformance of the ``ex Carbon list'' appeared near the beginning of the time series, and slight outperformance of the ``ex Carbon list'' emerged toward the end of the time series. The active return differential over the entire time series was 1.2 percent (120 basis points) in favor of the ``ex Carbon list'' relative to the full MSCI ACWI IMI. The tracking error relative to full index was 1.9 percent (190 bps).\footcite[][p. 5]{MCSIFFP}
\end{quote}
There is reason to believe, therefore, that divestment would involve only a limited risk of foregoing improved ratings and investment returns.
Indeed, divestment could actually benefit the portfolio, in that it would remove risk of being invested in companies whose ratings only appear likely to decline in the long term.



	\subsection {Market capitalization and value at risk}



In a report for the Canadian Centre for Policy Alternatives, Marc Lee and Brock Ellis explain that: ``Canada is experiencing a carbon bubble that must be strategically deflated in the move to a clean energy economy''.\footcite[][p.5]{CanadaCarbonLiabilities}
Even with a high estimate of how much of the world's total carbon budget Canada can use up, they conclude that: ``78\% of Canada’s proven reserves, and 89\% of proven-plus- probable reserves... need to remain underground''.\footcite[][p.6]{CanadaCarbonLiabilities}
In short, the business plans and stock market valuations of fossil fuel companies are based on the unjustified assumption that they can continue to use the global atmosphere as a free dumping ground for greenhouse gas pollution.
As the injury caused by climate change has more obvious, governments have become increasingly willing to regulate fossil fuel use.
This progression can be expected to continue in the future, eventually compelling fossil fuel companies to leave significant reserves unburned.\footcite[See, for instance: ][]{ThreatenGrowth}
If the damage from fossil fuel burning amounts to \$50 per tonne (the low end of Lee and Ellis' estimate) then the damage that burning all Canadian fossil fuels would do amounts to \$844 billion, equivalent to two-and-a-half times the market capitalization and nearly twice the total assets of Canadian fossil fuel companies.
Based on a high damage estimate of \$200 per tonne, burning Canada's fossil fuel reserves would cause \$5.7 trillion in damage --- a figure 17 times larger than the market capitalization of these 144 firms and 13 times larger than their assets.
The analysis from Lee, Ellis, and others makes two things clear: burning fossil fuels causes very substantial amounts of social injury, and the prospect of strengthened regulations on greenhouse gas pollution threatens the profitability and stockmarket value of fossil fuel companies.



A recent article in the \emph{National Post} describes the `carbon bubble' and the risks it poses for investors: ``energy sector valuations ignore the world's climate change target and could be decimated if the international community puts its money where its mouth is and collectively moves to protect Mother Earth by attacking demand for oil, coal and gas''.\footcite[][]{OilGiantsMajorPain} \footcite[See also: ][]{TimelyIssue}
It goes on to explain: ``if the international community gets serious about its stated [2˚C] temperature goal, about two-thirds of existing energy sector reserves, which currently support about US\$4 trillion in share value and back more than US\$1 trillion in debt, are actually superfluous to the world's needs''.
The Climate Commission established by the Australian government echoes these findings.\footcite[][]{CriticalDecade2013}
Simon McKeon, the executive chairman of the commission, argues that: ``[a]nyone who believes they have literally hundreds of millions tonnes of first rate high emitting \ce{CO2} coal can no longer blindly believe there will be a strong market for that in 20, 30 years'' and that ``the best of resource investors are absolutely on to this''.\footcite[][]{CoalWillBeLeft}



	\subsubsection {Stated policy objectives are incompatible with the current valuation of fossil fuel reserves}



Fossil fuels may provide a hedge against other asset classes, but only in scenarios where unconstrained emissions lead to accelerated and possibly catastrophic warming. 
The international community is in broad agreement that this must not happen.



	\subsubsection {Regulatory risk is not adequately priced} 



As one scenario for the \emph{World Energy Outlook} in 2012, the International Energy Agency assumes international cooperation to keep \ce{CO2} under 450ppm, which in their model constrains the likelihood of warming greater than 2°C to 55\%. 
This is in contrast to their baseline New Policies Scenario, which assumes modest reductions in the rate of emissions increase compared to the third scenario, Current Policies. 
Evaluating the effect of this scenario on the price of fossil fuels, they estimate:
\begin{quote}
Compared with the New Policies Scenario, the global oil price in the 450 Scenario in 2035 is \$25 per barrel lower and the coal price almost 40\% lower. The price for natural gas falls by 23\% in Europe and 4\% in North America.\footcite[][p. 257]{IEA2012}
\end{quote}
In a report for the Australian National University and the Investor Group on Climate Change, Dr. Michael H. Smith concluded that: ``Climate change is forecast to dramatically increase the exposure of oil and gas companies to climate, energy and carbon price risks''.\footcite[][p. 14]{RisksForInvestors}



For any scenario where emissions are constrained to keep warming under 2°C, market assumptions regarding the profitability of fossil fuel extraction are necessarily optimistic. 
Marginal projects will become unprofitable and returns to investors for even the most profitable projects will decline. 
Indeed, the study conducted by Standard \& Poors even indicates near-term threats to the stability of investing in some fossil fuel companies: ``Under our stressed scenario, the ratings on companies with high development and production costs, including those focused on unconventional resources, could see rating pressure build \emph{within one or two years}, especially if the companies are relatively undiversified'' (emphasis ours).\footcite{SandPConstrained}
The study continues: ``We see a deterioration in credit measures for these smaller oil companies over 2014--2015, to a degree that could potentially lead to negative outlook revisions and downgrades over 2014--2017... this could result in an earlier deterioration in our business risk profile assessments''.
Furthermore, even ``the financial risk profiles of the oil majors would weaken modestly over the next five years''.
It is also worth noting that the study claims: ``the core business model [of fossil fuel companies] could come into question,'' and that ``this could potentially result in \emph{a downgrade of more than one notch} if we were to place less reliance on undeveloped or probable reserves than at present'' (emphasis ours).



The possibility of significant new climate regulations in the years ahead is not idle.
The Obama administration is expected to move forward with emissions limits on existing power plants.\footcite[][]{ReadyingLimits} \footcite[][]{ObamaJune2013}
In anticipation of his June 25th 2013 climate speech, in which he was expected to announce restrictions on \ce{CO2} emissions from existing coal plants, the stock price of major coal companies fell significantly: shares in Consol Energy fell 7.2\%, those in Cliffs Natural Resources fell 7.6\%, Peabody Energy's share price fell 7.2\%, and those in Alpha Natural Resources fell 8\%.\footcite[][]{CoalSharesPlunge} \footnote{All four companies are included in \nameref{sec:200Companies}}
The speech saw President Obama pledge to regulate \ce{CO2} emissions from existing power plants, promote the deployment of renewable energy, modernize the electrical grid, and further increase fuel economy standards for vehicles.
The fall in the stock value of coal companies deepened after the details of the new Obama plan were announced.\footcite[][]{CCPlanPoundsCoal}
The Government of China may also be considering imposing a carbon tax.\footcite[][]{ChinaTaxingCarbon}
Tightened regulations could pose a major risk for the value of fossil fuel companies.
A study by HSBC concluded that: ``Oil and gas majors, including, BP, Shell and Statoil, could face a loss in market value of up to 60 percent should the international community stick to its agreed emission reduction targets''.\footcite[][]{EconomicCase}



	\subsubsection {There is a strong potential for malinvestment in capital-intensive, long-term projects}



The IEA's 2012 World Energy Outlook concluded that, ``more than two-thirds of current proven fossil-fuel reserves cannot be commercialized in a 2˚C world before 2050.'' \footcite{IEA2012}
The Standard \& Poors Carbon Tracker Initiative study raises concerns concerning the fossil fuel sector: ``This illustrates to us the apparent divergence between the assets owned by coal, oil, and gas companies and the direction of negotiations at UNFCCC conferences.'' \footcite{SandPConstrained}
The study concludes that up to \$6 trillion could be wasted in fossil fuel investments that become unviable because of tightened climate change policies globally.\footcite[][]{SixTrillionBubble}
As explained by the National Round Table on the Environment and the Economy (NRTEE):
\begin{quote}
Every year of delay in sending strong, economy-wide policy signals represents a wasted opportunity to take advantage of natural cycles of infrastructure and equipment renewal, making it more difficult and expensive to meet emissions reduction targets. Our analysis shows that waiting until 2020 to implement climate policy aimed at cutting emissions by 65\% from 2005 levels by 2050 implies close to \$87 billion in refurbishments, retrofits and premature retirement of assets.\footcite[][p. 19]{FramingFuture}
\end{quote}
Another study found that up to 75 gigawatts of coal-fired electricity capacity will need to be retired by 2030 because of tightened environmental regulations.\footcite[][p. 10]{RisksInCoal}
Investment in \ce{CO2}-emissions-enabling infrastructure is contrary to the international community's consensus about the direction of the future.

% Milan had proofread up to here, 1:50am 2013-06-25

The persistently high price of fuels on the world market in recent years has lead to unprecedented investment on the part of the fossil fuel industry in projects that were previously deemed too marginal to profitably develop. 
Development of unconventional hydrocarbon reserves such as tar sands, oil shale, offshore drilling in extremely deep water and the arctic, hydraulic fracturing and mountaintop removal coal mining entails extremely high capital investment. 
Scenarios in which carbon emissions are restricted sufficiently to keep global temperatures from rising more than 2°C would likely cripple the return on much of this investment.



In anticipating restrictions on carbon emissions, the fossil fuel industry has been pinning its hopes on the development of effective methods of carbon capture and sequestration (CCS). 
Despite tremendous investment in CCS technology on the part of both the private and public sectors, economically feasible sequestration of emissions at scales needed to mitigate climate change remains elusive.\footnote{For more information on CCS, see: \nameref{CCSSaves}}
There are currently no commercial scale CCS projects in operation on the planet, and in 2008 Cambridge Energy Research Associates (CERA) predicted that it would be another two decades before CCS saw large-scale deployment. \footcite{CERACrossing}
According to the Carbon Tracker Initiative, even if CCS is deployed in line with an idealised scenario by 2050, this would only extend fossil fuel carbon budgets by 12-14\%, or just 4\% of total global reserves.\footcite{CTI2013}
It must be remembered that at the current rate of global carbon emissions, the entire budget of carbon emissions would be spent by the late 2020s, several years before large-scale CCS can be expected to come online.\footcite{CTI2012}



CCS has many other problems associated with it. For example, CCS would use extra energy, potentially as much as 40\% of the power generated by a power station. \footcite{GPCCS}
This reduces the efficiency of the power plant, both increasing financial costs and increasing the amount of fuel needed per energy output, which in turn contributes to the problems associated with fossil fuel extraction.
Indeed, the increased cost of the energy provided by CCS-enabled power stations would likely be higher than the cost of energy from renewable sources, and so would almost certainly never be implemented. \footcite{SmartPlanetCCS}
Storing carbon underground is risky --- safe and permanent storage of \ce{CO2} cannot be guaranteed, and even very low leakage rates could undermine any climate mitigation efforts. \footcite{GPCCS}
Finally, money spent on CCS will divert investments away from sustainable solutions to climate change, which the world will need to transfer to eventually, whether or not it burns all the available (non-renewable) fossil fuels.
Therefore, pinning our hopes on a non-existent technology, that is likely to both be more expensive and problematic than other energy sources, is a false hope.



	\subsubsection {Fossil fuel reserves as stranded assets} 



Given the degree to which proven reserves of carbon exceed allowable emissions for sub-2°C warming, companies with fossil fuel reserves as their largest assets may be substantially overvalued under current market conditions. 
Stranded assets in the form of unburnable reserves and large liabilities incurred to develop those reserves combine to create a risk not only to equity, but to bondholders as well.
The Carbon Tracker Initiative reports that in 2012 the Fossil Fuel sector spent \$674 billion prospecting for new sources of carbon, sources which cannot be exploited if the 2°C target is to be met. \footcite{CTI2013}


As the Carbon Tracker Initiative's 2012 report made clear, fossil fuel companies have significantly more exploitable sources of carbon available than is safe to burn. \footcite{CTI2012}
Therefore, when considering ``What A Carbon-Constrained Future Could Mean For Oil Companies' Creditworthiness,'' Standard \& Poors decided that, ``instead of considering issues of peak oil in terms of supply, this introduces a concept of peak oil demand.'' \footcite{SandPConstrained}
Whether they took the form of tightened efficiency requirements, the establishment of a cap-and-trade scheme, or the promulgation of a carbon tax, enhanced climate regulations would generally have the objective of reducing fossil fuel demand.
A reduction in oil, gas, and coal demand would have serious consequences for the fossil-fuel industry, particularly in Canada.



For example, MIT conducted a study that explored the effects \ce{CO2} emissions reduction policies would likely have on Canada's bitumen industry.
The study reaches the conclusion that, ``with \ce{CO2} emissions caps implemented worldwide, the Canadian bitumen production becomes essentially non-viable even with CCS technology, at least through our 2050 horizon.'' \footcite{MITConstraints}



As any investment manager knows, past performance does not guarantee future results, and it is becoming increasingly common for analysts and investors to discuss the prospect that the historical outperformance of fossil-fuel companies may be similar to the tech boom of the 1990s and the housing bubble of recent years.
However, the so-called ``carbon bubble'' potentially poses a much greater risk than either of these previous bubbles.
``Conservative estimates for the financial worth of the unburnable carbon reserves have ranged from \$20 trillion to \$27 trillion, so any associated write-down of fossil-fuel company valuations could very easily dwarf the recent \$2 trillion housing meltdown—by a full order of magnitude.'' \footcite[][p. 3]{FossilFreeInvesting}
According to John Fullerton, founder and president of the Capital Institute, this multi-trillion ``externality'' presents civilization with a ``Big Choice'': ``either we must absorb a \$20tn write-off into our already stressed global economy over the next decade, or we will implicitly accept civilization-transforming climate change''.\footcite{BigChoice}



\subsubsection {Volatility of investor sentiment}



Current market capitalization of the fossil fuel industry rests in part on the assumption that the global investor class will continue to see the sector as a reliable investment even as damage from climate change becomes apparent. 
This assumption has been increasingly challenged from both outside and within the financial industry. 
Traditionally conservative-minded publications such as \emph{The Economist}, \footcite{EconomistUnburnable} \emph{Business Week} \footcite{BusinessWeekOvervalued} and the \emph{Financial Times} \footcite{FTOvervalued} have published articles suggesting the fossil fuel sector is overvalued. 
In recent months, other voices within the financial industry such as investor groups and hedge fund managers have been increasingly sounding the alarm over the ``Carbon Bubble''. \footcite{JeremyGrantham} 
The Guardian recently reported
\begin{quote}
The message to all the players across the financial chain, from ratings agencies through accountants, to actuaries, investment advisors and all the rest, is also obvious. If the regulators won't do their job, do it for them. \emph{Jump, before you are pushed} (emphasis ours). \footcite{Guardian6Trillion}
\end{quote}
The afore-mentioned Standard \& Poors study, which saw a declining trend on both the short-term and long-term outlook for fossil fuel companies (both mid-size and large), reached its conclusion without ``explicitly [factoring in] any mitigating measures such as ... material cuts in near-term capital investment.'' \footcite{SandPConstrained}
However, there is already a significant, international fossil-fuel divestment movement that could result in such material cuts: over 300 colleges and universities and over 100 cities and states currently have divestment campaigns, along with several religious institutions.\footnote{For detailed information on the status of ongoing divestment campaigns, see: \nameref{PeerSchools}}




	\subsection {Fossil fuels represent a risk to the university's other investments}



Institutional investors, and universities in particular, are often expected to plan financially on a timescale far longer than average. 
On timescales of 50 years or more, the consequences of unconstrained emissions are very likely to overshadow all other financial considerations.
According to a 2012 report by DARA, climate change is already costing the world more than \$1.2 trillion, wiping 1.6\% annually from global GDP.
By 2030, the researchers estimate, the cost of climate change and air pollution combined will rise to 3.2\% of global GDP, with the world's least developed countries forecast to bear the brunt, suffering losses of up to 11\% of their GDP. \footcite{DARACVM}
Going further into the future is increasingly hard to predict, with estimates varying widely: the Stern Review estimates losses of between 5\%-20\%, \footcite{Stern2007} and a United Nations report asserts that climate change could cost Latin American and Carribean countries 137\% of GDP by 2100. \footcite{CCLatinAmerica}
However, regardless of the variations of predictions, the trend is clear: the more the climate changes, the greater the reductions to GDP.
As the Stern Review explains, ``[t]he benefits of strong, early action on climate change outweigh the costs''.\footcite[][Executive summary at: \url{http://www.hm-treasury.gov.uk/d/Executive_Summary.pdf}]{Stern2007}
Therefore, mitigating climate change can be expected to result in a relatively higher GDP, and to result in greater returns on the university's investments over the long term.\footcite[See also: ][]{EconomicCase}



In 2012, the National Round Table on the Environment and the Economy studied the risks posed to business by climate change.
They identified many categories of associated risk, including fire and property damage, storms and other natural perils, business interruption, disease and disability, and liability claims.\footcite[][p. 3]{ManagingBusinessRisks}
The report concludes that: ``some industries will be impacted significantly and permanently''.\footcite[][p. 2]{ManagingBusinessRisks} \footcite[See also: ][]{LeveragingInvestmentsCScience}
They project that climate change will cost Canada roughly \$5 billion per year by 2020, rising to between \$21 billion and \$43 billion per year by mid-century.\footcite[][p. 2]{ManagingBusinessRisks}
In ``Facing the Elements: Building Business Resilience in a Changing Climate'', the NRTEE identifies the oil and gas sector, mining, agribusiness, retail and distribution, hydroelectricity, technology and communications, and financial services as industries at risk of being negatively impacted by climate change.\footcite[][p. 9--10]{FacingTheElements}




	\subsection {Attractive substitutes exist for divested equities}



There are many attractive alternatives that could form substantial portions of the university's portfolio.
The renewable energy sector has enormous growth potential and is starting to match even conventional fossil-fuel energy prices (let alone unconventional energy prices).
According to NRTEE: ``Understanding the implications of the global low-carbon transition for Canada and making choices that maximize the opportunities and minimize the risks are critical to Canada's long-term prosperity''.\footcite[][p. 15]{FramingFuture}
In particular, they argue that the ``public and private sectors need to mobilize investment in low-carbon infrastructure and technology''.\footcite[][p. 17]{FramingFuture}
Unsubsidised renewable energy is now cheaper than electricity from new-build coal- and gas-fired power stations in Australia, according to new analysis from research firm Bloomberg New Energy Finance. \footcite{BlombergAussieWind}
Solar power is predicted to be cheaper than fossil fuel power in the USA as soon as 2015. \footcite{GlobalDataSolar}
In March 2013, 100\% of the new energy on the U.S. grid was solar power. \footcite{SmartPlanetSolar100}



There are three broad-based mutual funds that are completely fossil free: Green Century Balanced Fund (GCBLX), Portfolio 21 Global Equity Mutual Fund (PORTX), and Shelton Green Alpha Fund (NEXTX). \footcite{GFFMyMoney}
The GCBLX is solidly in the middle of its grouping regarding overall rating, returns and risk of the category, \footcite{GCBLX} and as mentioned earlier, the PORTX has actually out-performed its peers. \footcite{PORTX}
The Shelton Green Alpha Fund only started recently, and hasn't yet received a rating. \footcite{NEXTX}



There is a body of credible evidence demonstrating that environmental, social and governance (ESG) considerations often have a role to play in the proper analysis of investment \emph{value}. 
As such they cannot be ignored, because doing so may result in investments being given an inappropriate value.
A UNEP Finance Initiative report points out, ``many people wonder what good an extra percent or three of patrimony are worth if the society in which they are to enjoy retirement and in which their descendants will live deteriorates.'' \footcite{UNEPFinanceInit}
The report notes that, in Canada, there is an increasing frequency of debate regarding ESG issues, and that there is ``considerable and persistent support'' for increased consideration of ESG investment strategies.



\subsection {Pensions and climate change}



Pensions are intended to allow the pensioner to enjoy a satisfactory, even comfortable, retirement.
However, the more the climate changes, the lower the retirees' quality of life will be.
A study conducted by the World Bank makes it clear that a 4˚C hotter world will be a much more hostile world than one in which there has only been a 2˚C rise, with 6˚C or greater rises being more hostile still. \footcite{WorldBank4C}
The previous sections of this brief demonstrate that even a 2˚C rise will result in a greater frequency of natural disasters than the relative climate stability of the development of human civilisation thus far.
Indeed, the world's top scientists have calculated that a concentration of \ce{CO2} in the atmosphere that is higher than 350ppm is not compatible with the planet ``on which civilization developed and to which life is adapted.'' \footcite{TargetAtmosphere}
The relationship between pension obligations and climate change has already been acknowledged by financial analysts.
In their report to the UNEP Finance Initiative, Freshfields Bruchhaus Deringer explain: ``Following the recent release of a report by Mercer Investment Consulting noting the financial impact that climate change has already had on companies' costs, revenues, assets and liabilities, the UK Carbon Trust expressed the view that `Pension fund trustees have a duty to address the financial risk posed by climate change when making investment decisions'''\footcite[][p. 11]{UNEPFinanceInit}
The Carbon Trust report further explains that: ``[i]t is crucial that actions to address climate risk be taken by pension fund trustees'' and ``the impact of climate change on corporations is not just something to worry about over the longer-term, it is an issue to consider today''.\footcite[][p. 2, 10]{TrusteesGuide}



In their report for the Canadian Centre for Policy Alternatives, Marc Lee and Brock Ellis also consider the special question of divestment and pension funds.
They highlight how ``[a]ddressing risk is inherent to financial market investment'' but point out that ``there has been a general failure to account for climate risks, and a tendency to view any screening for environmental purposes to be detrimental to financial performance''.\footcite[][p.8--9]{CanadaCarbonLiabilities}
They also argue that: ``by not accounting for climate risk, large amounts of invested capital are vulnerable to the carbon bubble''.\footcite[][p.9]{CanadaCarbonLiabilities}
In assessing the university's obligation toward pensioners, it is also worth thinking beyond the simple metric of the financial performance of their pension funds.
Unmitigated climate change is expected to cause substantial harm to both human prosperity and the quality of the natural environment around the world.
In his comprehensive study of the economics of climate change, Nicholas Stern concluded that failing to mitigate climate change ``create risks of major disruption to economic and social activity... on a scale similar to those associated with the great wars and the economic depression of the first half of the 20th century''.\footcite[][See also: \url{http://www.hm-treasury.gov.uk/d/Executive_Summary.pdf}]{Stern2007}
Stern projected that up to 20\% of global GDP could be swallowed up by damage from climate change.
Since the publication of the Stern Review in 2007, Nicholas Stern has stated that they underestimated the threat in their assessment.\footcite{Stern2008}
He has also drawn specific attention to the mismatch between the size of proven fossil fuel reserves and the quantity of fossil fuels that can be burned without exceeding the 2˚C target.\footcite[][]{FTOvervalued}
In evaluating its obligations to pensioners, the university must consider both their financial welfare (which is threatened by unmitigated climate change) and the kind of impoverished world future pensioners can be expected to inhabit if nothing is done to seriously constrain how much fossil fuel is burned.



It is in the best interests of the future pensioners of the University of Toronto (its current employees) to live in a world with a stable climate.
Elizabeth Sawin of the Sustainability Institute explains the long lifetime of GHGs in the atmosphere by comparing it with educational timelines.
By the time a college president who is now 55 retires, 89\% of the \ce{CO2} released between 2012 and 2016 would remain in the atmosphere.
The same article also highlights the perspective of students: 
\begin{quote}
Even if today's college students live to be 100 years old, more than half of the \ce{CO2} released into the atmosphere during the four years they are in college will still be present there at the end of their lives --- warming the planet and contributing to extreme events, like droughts, floods, and storms all the while --- long after the decision makers behind those investment choices will have left office. The college students across the US who are arguing that their education should not be funded by actions that diminish the health of the world in which their future will unfold have a strong case, supported by the basic physics of the climate.\footcite{ClimateInteractivePersist}
\end{quote}
James Powell, former-president of Oberlin, Franklin and Marshall, and Reed College, further reinforces this concept, suggesting that trustees have a quasi-legal duty to do all they can about climate change: ``The board is supposed to make sure that the endowment allows for intergenerational equity, that the students who are going to Oberlin in 2075 get as much benefit from it as those there now. But with global warming, you're guaranteeing a diminution of quality of life decades out''.\footcite{CaseForDivestment}



Therefore, not only is divestment from the fossil fuel industry a sound financial decision for meeting the financial obligations of prudent investment, the current employees of the University of Toronto will benefit from such divestment.



		\subsection{The significance of the University of Toronto's investments}



The University of Toronto has significant holdings in fossil fuel companies.
For example, the largest two single-company holdings listed by the University of Toronto Asset Management Corporation (UTAM) in 2012 were Royal Dutch Shell Plc (\$9.8 million) and BP PLC (\$78 million).\footcite{UTAM_2012_Int}
Although not all investment quantities are listed by UTAM, and the precise quantity of investment is likely to change daily, it is clear from UTAM's reports that direct holdings of fossil fuel companies are a significant part of the University of Toronto's endowment.



Likewise, although not all of the 200 companies listed in \nameref{sec:200Companies} are purely fossil-fuel companies, the holdings of those companies is significant.
Another large investment listed in 2012, Rio Tinto PLC (\$5.3 million), is primarily a mining company, but also has large fossil fuel reserves, with a  \ce{CO2} estimated at 5.23 Gt, giving it the rank of 14th-largest reserves out of the coal companies listed.\footcite{CTI2012}
The Energy and Minerals section of the Rio Tinto Group was responsible for 28\% of underlying earnings in 2008. \footcite{RioTintoChartbook}
Most of the 200 companies listed in the appendix have much higher proportion of fossil-fuel attachments, as many of them are primarily fossil-fuel companies, and collectively they possess a quantity of fossil fuels sufficient to breach the 2˚C barrier and impose dangerous climate change on the rest of the world.
In most if not all cases, more than ten percent of the revenues of the 200 listed firms are derived from the undesirable activity of fossil fuel production, as specified in the \nameref{ProceduresResponding}.



If U of T decided to divest, it would have an impact out of proportion with how much of each firm's market capitalization is owned by the university.
This is because of the important signal divestment would send to the university's peer schools, as well as to other institutional investors.



		\subsection{Towards divestment}



Many resources now exist to guide those who are considering fossil fuel divestment. 
In addition, asset managers, indexing firms, and other financial intermediaries are rapidly developing new products and services to respond to investor demand for fossil-free investment options.
The following resources provide valuable guidance:
\begin{itemize}
	\item Institutional Pathways to Fossil-Free Investing by Joshua Humphreys\footcite[][]{FossilFreeInvesting}
	\item Resilient Portfolios \& Fossil-Free Pensions by HIP Investor Inc. and GoFossilFree.org\footcite[][]{ResPortFFPensions}
	\item Divestment from Fossil Fuels: A guide for city officials and activists\footcite[][]{MayorsInnovationDivestGuide}
\end{itemize}



% END SECTION 4 - STUART