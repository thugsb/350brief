% BEGIN SECTION 4 - STUART
% Last updated by Milan Ilnyckyj 2013-APR-24



		\begin{samepage}
		\section {Divestment is compatible with the university's fiduciary duties}

\begin{vcom}
	Stuart is taking over the drafting of this section. This may be a place where we can make especially good use of documents already prepared by other schools, including: (a) schools U of T regards as peers, like Harvard (b) schools in a similar position to U of T, like McGill and UBC and (c) schools that have already divested.
\end{vcom}

\begin{vcom}
	This section also needs to specifically address this section from the Procedures for Responding to Social and Political Issues with Respect to University Divestment: ``the extent and significance of the University’s investment in a particular entity.  Determination of whether investments are considered significant will depend on the committee's judgment of the relative magnitude of the University’s holdings both as a fraction of all University investments and in relation to the market capitalization of the entity under review.''
\end{vcom}


	\subsection {No evidence of a divestment penalty for investors}
	
Several studies have attempted to quantify the financial consequences of divestment from the fossil fuel industry or heavy polluters in general. 
In aggregate, these studies found no significant impact on investment risk in predictive models, nor a performance penalty in tests using historical data.
\begin{description}
\item[historical] UNEP FI meta-analysis finds no evidence of penalty to screened portfolios \footcite{UNEPFI2007}
\item[risk based] aperio group: impact of divestment not found significant risk based modeling shows very small differences between screened matching portfolio and index (russel 3000) \footcite{Aperio2013}
\end{description}


	\subsection {Market capitalization and value at risk}

\subsubsection {Stated policy objectives incompatible with valuation of fossil fuel reserves}

Fossil fuels may provide a hedge against other asset classes, but only in scenarios where unconstrained emissions lead to accelerated and possibly catastrophic warming. 
The international community is in broad agreement that this must be prevented from happening.
\end{samepage}



\subsubsection {Regulatory risk not adequately priced} 



As one scenario for the World Energy Outlook in 2012, the International Energy Agency assumes international cooperation to keep \ce{CO2} under 450ppm, which in their model constrains the likelihood of warming greater than 2°C to 55\%. 
This is in contrast to their baseline New Policies Scenario, which assumes modest reductions in the rate of emissions increase compared to the third scenario, Current Policies. 
Regarding the effect this policy environment would have on the price of fossil fuels they estimate:
\begin{slquote}Compared with the New Policies Scenario, the global oil price in the 450 Scenario in 2035 is \$25 per barrel lower and the coal price almost 40\% lower. The price for natural gas falls by 23\% in Europe and 4\% in North America.\footcite[][p. 257]{IEA2012}\end{slquote}



For any scenario where emissions are constrained to keep warming under 2°C, market assumptions regarding the profitability of fossil fuel extraction are necessarily optimistic. 
Marginal projects will become unprofitable and returns to investors for even the most profitable projects will decline. 


\subsubsection {Potential for malinvestment in capital intensive, long-term projects}



The persistently high price of fuels on the world market in recent years has lead to unprecedented investment on the part of the fossil fuel industry in projects that were previously deemed too marginal to profitably develop. 
Development of unconventional hydrocarbon reserves such as tar sands, oil shale, offshore drilling in extremely deep water and the arctic, hydraulic fracturing and mountaintop removal coal mining entails extremely high capital investment. 
Scenarios in which carbon emissions are restricted sufficiently to keep global temperatures from rising more than 2°C would likely cripple the return on much of this investment.

In anticipating such a scenario, the fossil fuel industry has been pinning its hopes on the development of effective methods of carbon capture and sequestration (CCS). 
Despite tremendous investment in this technology on the part of both the private and public sectors, economically feasible sequestration of emissions at scales needed to mitigate climate change remains elusive.

\begin{vcom}
	We should probably include a longer explanation for why CCS is unlikely to solve the problem, given (a) the degree to which Canadian federal and provincial climate plans depend on it becoming cheap and effective very soon and (b) the likelihood that fossil fuel companies will point to CCS as a way of squaring their intent to burn all their reserves with the need to avoid dangerous climate change
\end{vcom}



\subsubsection {Fossil Fuel Reserves as Stranded Assets} 



Given the degree to which proven reserves of carbon exceed allowable emissions for sub-2°C warming, companies with fossil fuel reserves as their largest assets may be substantially overvalued under current market conditions. 
Stranded assets in the form of unburnable reserves and large liabilities incurred to develop those reserves combine to create a risk not only to equity, but to bondholders as well.



\subsubsection {Volatility of investor sentiment}



Current market capitalization of the fossil fuel industry rests in part on the assumption that the global investor class will continue to see the sector as a reliable investment even as damage from climate change becomes apparent. 
This assumption has been increasingly challenged from both outside and within the financial industry. 
Traditionally conservative-minded publications such as \emph{The Economist}, \emph{Business Week} and the \emph{Financial Times} have published articles suggesting the fossil fuel sector is overvalued. 
Other voices within the financial industry such as investors groups and hedge fund managers have been more unequivocal, with talk of a ``carbon bubble'' beginning to gain currency with many investors and industry experts. 

\begin{vcom}
	The articles mentioned here should be cited
\end{vcom}



\subsection {Fossil Fuels represent a risk to the university's other investments}



Institutional investors, and universities in particular are unique in that they are often expected to plan financially on a timescale far longer than average. 
On timescales of 50 years or more, the consequences of unconstrained emissions are very likely to overshadow all other financial considerations.



\subsection {Attractive substitutes exist for divested equities}



The renewable energy sector has enormous growth potential. Even under current regulatory regimes, the cost of electricity 	
 


\subsection {Pensions and climate change}

\begin{vcom}
	It is virtually certain that the ad hoc committee will consider the special responsibility the university has toward current and future retirees. We need to make the case that (a) divestment is a sound financial decision for meeting those obligations and (b) if current employees want a decent retirement, dangerous climate change must be avoided.
\end{vcom}



% END SECTION 4 - STUART