% BEGIN SECTION 7
% Last updated by Milan Ilnyckyj 2013-06-08



	\section{Short answers to common questions}



	\subsection{Why should the university `take sides' in this matter? Is it appropriate for the university to take stances on social and political issues?}
	


	\subsection{Isn't shareholder activism a better option?}
	
	
	
	\subsection{Other people will buy the stocks we sell, so how does this make a difference?}
	
	
	
	\subsection{What are the University of Toronto's peer schools doing?}
	


On May 30th 2013, the faculty senate at the University of California, Santa Barbara voted in favour of fossil fuel divestment.
Earlier that week, the student government at Stanford University voted in favour of divestment.
In total, the student governments at seven campuses of the University of California have voted in favour of divestment.\footcite[][]{UCSB2013}


	
	\subsection{But don't fossil fuel companies also invest in renewable energy?}

\begin{vcom}
A good way to start this answer may be drawing attention to BP's shallow and short-lived `beyond petroleum' rebranding
\end{vcom}



The sums being invested in renewable energy by fossil fuel companies are dwarfed by the investments they are making in unconventional sources of coal, oil, and gas.
For example, BP has announced its intention to increase spending on arctic drilling by \$1 billion over five years, increasing its fleet of oil rigs from seven to nine by 2016.\footcite[][]{BPArcticBillion}



Conventional fossil fuel sources are more than sufficiently abundant to allow humanity to far exceed the 2˚C `safe limit' for climate change.
The costly pursuit of exotic new reserves shows how fossil fuel companies have failed to internalize the reality of climate change and are continuing to implement investment plans that are sharply at odds with planetary safety.

	
	
	\subsection{In what cases have courts found that fossil fuel companies caused injury?}
	
	
	
	\subsection{Isn't the energy sector, including oil and gas extraction, production and distribution, highly regulated by government at all levels?}
	
	
	
	\subsection{Can humanity manage without fossil fuels?}
	


This question was extensively examined by Cambridge physicist David MacKay, resulting in his 2009 book \emph{Sustainable Energy – without the hot air}.\footcite[][]{MacKay2009}


The entire book is available for free online at: \url{http://withouthotair.com/}


	\subsection{Won't divestment hurt the endowment, including U of T's ability to provide scholarships?}
	
	
\begin{vcom}
	This should cover both the effect on returns and the effect on portfolio diversity
\end{vcom}



	\subsection{Shouldn't U of T fight climate change through research and education?}



% END SECTION 7