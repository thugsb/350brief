% BEGIN SECTION 2 - MILAN
% Last updated by Milan Ilnyckyj 2013-06-02

		\section{Climate change is settled science}


	
	\subsection{From the U of T divestment policy}


\begin{itquote}	
The University's core academic values include freedom of inquiry and open debate.
As a general matter, the University does not take positions on social or political issues apart from those directly pertinent to higher education and academic research. 
Instead, its role is to provide a forum within which issues can be studied carefully and debated vigorously. 
Given these values, the University will not consider any proposals for restrictions on its investments that require the institution to take sides in matters that are properly the subject of ongoing academic inquiry and debate.
\end{itquote}



	\subsection{It is not properly the subject of ongoing academic debate that}


\begin{vcom}
Ideally, for each claim we should cite the best possible accessible source for laypeople as well as the most authoritative possible scientific source. It would be great if people could add additional references to this section.
\end{vcom}

\begin{itemize}
	\item The 10,000 years of human civilization have taken place during a span of relative climatic stability.\footnote{This claim is supported by evidence from ice core samples taken in Vostok, Antarctica as well as other proxy measures of climate such as pollen in lake sediments and tree rings.}\footcite[][p. 4]{Alley2000}
	\item Burning coal, oil, and gas produces known quantities of carbon dioxide (\ce{CO2}).\footcite[For example, the U.S. Environmental Protection Agency lists quantities of \ce{CO2} produced by burning a barrel of oil, metric tonne of coal, or therm of natural gas:][]{CalculationsReferences}
	\item Before the industrial revolution, the concentration of \ce{CO2} in the atmosphere was approximately 280 parts per million (ppm).\footnote{Evidence for this includes the records of how much fossil fuel has been burned, as well as the changing isotopic ratio of carbon in the atmosphere.}\footcite[][]{IPCC4ARdrivers}
	\item It has now risen to over 390 ppm, largely because of the burning of fossil fuels.
	\item At present, the concentration of \ce{CO2} in the atmosphere is rising at a rate of approximately 2.0 ppm per year.\footcite[][]{NOAATrends}
	\item If humanity continues to burn fossil fuels at the present rate, the concentration of \ce{CO2} in the atmosphere will rise to well over 550 ppm by 2100.
	\item Adding carbon dioxide to the atmosphere reduces the amount of energy the Earth radiates into space. This causes the planet to warm.\footcite[][]{IPCC4ARdrivers}
	\item Based on evidence from ice cores, we know that doubling the amount of \ce{CO2} in the atmosphere causes global temperatures to rise by about 3˚C.
	\item Governments around the world, including the government of Canada, have adopted 2˚C as the threshold beyond which climate change should be considered `dangerous'.\footcite{CopenhagenAccord}
	\item If the world is to avoid crossing the 2˚C limit, most of the world's remaining fossil fuels must be kept in the ground.\footcite[][]{IEA2012}
\end{itemize}



Comprehensive and authoritative scientific statements on the key elements of climate change date back at least to the 1979 U.S. National Academy of Sciences report (the Charney report).\footcite[][]{Charney1979}
The report concluded that human activities --- particularly greenhouse gas emissions --- were altering the climate in potentially dangerous ways. These conclusions have been subsequently re-affirmed in the four major reports of the Intergovernmental Panel on Climate Change (IPCC) in 1990, 1995, 2001, and 2007.\footcite[][]{IPCC1990}\footcite[][]{IPCC1995}\footcite[][]{IPCC2001}\footcite[][]{IPCC2007}


\begin{vcom}
	Add reference to G8 science academy statement: http://www.nationalacademies.org/includes/G8+5energy-climate09.pdf
\end{vcom}


Mitigating climate change is important for allowing the university to achieve its academic mission. 
In the event that the world fails to curb greenhouse gas emissions and produces well over 2˚C of climate change, substantial damage is expected to be imposed on the global economy.
The Stern Review on the economics of climate change concluded that under a business-as-usual scenario, there is ``at least a 50\% risk of exceeding 5°C global average temperature change'' and that ``[s]uch changes would transform the physical geography of the world. A radical change in the physical geography of the world must have powerful implications for the human geography - where people live, and how they live their lives.''\footcite[][See long executive summary at: \url{http://www.hm-treasury.gov.uk/d/Executive_Summary.pdf}]{Stern2007}
Such an outcome threatens the growth prospects of the endowment and pension funds of the University of Toronto. 
It also creates additional geopolitical risks such as agricultural disruption and forced migration.



James Powell, former President of Oberlin, Franklin and Marshall, and Reed College, argues that university trustees have a quasi-legal duty to do all they can about climate change, arguing:
\begin{quotation}
``The board is supposed to make sure that the endowment allows for intergenerational equity, that the students who are going to Oberlin in 2075 get as much benefit from it as those there now. But with global warming, you’re guaranteeing a diminution of quality of life decades out."''
\end{quotation}
Taking action to address climate change is not an example of needlessly taking sides in a controversial issue. Rather, it is a matter of taking part in a necessary global transition. 
If the world fails to constrain the worst impacts of climate change, serious deleterious impacts can be expected for Canada and the University of Toronto.



	\subsection{The University of Toronto is already taking action on climate change}


In response to the settled science of climate change, the university has already taken a number of actions motivated by concern about climate change and a desire to reduce the university's greenhouse gas pollution impact.
The university's actions show climate change to be ``directly pertinent to higher education and academic research''.


In November 2009, the University of Toronto signed the ``Ontario Universities Committed to a Greener World'' pledge. In part, the pledge reads:
\begin{quote}
The Ontario university community is deeply aware of the challenges that face the world arising from climate change and the degradation of natural environments. Our universities accept this special responsibility on three scores: to assist in finding solutions to the challenges of environmental sustainability; to share knowledge about sustainability and climate change; and to incorporate, wherever possible, principles of sustainability into our own operations.\footcite[][]{OntarioPledge}
\end{quote}
The decision to divest from fossil fuels stocks would be wholly in keeping with these objectives.
Redirecting investment away from fossil fuels is a key part of solving the challenge of environmental sustainability.
Furthermore, by taking the lead and choosing to divest, the University of Toronto would send an important signal about how it views the future of energy.
This is also an opportunity to incorporate sustainability into university operations in a critical way, by having the university's values reflected in its stock portfolio.



\textbf{Policies and infrastructure decisions justified with reference to climate change}

The university has adopted an Environmental Protection Policy

\begin{vcom}
	This section needs to be filled out
\end{vcom}




\textbf{Academic programming}



In 2005, the university established a new Centre for Global Change Science (CGCS), which has since conducted exemplary research into climate change effects, as well as a wide array of public lectures focused around climate-related themes. The CGCS has hosted a number of talks as part of its Distinguished Lecture Series including:
\begin{itemize}
	\item Successes and Challenges for Biodiversity Science: Distribution Responses to Climate Change --- James Clark, Duke University, September 18, 2012
	\item High Altitude Climate Change: The Survival Struggle of our Earth’s Alpine Glaciers --- Andrew Bush, University of Alberta, October 16, 2012 
	\item Assessing Vegetation Responses and Feedbacks to Climate Change --- John Gamon, University of Alberta, November 6, 2012
	\item Cumulative Carbon Emissions and the Climate Mitigation Challenge --- Damon Matthews, McGill University, February 5, 2013; and
	\item Trees to Tailpipes: Natural and Anthropogenic Influences on Global Atmospheric Composition --- Colette Heald Massachusetts Inst. of Technology, March 5, 2013.\footcite[][]{DistinguishedLecturer}
\end{itemize}



The university also offers courses on climate-change-related themes, including:
\begin{itemize}
	\item Applied Climate Change
	\item Gaining Practical Skills for Climate Change Adaptation (UTSC Summer Institute 2013)
	\item Climate Change Law (LAW269H1S); and
	\item Climate Change and Human Health (CEM 406).
\end{itemize}



Climate change is certainly an area of active scholarly research, but that research does not question the fundamental connection between burning fossil fuel and warming the planet. 
Nor does it challenge the argument that climate change is likely to cause a great deal of social injury and human suffering.
Rather, the academic work being conducted on climate change at U of T reinforces the case for divestment.



	\subsection{Fossil fuel companies acknowledge the reality and danger of climate change}
	
\begin{vcom}
	David Naylor apparently found this a pursuasive argument before: Naylor noted, ``that there is no serious academic or social debate about tobacco's health effects – even tobacco manufacturers by now concede them.'' \url{http://www.thestar.com/news/2007/04/10/u_of_t_to_sell_off_its_tobacco_holdings.html}
	
	There are lots of greenwashing ads - possibly even from Shell - that we can use to demonstrate that the fossil fuel industry itself acknowledges climate change.
	
	
\end{vcom}


% END SECTION 2 - MILAN