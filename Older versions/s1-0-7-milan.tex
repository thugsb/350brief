% BEGIN SECTION 1 - MILAN
% Last updated by Milan Ilnyckyj 2013-06-08

		\section{Executive summary}
		\label{sec:ExecutiveSummary}


The governments of the world, including the government of Canada, have agreed that raising global temperatures to more than 2˚C above where they were before we started burning fossil fuels would be ``dangerous''.\footcite[][]{CopenhagenAccord}
If we are to achieve that objective, we cannot burn most of the fossil fuels that remain on the planet.
Based on hundreds of thousands of years of evidence on how the climate responds to greenhouse gasses (GHGs), we can calculate that avoiding a 2˚C increase means we must keep future GHG pollution to no more than 565 billion tonnes (gigatonnes) of carbon dioxide (\ce{CO2}).\footcite[For an excellent summary that is accessible to non-experts see: ][]{TerrifyingNewMath}
At the same time, we know that burning the world's proven reserves of coal, oil, and natural gas would produce 2,795 gigatonnes of \ce{CO2} --- nearly five times as much as it would be safe to burn.\footcite[][]{CTI2012}
That means we need to find a way to keep 80\% of the world's fossil fuel reserves unburned.\footcite[Another accessible summary of the issue can be found in: ][]{HotBackyard}



Climate change is a defining example of social injury.
Firms that produce fossil fuels do not bear any economic burden as a result of the many forms of harm they are imposing on other people, including agricultural impacts, sea level rise, human health impacts, and more severe extreme weather.
Likewise, those who use fossil fuels enjoy the benefits while imposing these costs on others.
In order to avoid severe global injury, the total quantity of fossil fuels burned by humanity must be capped at a level far below the level of fossil fuels available to be burned.



The business plans of fossil fuel companies do not take this objective into account.
They assume they can burn all of their proven reserves, along with any additional reserves they discover in unconventional areas like the arctic, the deep ocean, and Canada's bituminous sands.
Right now, we are adding about 30 gigatonnes of \ce{CO2} to the atmosphere each year, and the amount we add is increasing at a rate of about 3\%.
That means that we are on track to exceed the 565 gigatonne limit within 15 years.



Two big implications arise from this. 
First --- we need to find a way to meet the world's energy needs without burning most of the Earth's remaining fossil fuels. 
This requires a massive redirection of investment from fossil fuel energy sources to different energy sources that do not alter the climate.
Second --- the stockmarket value of fossil fuel companies is based on the assumption that they will be able to dig up and burn what they own.
If they are allowed to do this, the global effects will be catastrophic.
As such, much of the value of these companies is an illusion, based on an out-dated assumption that we can use the atmosphere forever as a free dumping ground for \ce{CO2}.

\begin{vcom}
The paragraph below is intended to answer the question ``Why should we pursue divestment? Is it a good use of my limited resources and time?'' raised by Stu's friend Sandy MacKay
\end{vcom}


The University of Toronto has an opportunity to do two things: to take part in the redirection of investment that is necessary to prevent climatic catastrophe, and to sell its shares in fossil fuel companies before the general public accepts that most of their reserves are unburnable.
In order to avoid dangerous climate change, the whole world must progressively replace its fossil fuel-intensive energy infrastructure with one that is compatible with climate stability.
This requires the massive redirection of investment away from new fossil fuel facilities like pipelines and unconventional oil and gas facilities.
The redirection must be toward cost-effective approaches to \ce{CO2} mitigation, including energy conservation and renewable energy deployment.\footcite[The consultancy McKinsey \& Company has studied and ranked global mitigation options, in terms of their cost, plausible deployment speed, and their capacity to mitigate greenhouse gas pollution. See: ][p. 8]{McKinseyCurve}
Divestment on the part of universities could play an important role in driving this shift.
That is true in the first instance because universities collectively have endowments and pension funds worth many billions of dollars.
Secondly and probably more importantly, university divestment would demonstrate that the `smart money' has become concerned enough about climate change to start taking substantial and meaningful action.
This would decrease investor confidence in the viability of new projects like coal-fired power plants and oil pipelines, while prompting other investors to think about how the reallocation of their capital can help with the global clean energy transition, while continuing to produce stable and attractive returns.




This brief explains in detail why divestment from fossil fuel companies is in keeping with the values of the university and why it is feasible and financially prudent.
This document is laid out to do three things: respond directly to the University of Toronto's divestment policy and procedures, provide a comprehensive and well-documented case supporting each major claim, and give people the opportunity to understand the main elements of the argument quickly.
Those seeking a relatively quick overview should examine the section headings in the table of contents, the executive summary, and the `short answers to common questions' provided.
Those seeking detailed information should read the entirety of the section concerned.
The petitioners who are supporting this proposal have affirmed that they have read and agree with this brief as a whole.



\begin{vcom}
We must find the best way using XeLaTex and the hyperref package to produce internal links within this document, including to the sections mentioned above.
\end{vcom}


Specifically, the authors and supporters of this brief call upon the University of Toronto to:
\begin{itemize}
	\item Make an immediate statement of principle, expressing its intention to divest its direct holdings of stock in fossil fuel companies within five years,
	\item Immediately stop making new investments in the industry,
	\item Instruct its investment managers to wind down the university’s existing direct stock holdings in the 200 fossil fuel companies listed in [LINK TO SECTION] over five years, and
	\item Divest from Royal Dutch Shell by the end of 2013.
\end{itemize}



By making the choice to divest, the University of Toronto can improve its future financial prospects, uphold its values, and take part in a necessary global transition away from \ce{CO2}-intensive forms of energy.
The University of Toronto's Statement of Institutional Purpose includes ``a resolute commitment to the principles of equal opportunity, equity and justice.'' \footcite{InstitutionalPurpose}
If future generations are to have equal opportunities, they cannot inherit a planet that has been impoverished by uncontrolled climate change.
Similarly, the principles of equity and justice forbid us from ignoring what we know about the harms of greenhouse gas pollution by continuing to impose risk and suffering on innocent people around the world and in future generations.



Across North America, the peer schools of the University of Toronto are considering divestment.
There include Harvard, Yale, Princeton, Stanford, and MIT.
[NUMBER] schools have already committed to divest.
By leading the way and divesting first, the University of Toronto can distinguish itself as being ahead of the pack on one of the major issues of the 21st century.
There is also a convincing argument that those who have benefitted substantially from the use of fossil fuels to date have a particular obligation to help the world transition onto a low-carbon development pathway.
The per-capita emissions of Canadians are extremely high, and both Canadians in general and the University of Toronto in particular occupy privileged places in the world.
By showing moral leadership, the University of Toronto can demonstrate the degree to which it understands the major trends now shaping the world, and its determination to end up on the right side of history, both in terms of upholding values and in terms of protecting its financial security.


% END SECTION 1 - MILAN