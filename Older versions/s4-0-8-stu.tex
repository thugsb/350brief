% BEGIN SECTION 4 - STUART
% Last updated by Stu Basden 2013-MAY-06



\section {Divestment is compatible with the university's fiduciary duties}

\begin{vcom}
  Stuart is taking over the drafting of this section. This may be a place where we can make especially good use of documents already prepared by other schools, including: (a) schools U of T regards as peers, like Harvard (b) schools in a similar position to U of T, like McGill and UBC and (c) schools that have already divested.
\end{vcom}

\begin{vcom}
  This section also needs to specifically address this section from the Procedures for Responding to Social and Political Issues with Respect to University Divestment: ``the extent and significance of the University’s investment in a particular entity.  Determination of whether investments are considered significant will depend on the committee's judgment of the relative magnitude of the University’s holdings both as a fraction of all University investments and in relation to the market capitalization of the entity under review.''
\end{vcom}

\subsection {From the U of T divestment policy}
(i) prudent investment. The University has a fiduciary duty to manage investments responsibly to maximize return on its investments within a policy risk tolerance as approved by Business Board from time to time.

...

The committee will consider the following guidelines in considering the appropriate response to any request:
\begin{itemize}
  \item the extent and significance of the University's investment in a particular entity. Determination of whether investments are considered significant will depend on the committee’s judgment of the relative magnitude of the University’s holdings both as a fraction of all University investments and in relation to the market capitalization of the entity under review.
  \item the degree to which the entity itself is involved in the undesirable activity.
\end{itemize}
Normally, activity is considered significant if more than ten percent of the entity's revenues are derived from the undesirable activity.

\subsection {There is no evidence of a divestment penalty for investors}

Several studies have attempted to quantify the financial consequences of divestment from the fossil fuel industry or heavy polluters in general.
In aggregate, these studies found no significant impact on investment risk in predictive models, nor a performance penalty in tests using historical data.
\begin{description}
  \item[Historical] The UN Environment Program Finance Initiative meta-analysis found there to be no evidence of penalty to screened portfolios \footcite{UNEPFI2007}
  \begin{vcom}
    I (Stu) didn't really understand the document. This comment should be clarified.
  \end{vcom}
  \item[Risk Based Assessment] The aperio group found that divesting from the ``Filthy Fifteen'' "increases absolute portfolio risk by only 0.0006\%, or about a half of one one-thousandth of a percent." Even divesting from the entire Fossil Fuel sector only results in a 0.0034\% return penalty. In other words, the portfolio does become riskier, but by such a trivial amount that the impact is statistically irrelevant. \footcite{Aperio2013}
\end{description}


\subsection {Market capitalization and value at risk}

\subsubsection {Stated policy objectives incompatible with valuation of fossil fuel reserves}

Fossil fuels may provide a hedge against other asset classes, but only in scenarios where unconstrained emissions lead to accelerated and possibly catastrophic warming. 
The international community is in broad agreement that this must be prevented from happening.




\subsubsection {Regulatory risk not adequately priced} 



As one scenario for the World Energy Outlook in 2012, the International Energy Agency assumes international cooperation to keep \ce{CO2} under 450ppm, which in their model constrains the likelihood of warming greater than 2°C to 55\%. 
This is in contrast to their baseline New Policies Scenario, which assumes modest reductions in the rate of emissions increase compared to the third scenario, Current Policies. 
Regarding the effect this policy environment would have on the price of fossil fuels they estimate:
\begin{slquote}Compared with the New Policies Scenario, the global oil price in the 450 Scenario in 2035 is \$25 per barrel lower and the coal price almost 40\% lower. The price for natural gas falls by 23\% in Europe and 4\% in North America.\footcite[][p. 257]{IEA2012}
\end{slquote}



For any scenario where emissions are constrained to keep warming under 2°C, market assumptions regarding the profitability of fossil fuel extraction are necessarily optimistic. 
Marginal projects will become unprofitable and returns to investors for even the most profitable projects will decline. 


\subsubsection {Potential for malinvestment in capital intensive, long-term projects}



The persistently high price of fuels on the world market in recent years has lead to unprecedented investment on the part of the fossil fuel industry in projects that were previously deemed too marginal to profitably develop. 
Development of unconventional hydrocarbon reserves such as tar sands, oil shale, offshore drilling in extremely deep water and the arctic, hydraulic fracturing and mountaintop removal coal mining entails extremely high capital investment. 
Scenarios in which carbon emissions are restricted sufficiently to keep global temperatures from rising more than 2°C would likely cripple the return on much of this investment.

In anticipating such a scenario, the fossil fuel industry has been pinning its hopes on the development of effective methods of carbon capture and sequestration (CCS). 
Despite tremendous investment in this technology on the part of both the private and public sectors, economically feasible sequestration of emissions at scales needed to mitigate climate change remains elusive. 
There are currently no commercial scale CCS projects in operation on the planet, and in 2008 Cambridge Energy Research Associates (CERA) predicted that it would be another two decades before CCS saw large-scale deployment. \footcite{CERACrossing}
According to the Carbon Tracker Initiative, even if CCS is deployed in line with an idealised scenario by 2050, this would only extend fossil fuel carbon budgets by 12-14\%, or just 4\% of total global reserves. \footcite{CTI2013}
It must be remembered that at the current rate of global carbon emissions, the entire budget of carbon emissions would be spent by the late 2020s, several years before large-scale CCS can be expected to come online. \footcite{CTI2012}

CCS has many other problems associated with it. For example, CCS would use extra energy, potentially as much as 40\% of the power generated by a power station. \footcite{GPCCS}
This reduces the efficiency of the power plant, both increasing financial costs, and increasing the amount of fuel needed per energy output, which in turn increases the problems associated with fossil fuel extraction.
Indeed, the increased cost of the energy provided by CCS-enabled power stations would likely be higher than the cost of energy from renewable sources, and so would almost certainly never be implemented. \footcite{SmartPlanetCCS}
Storing carbon underground is risky—safe and permanent storage of CO2 cannot be guaranteed, and even very low leakage rates could undermine any climate mitigation efforts. \footcite{GPCCS}
Finally, money spent on CCS will divert investments away from sustainable solutions to climate change, which the world will need to transfer to eventually, whether or not it burns all the available (non-renewable) fossil fuels.

Therefore, pinning our hopes on a non-existent technology, that is likely to both be more expensive and problematic than other energy sources, is a false hope.

\begin{vcom}
  We should probably include a longer explanation for why CCS is unlikely to solve the problem, given (a) the degree to which Canadian federal and provincial climate plans depend on it becoming cheap and effective very soon and (b) the likelihood that fossil fuel companies will point to CCS as a way of squaring their intent to burn all their reserves with the need to avoid dangerous climate change
\end{vcom}



\subsubsection {Fossil Fuel Reserves as Stranded Assets} 



Given the degree to which proven reserves of carbon exceed allowable emissions for sub-2°C warming, companies with fossil fuel reserves as their largest assets may be substantially overvalued under current market conditions. 
Stranded assets in the form of unburnable reserves and large liabilities incurred to develop those reserves combine to create a risk not only to equity, but to bondholders as well.
The Carbon Tracker Initiative reports that in 2012 the Fossil Fuel sector spent \$674 billion prospecting for new sources of carbon, sources which cannot be exploited if the 2C target is to be met. \footcite{CTI2013}



\subsubsection {Volatility of investor sentiment}



Current market capitalization of the fossil fuel industry rests in part on the assumption that the global investor class will continue to see the sector as a reliable investment even as damage from climate change becomes apparent. 
This assumption has been increasingly challenged from both outside and within the financial industry. 
Traditionally conservative-minded publications such as \emph{The Economist}, \footcite{EconomistUnburnable} \emph{Business Week} and the \emph{Financial Times} have published articles suggesting the fossil fuel sector is overvalued. 
In recent months, other voices within the financial industry such as investors groups and hedge fund managers have been increasingly sounding the alarm over the ``Carbon Bubble''. \footcite{JeremyGrantham} 
The Guardian recently reported
\begin{slquote}
The message to all the players across the financial chain, from ratings agencies through accountants, to actuaries, investment advisors and all the rest, is also obvious. If the regulators won’t do their job, do it for them. Jump, before you are pushed. \footcite{Guardian6Trillion}
\end{slquote}


\begin{vcom}
  The articles mentioned here should be cited - NEAL!!! (Stu couldn't find them all)
\end{vcom}



\subsection {Fossil Fuels represent a risk to the university's other investments}

Institutional investors, and universities in particular are unique in that they are often expected to plan financially on a timescale far longer than average. 
On timescales of 50 years or more, the consequences of unconstrained emissions are very likely to overshadow all other financial considerations.
\begin{vcom}
I'm not sure what else needs to go in this section --Stu
\end{vcom}


\subsection {Attractive substitutes exist for divested equities}

There are many attractive alternatives that could form substantial portions of the Universities' portfolio.
The renewable energy sector has enormous growth potential, and is starting to match even convention fossil-fuel energy prices (let alone unconventional).
Unsubsidised renewable energy is now cheaper than electricity from new-build coal- and gas-fired power stations in Australia, according to new analysis from research firm Bloomberg New Energy Finance. \footcite{BlombergAussieWind}
Unsubsidised renewable energy is now cheaper in Spain and Cuba. \begin{vcom}  Need citations!\end{vcom}
In March 2013, 100\% of the new energy on the U.S. grid was solar power. \footcite{SmartPlanetSolar100}



\subsection {Pensions and climate change}

\begin{vcom}
  It is virtually certain that the ad hoc committee will consider the special responsibility the university has toward current and future retirees. We need to make the case that (a) divestment is a sound financial decision for meeting those obligations and (b) if current employees want a decent retirement, dangerous climate change must be avoided.
\end{vcom}



% END SECTION 4 - STUART