% BEGIN APPENDIX I
% Last updated by Milan Ilnyckyj 2013-MAR-27

		\section{Appendix I: Issues With Respect to University Divestment}
		
	\subsection{Policy on Social and Political Issues With Respect to University Divestment}


	
March 4, 2008



\textbf{Preamble}

 

The University's core academic values include freedom of inquiry and open debate. As a general matter, the University does not take positions on social or political issues apart from those directly pertinent to higher education and academic research.  Instead, its role is to provide a forum within which issues can be studied carefully and debated vigorously. Given these values, the University will not consider any proposals for restrictions on its investments that require the institution to take sides in matters that are properly the subject of ongoing academic inquiry and debate.

 

As a corollary, the University's response to any petition regarding divestment must be governed by the fundamental place of diversity of opinion within its community. Except in those situations in which the University must settle on an answer to controversial questions about how best to achieve its academic mission, the University risks abandoning its core values if it takes sides in ongoing debates and is perceived to be advancing a specific political or social position.



\textbf{Principles}

 

In responding to questions about social and political issues with respect to University investment, it is acknowledged that first and foremost, maximizing economic return consistent with the University’s stated risk tolerance should be the criterion for purchase and sale of stock in all normal circumstances. In specific instances where the University's social responsibility as an investor is questioned, however, credible and effective procedures for responding should exist.

 

Responses should be based on the following principles:



(i) prudent investment. The University has a fiduciary duty to manage investments responsibly to maximize return on its investments within a policy risk tolerance as approved by Business Board from time to time.

 


(ii) the Yale University concept of social injury:

 

(a) Social injury is the injurious impact which the activities of a company are found to have on consumers, employees, or other persons, particularly including activities which violate, or frustrate the enforcement of, rules of domestic or international law intended to protect individuals against deprivation or health, safety, or basic freedoms; for purposes of this Policy, social injury shall not consist of doing business with other companies which are themselves engaged in socially injurious activities.

 

(iii) actions taken by the Canadian government or other national or international bodies with regard to the particular issue of concern.

 

Consideration of questions about social and political issues with respect to University investment must take into account applicable legislative requirements and government or University policy, as well as the legal standards applicable to prudent institutional investors.




\textbf{Advisory Committee}

 

The President will establish an \emph{ad hoc} committee of qualified individuals to review any investments claimed to be in conflict with University’s social and political positions and to advise the President on possible actions to be taken.  Chaired by a senior University officer designated by the President, the committee will consist of individuals with relevant expertise from among the teaching staff, students, administrative staff and alumni.  The Executive Committee of the Governing Council will be asked to approve the appointments on the recommendation of the President.

 

In making recommendations regarding the membership, the President will take into account any potential conflicts of interest proposed members might be expected to have with a view to minimizing such conflicts of interest on the part of committee members

 

The committee’s report and the President’s decision will be reported to the Governing Council through the Executive Committee.

 

The President will issue procedures regarding the implementation of this policy. The first such procedures are included here for information, and the President will review any substantive changes in those procedures with the Executive Committee of the Governing Council.
	
	
	
	\subsection{Procedures for Responding to Social and Political Issues with Respect to University Divestment}
	
	
	
January 2008


\textbf{Raising Issues}

 

Members of the University of Toronto community who wish to raise issues with regard to University investments that are in conflict with stated University policies may do so by:
\begin{itemize}
	\item preparing a convincing brief establishing the case; and
	\item presenting the evidence of general concern in the University community by collection of signatures.
\end{itemize}
Responsibility for initiating a request for University action regarding its investments rests with members of the University community. One or more individuals must prepare a fully documented brief identifying the social or political issue that they believe requires divestment.

 

When the brief has been fully prepared, the initiators of the request must secure evidence of support for their cause through the collection of at least 300 signatures endorsing the brief. Up to 200 of the signatures could come from a single constituency of the University community (for the purposes of these procedures, the constituencies are teaching staff, students, administrative staff, and alumni); the remaining 100 signatures must be from at least two other University constituencies with a minimum of 25 signatures from any individual constituency.  Each signatory must attest that he/she has read and agrees with the entire content of the brief.

 

When signatures have been added to the brief, the material is to be deposited with the office of the President.

 

\textbf{Response}

 

The administration will respond by establishing an \emph{ad hoc} review committee as specified by the policy.

 

This committee, chaired by a senior officer designated by the President, will consider the briefs. The committee may consult with investment and other experts as they deem necessary.

 

If the committee determines that the brief repeats previous submissions, or is vexatious or frivolous, it will cease its deliberations and recommend to the President that the brief be dismissed. If the brief is not repetitive, vexatious or frivolous, the committee shall complete its deliberations and provide its recommendation in writing to the President regarding the appropriate action to be taken by the University.

 

The committee will consider the following guidelines in considering the appropriate response to any request:
\begin{itemize}
 	\item the extent and significance of the University's investment in a particular entity.  Determination of whether investments are considered significant will depend on the committee’s judgment of the relative magnitude of the University’s holdings both as a fraction of all University investments and in relation to the market capitalization of the entity under review.
	\item the degree to which the entity itself is involved in the undesirable activity.
\end{itemize}
Normally, activity is considered significant if more than ten percent of the entity's revenues are derived from the undesirable activity.

 

The President will consider the recommendations and make the final decision.

 

\textbf{Reporting}

 

The written report and the President's decision will be provided to the Governing Council through the Executive Committee. Following receipt of the report and the President’s decision by the Governing Council, the report and decision will be provided to the petitioners.

\emph{Approved by Governing Council on March 4, 2008, replacing the Policy on Social and Political Issues with Respect to University Investment revised and approved by the Governing Council on December 14, 1994.}

% END APPENDIX I