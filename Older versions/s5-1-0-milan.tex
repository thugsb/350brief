% BEGIN SECTION 5 - MILAN
% Last updated by Milan Ilnyckyj 2013-MAR-30



	
	
		\section{Actions have been taken by the Canadian government and international bodies on this issue}



All three levels of Canada's government have taken action in response to the threat of climate change.

		\subsection{From the U of T divestment policy}


\begin{itquote}	
Responses should be based on the following principles:

...

(iii) actions taken by the Canadian government or other national or international bodies with regard to the particular issue of concern.
\end{itquote}


		
		\subsection{Federal government}
		
		
		
\begin{description}
	\item[Emission standards for passenger vehicles and light trucks] In November 2012, proposed regulations were released for vehicles beginning with the 2017 model year. 
	Average emissions from vehicles in 2025 are expected to be 50\% of those sold in 2008.
	\item[Heavy duty vehicles] In April 2012, the federal government released regulations for heavy duty vehicles beginning with the 2014 model year.
	\item[Coal-fired power plants] In September 2012, final regulations were introduced to limit emissions from the coal-fired electricity sector.
	\item[Renewable fuel requirement] As of December 2010, gasoline is required to contain an average of 5\% renewable content, with a 2\% requirement for diesel fuel.
	\item[Carbon capture and storage (CCS)] Canada's federal and provincial governments have committed a total of approximately \$3 billion in funding for CCS, which could lead to as many as five to six large-scale demonstration projects in Canada.
	\item[Agricultural greenhouse gases] Canada is contributing \$27 million toward the Global Research Alliance on Agricultural Greenhouse Gases, a group created to advance research, technology transfer, and adoption of beneficial management practices to mitigate agricultural greenhouse gases.
\end{description}		
		
		
		
		\subsection{Government of Ontario}
\begin{description}
	\item[Emission reduction targets] The Government of Ontario has legislated greenhouse gas emission reduction targets of 6\% below 1990 levels by 2014, 15\% below by 2020, and 80\% below by 2050.
	\item[Phasing out coal] The Government of Ontario has committed to phasing out coal-fired electricity generation by 2014.
	\item[Public transit investments] The Ontario government is contributing over \$9 billion to the Metrolinx Regional Transportation Plan.
	\item[Green Energy Act] Ontario's 2009 Green Energy Act created a system of feed-in tariffs to support the deployment of renewable energy options including solar photovoltaic, biogas, biomass, landfill gas, and wind power. 
It established a right for all renewable energy projects to be connected to the grid, streamlined the approval process for green energy projects, and began the implementation of a `smart' energy grid.
	\item[Forest protection] Ontario has protected roughly half of the province's boreal forest from mining and forestry, motivated in part by the forest's importance as a carbon sink.
	\item[Establishment of a Climate Change Secretariat] In 2008, the province created a permanent secretariat to coordinate its \emph{Climate Change Action Plan}.
	\item[Community Go Green Fund] The province provided \$6 million to 90 community groups in order to help charitable or environmental organizations, youth or cultural associations, educational institutions and Aboriginal communities reduce their carbon footprint. 
\end{description}	



	\subsection{City of Toronto}



\textbf{Climate Change Action Plan}

The city's Climate Change, Clean Air and Sustainable Energy Action Plan was unanimously adopted by Toronto City Council in July 2007. The city allocated over \$1 billion over the next five years to projects to reduce greenhouse gas emissions.\cite{TorontoEnvOff2008}

These commitments included:
\begin{itemize}
	\item \$67 million for the Better Building Partnership and the Sustainable Energy Funds, which are low interest revolving loan funds that support energy conservation and renewable energy
	\item \$136 million for energy retrofits to and installation of renewable energy systems on City owned buildings; 
	\item \$24 million for tree planting, in addition to the \$40 million a year operating budget for the city's Forestry Unit; 
	\item \$36 million to accelerate implementation of the City's Bike Plan; 
	\item \$20 million for the Live Green Toronto program which provides support for neighbourhood and community groups in taking action on Climate Change; 
	\item \$10 million for continued conversion of traffic signals to LED lights; 
	\item \$7 million for the Clean Roads to Clean Air street sweeping initiative; 
	\item \$186 million for water efficiency and improved energy efficiency in Toronto Water operations that will achieve an annual avoidance of an estimated 14,000 tonnes of greenhouse gas emissions; 
	\item \$21 million for methane gas capture and control at closed and operating landfills; 
	\item \$67 million to build anaerobic digestion facilities that will capture biogas from collected Green Bin organic materials and generate enough electricity to power an estimated 1,700 homes; 
	\item \$380 million to improve rapid transit services, such as, new light rapid transit lines, rapid transit routes for buses and an improved signalling system that will increase the capacity of the Yonge subway line; 
	\item \$400 million for the purchase of electric-hybrid buses; and 
	\item \$10 million plus for a range of initiatives including the Green Fleet Transition, the Eco-Roofs and Greenroofs Incentive programs, and support initiatives that promote production and consumption of locally grown food.
\end{itemize}

These investments are specifically justified with reference to the danger of climate change, with expected impacts on the city including rising mean temperatures, warmer winters, changes in disease vectors, changes in precipitation patterns, increased extreme weather, falling lake and stream levels, and rising sea levels.\cite{TorontoEnvOff2008}

The City of Toronto has also committed to specific greenhouse gas reduction targets, starting with the city's 1990 baseline level of approximately 22 million tonnes per year:\cite{TorontoAQandCC}
\begin{itemize}
	\item 6 percent by 2012 (1,320,000 tonnes per year)
	\item 30 percent by 2020 (6,600,000 tonnes per year)
	\item 80 percent by 2050 (17,600,000 tonnes per year)
\end{itemize}

Other actions taken by the city include:
\begin{description}
	\item [Adaptation] The city is making efforts to prepare for the impacts of climate change, through programs and policies including Toronto’s Heat Alert system and Hot Weather Response Plan, The Wet Weather Flow Master Plan, Green Roof Pilot Incentive Program, Deep Lake Water Cooling (Enwave), Peaksaver and Keep Cool Programs (Toronto Hydro), Green Development Standard, and Better Buildings Partnership.
	\item [Great Lakes Climate Change Policy Coordination] Along with 10 other cities in the Great Lakes region, Toronto is working to develop an international city-level policy on climate change.
	\item [Live Green Toronto] This  five-year, \$20-million dollar program is intended to promote and support actions by residents and community groups to reduce emissions, clean our air and protect our climate.
	\item [Landfill gas] The City of Toronto collects and burns landfill gases that are emitted at its three largest landfill sites: Keele Valley, Brock West and Beare. The city explains that: ``the process of collecting and incinerating landfill gases is crucial to the goal of combating the emission of greenhouse gases into the atmosphere''.\cite{TorontoAQandCC}
	\item [Greenhouse Gas and Air Emissions Inventory] In 2007, the city completed a combined greenhouse gas and air quality emissions inventory, with information about energy consumption and pollutants within the city. 
	\item [Concern about oil sands pipelines] The Toronto City Council has expressed its desire to review the application of Enbridge to reverse their Line 9 pipeline to carry diluted bitumen from the oil sands. 
	The city may apply to become an intervenor in the National Energy Board process.
\end{description}



\textsf{Can start with documents from Yasmin}


		
		\subsection{Actions taken by other national bodies}
		
Governments around the world have also been acting to mitigate the seriousness of climate change. In many cases, they have implemented significantly more ambitious policies than those enacted in Canada to date.

United States
\begin{description}
	\item[A]
	\item[B]
	\item[C]
\end{description}



United Kingdom
\begin{description}
	\item[A]
	\item[B]
	\item[C]
\end{description}



France
\begin{description}
	\item[A]
	\item[B]
	\item[C]
\end{description}



Germany
\begin{description}
	\item[A]
	\item[B]
	\item[C]
\end{description}



Japan
\begin{description}
	\item[A]
	\item[B]
	\item[C]
\end{description}




China
\begin{description}
	\item[A]
	\item[B]
	\item[C]
\end{description}



		\subsection{Actions taken by international bodies}
		
		

International efforts to address climate change have often been centred around the United Nations Framework Convention on Climate Change (UNFCCC), though many other international forums and organizations have also made efforts to address the issue.

United Nations Framework Convention on Climate Change (UNFCCC)
\begin{itemize}
	\item Signed in 1992, came into force in 1994 with 50 ratifications
	\item Objective: ``stabilize greenhouse gas concentrations in the atmosphere at a level that would prevent dangerous anthropogenic interference with the climate system''
	\item This has subsequently come to be understood to mean limiting warming to less than 2˚C
\end{itemize}
	
\textsf{Other sub-components of UNFCCC – adaptation funding, etc}




Canada has repeatedly endorsed the 2˚C limit for warming
\begin{itemize}
	\item The 2009 Copenhagen Accord - signed by Canada - recognizes ``the scientific view that the increase in global temperature should be below 2 degrees Celsius''
	\item \textsf{Other references to the limit}
	\item For the world to reach this goal, fossil fuels need to be phased out aggressively
\end{itemize}
	
% END SECTION 5 - MILAN