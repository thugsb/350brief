% BEGIN SECTION 3 - MIE
% Last updated by Milan Ilnyckyj 2013-MAY-25 with input from Jon Yazer

		\section{The activities of fossil fuel companies are socially injurious, and this social injury cannot be reasonably remedied through shareholder voice}

\begin{vcom}		
Since this whole section is largely about law, we need to (a) be completely clear and correct in what we say about the contents and interpretation of these laws (b) make completely defensible claims about the impacts of climate change and fossil fuel extraction generally (c) make sure to run this by some lawyers and people familiar with the U of T administration before we submit it
\end{vcom}

\begin{vcom}
Milan incorporated text supplied by Monica and Jon, but it was problematic in places. Problems have been flagged with comments in this document.
\end{vcom}

	\subsection{From the U of T divestment policy}

\begin{itquote}
Social injury is the injurious impact which the activities of a company are found to have on consumers, employees, or other persons, particularly including activities which violate, or frustrate the enforcement of, rules of domestic or international law intended to protect individuals against deprivation or health, safety, or basic freedoms
\end{itquote}



	\subsection{Social injury}


The primary activities of fossil fuel companies impose social injury on consumers, employees, or other persons.
The burning of a large portion of the world's reserves of fossil fuels would inflict great social injury through:
\begin{enumerate}
\item Impacts on agriculture
\item The inundation of coastal areas
\item Storms, droughts, other extreme weather
\item Increased risks to human health
\item Ecosystem collapse 
\item Threats to First Nations groups and indigenous cultures
\item Threats to the infrastructure of cities, including Toronto
\item The threat of abrupt and non-linear adverse climate impacts, arising from positive feedback effects and important thresholds in the climate system
\item Security implications
\end{enumerate}
In 2011, the National Roundtable on the Environment and the Economy (NRTEE) concluded that ``Climate change will be expensive for Canada and Canadians. Increasing greenhouse gas emissions worldwide will exert a growing economic impact on our own country, exacting a rising price from Canadians as climate change impacts occur here at home''.\footcite[][p.15]{NRTEEPrice}
They also concluded that: ``Global mitigation leading to a low climate change future reduces costs to Canada in the long term.''\footcite[][p.16]{NRTEEPrice}



The NRTEE highlighted how Canada and the rest of the world must choose between two futures: one in which action is taken (necessarily diminishing the profits and stockmarket value of fossil fuel companies) and another in which the world suffers the unmitigated consequences of climate change:
\begin{quote}
``Examining long-term economic costs of climate change to Canada raises the spectre of two futures: one where the world acts — and keeps global warming to 2°C by 2050 as world leaders have pledged — and one where it doesn't and climate change impacts grow and accelerate beyond targets. At slightly under 2°C of global warming, the economic costs of climate change to Canada in 2050 would be between \$21 billion and \$43 billion with no adaptive action taken; costs could be at the lower end of range if economic growth slowed as part of domestic mitigation or for other reasons. If the world acts to limit warming to 2°C, future costs could stabilize around this 2050 level since emissions growth would have been dampened and plateaued to reach this new global reality.''\footcite[][p.18]{NRTEEPrice}
\end{quote}
The sections below will elaborate on these forms of social injury, providing empirical evidence of the observed adverse impacts and predicted future risks from climate change.



	\subsubsection{Impacts on agriculture}



Agriculture is widely considered to be one of the most vulnerable systems to climate change in large part because its productivity is highly dependent on stable climate cycles and weather patterns.
For instance, in their Fourth Assessment Report, the IPCC concluded that some African countries agricultural production, including access to food, ``is projected to be severely compromised''.\footcite[][See: Synthesis report, Table SPM.2. Examples of some projected regional impacts. \url{https://www.ipcc.ch/publications_and_data/ar4/syr/en/spms3.html}]{IPCC2007}
Production from agriculture and forestry is expected to decline in Australia and New Zealand by 2030, and in Latin America ``[c]hanges in precipitation patterns and the disappearance of glaciers are projected to significantly affect water availability for human consumption, agriculture and energy generation.''



Changes in climate that will affect Canadian agricultural production include events such as heat waves and droughts, infestation of pests, and severe storms.
The Ontario Ministry of Agriculture and Food's website lists projected impacts including:
\begin{itemize}
	\item increased heat stress on livestock
	\item increased pest volumes and number of pest species
	\item modified geographical extent of agricultural production and locational shifts for growth of certain crops
	\item potential limitations on food processing expansions due to water quality and quantity issues
	\item financial challenges for rural municipalities exposed to extreme weather events and needing large infrastructure enhancements to cope with such events (bridges, roads, etc.)\footcite{OntarioCCandAg}
\end{itemize}
Climate change is also expected to do between \$2 and \$7 billion in damage to Canada's timber industry by 2050, ``through changes in pests, fires, and forest growth''.\footcite[][p.16]{NRTEEPrice}


Studies exploring economic approaches to dealing with climate change show that adaptation can provide one route to alleviate risks to Canada's agricultural sector.\footcite{Amiraslany2010}
However, extreme weather events, which are predicted to occur with increasing frequency as global temperatures rise are significant drivers of yield and impact changes and can therefore disrupt adaptation practices and threaten the health and prosperity of agricultural systems.\footcite{IsikDevadoss2006}
Indeed, possibilities of extreme weather events are often outside the scope of adaptation policies that outline strategies and recommendations for coping with less acute impacts such as those listed above.\footcite[See for instance:][]{Malcolm2012}


The ongoing drought in the United States provides a glimpse of what may become increasingly routine in a world altered by climate change.
Beginning in the spring of 2012, the drought originally affected areas along the plains and western mid-west regions of the country. 
As the drought continued, the federal government declared most of the central and southern U.S. wheat belt a natural disaster area. 
By July, the drought had reached such extreme conditions that officials in north-central Oklahoma declared a state of emergency on account of record-low reservoir conditions. 
Furthermore, the U.S Department of Agriculture (USDA) granted eligibility for low-interest emergency loans to wheat growers in four major wheat-growing states: Kansas, Colorado, Oklahoma and Texas. 
In early 2013, experts from the National Oceanic and Atmospheric Administration's Climate Prediction Center and the National Drought Mitigation Center at the University of Nebraska-Lincoln predicted that, despite various localized improvements, the drought is set to worsen in general through spring 2013, and will in fact expand to affect areas in California, Texas and Florida.\footcite[][]{NOAAteleconf2013}
Moreover, less than average snow accumulation in surrounding areas including the central and southern Rockies, results in a decrease of water flowing from streams and rivers to reservoirs, which adds to concerns about the potential for the drought to increase in scope.



Prolonged heat waves and periods of drought are projected to intensify globally concurrent with accelerating warming of global temperatures caused by the increase of GHG levels in the atmosphere.
The IPCC expects increased incidence of drought in asia, Australia and New Zealand, and Europe.
In North America, it expects ``[w]arming in western mountains... to cause decreased snowpack, more winter flooding and reduced summer flows, exacerbating competition for over-allocated water resources''.\footcite[][See: Synthesis report, Table SPM.2. Examples of some projected regional impacts. \url{https://www.ipcc.ch/publications_and_data/ar4/syr/en/spms3.html}]{IPCC2007}
Canada has experienced significant extreme heat and drought events in its recent history. 
For instance, six wide ranging and severe droughts took place over southern Ontario between 1936 and 1998. 
Two droughts, one in 1988 and the other ten years later in 1998, were both consistent with predictions in climate change scenarios for the Great Lakes region.\footcite[][]{Koshida2005}
Furthermore, the IPCC reports that climate change will continue to significantly threaten the sustainability of water supplies on a global scale, with water scarcity potentially impacting hundreds of millions of people by the end of the century.

\begin{vcom}		
I have done some searching around in IPCC documents, but cannot find anything like this claim. The only reference in the Word file from Jon and Monica as `IPCC (2007)'.
\end{vcom}

The International Food Policy Research Institute (IFPRI) finds that declines in yields of one critical world crop --- wheat --- will become greater the longer mitigation is delayed. 
Using a 2000 baseline, they project a decline in yield for rainfed wheat in the developed world of 1.3 percent by 2030, 4.2 percent by 2050, and 14.3 percent by 2080.\footcite[][p. 85]{Farming2050}
Up to 2050, climate change's impact on agriculture might be manageable to some extent; however, the IFPRI report concludes: ``Starting the process of slowing emissions growth today is critical to avoiding a calamitous post-2050 future''. (Gray et al., 2010).\footcite[][p. 86]{Farming2050}
While adaptation strategies may provide certain methods for dealing with select risks to agricultural production that are directly associated with climate change, mitigation in the form of reducing GHG emissions is essential to the long-term health and prosperity of the agricultural sector in Canada.



	\subsubsection{The inundation of coastal areas}
	
	
	
Across Canada, coastal communities, forests, agriculture, and fisheries are increasingly at risk from climate change.\
In the Natural Resources Canada report \emph{Climate Change Impacts and Adaptation: A Canadian Perspective}, ``sea level rise, resulting from thermal expansion of ocean waters and increased melting of glaciers and ice caps'' is identified as ``the main issue for marine regions''.\footcite[][p. xvi]{Lemmen2010}
The report explains that ``[o]verall, more than 7000 kilometres of Canada's coastline are considered highly sensitive to future sea level rise'' and that ``climate change [are]is expected to lead to a suite of biophysical and socio-economic impacts'' including coastal inundation, increased coastal erosion, saltwater intrusion into freshwater aquifers, reduced sea-ice cover, higher storm-surge flooding, higher sea surface temperatures, loss of coastal habitat, damage to coastal infrastructure, increased property loss, increased risk of disease, increased flood risks and potential loss of life, and loss of cultural resources and values.\footcite[][p. xvii]{Lemmen2010}
In 2011, the NRTEE projected that ``The costs of flooding from climate change could be between \$1 billion and \$8 billion per year by the 2050s''.\footcite[][p.16]{NRTEEPrice}


A closer look at the potential impacts of changing temperatures to the economic stability of Canada's Atlantic provinces illustrates some of these risks in more detail. 
The federal government report \emph{From Impacts to Adaptation: Canada in a Changing Climate 2007} provides a detailed analysis of both current and projected effects of climate change to different areas in Canada, including an extensive discussion on effects specific to the Maritimes region.\footcite[][]{ImpToAda}
The study projects major climatic changes in the region: ``By 2050, there would be a 2 to 4˚C increase in summer temperature... Future warming of 1.5 to 6˚C during winter can be anticipated''.\footcite[][p.131]{ImpToAda}
The study also concludes that: ``Rising sea level will result in flooding of higher, previously immune areas... and more frequent flooding of low-lying areas''.


These effects interact to have major economic and environmental consequences for the Maritime provinces.  
For instance, there is general consensus amongst fisheries scientists that the changing climate is going to significantly impact the Canadian fishing industry.

\begin{vcom}		
Can we get a source for this claim?
\end{vcom}

The harvesting of wild fish and shellfish, or the raising of these same species in anchored cages, is a major business in many Maritime coastal communities.  
However, warmer water temperatures could lead to the migration of various fish species to other areas. 
Similarly, increased land erosion causes greater amounts of sediment to fall into surrounding waters, which can disrupt the feeding and breeding patterns of many species of fish.



Many Maritime coastal communities such as those along the Bay of Fundy are also at risk due to the melting ice sheets, glaciers, and ice caps that are causing the steady and continuous rising of sea levels across the globe.\footcite[][]{PercyRisingTide}
Concurrent with rise in water levels, the land around the Bay of Fundy is subsiding by almost a foot every 100 years.
Taken together, these two effects could result in the rise of sea level along the Fundy coast of almost two feet by the end of the century.
This seemingly insignificant rise could in fact have a devastating effect on many local coastal areas.
Firstly, the increase in coastal erosion caused by rising sea levels will affect sensitive regions along the bay, including vulnerable areas in the northern edges as well as the large low-lying sections of the coast that are already well below sea level and that accommodate roads, railways, businesses, and residential areas. 
Moreover, the threat of more frequent severe storms poses risks to lands and buildings guarded by the many dykes along the coast, since these structures could prolong flooding by preventing seawater drainage in the increasingly likely case of extreme weather or heavy rainfall events. 
Taken together, threats to natural resources, increased frequency of extreme weather events, the acceleration of coastal erosion, and the threats to safety and stability of infrastructure due to rising sea levels, could have unparalleled consequences for Maritime communities. 

\begin{vcom}		
Potentially include sea level rise vulnerability map from Jon's Word file 
% (Elaborations_SocialInjury (JY_edits).docx)
\end{vcom}

From Vancouver to Halifax, communities across Canada face significant risks from sea-level rise and accompanying impacts.
In the long-term, unmitigated climate change risks causing Greenland and the West Antarctic ice sheet (WAIS) to melt.
According to the IPCC: ``Near-total deglaciation would eventually lead to a sea-level rise of around 7 m and 5 m from Greenland and the WAIS, respectively, with wide-ranging consequences including a reconfiguration of coastlines worldwide and inundation of low-lying areas, particularly river deltas''.\footcite[][See: "Deglaciation of West Antarctic and Greenland ice sheets" \url{https://www.ipcc.ch/publications_and_data/ar4/wg2/en/ch19s19-3-5-2.html}]{IPCC2007}
It goes on to say that: ``Widespread deglaciation would not be reversible except on very long time-scales, if at all''.
Sea level rise on this scale would constitute an exceptionally severe social injury --- with entire countries like Bangladesh and the Netherlands massively inundated, along with low-lying regions like Florida, New York City, and many of the world's other densely populated areas.
The IPCC identifies the ``threshold for near-total deglaciation'' at 3.2--6.2°C local warming (1.9--4.6°C global warming).
This is within the range of warming projections generated by several emission scenarios studied by the IPCC, corresponding to the absence of aggressive migitation action on the part of governments.\footcite[][See: "Projected climate change an its impacts" \url{https://www.ipcc.ch/publications_and_data/ar4/syr/en/spms3.html}"]{IPCC2007}


	
	\subsubsection{Storms, droughts, other extreme weather}



The Earth's changing climate has led to a notable rise in the number of great natural catastrophes that are driven by climate-related events over the past 25 years.\footnote{According to Munich Re, weather-related hazards can be described as a ``great natural catastrophes'' if it results in any one or a combination of the following attributes: i) number of fatalities exceeds 2,000; ii) number of homeless exceeds 200,000; iii) the country’s Gross Domestic Product (GDP) severely declines; and/or iv) the country is dependent on international aid} DISABLEDfootcite[See also:][]{EC2011b}

\begin{vcom}
This citation is included in the previous footnote, but it is not at all clear why: Environment Canada. 2011b. National Inventory Report: Greenhouse Gas Sources and Sinks in Canada 1990-2009.
\end{vcom}

Over the past 10 years, countries around the world have experienced approximately 785 natural catastrophes per year. 
During 2010 alone, a total of 950 natural catastrophes took place, nine-tenths of which were weather-related events such floods, hurricanes and storms.

\begin{vcom}
The source cited here doesn't say anything about this: Environment Canada. 2010a. News Release: Canada Announces a New Federal Sustainable Development Strategy. Environment Canada. Canada’s Domestic Action NDa. Available from \url{http://www.climatechange.gc.ca/default. asp?lang=En&n=4FE85A4C-1} Now available at: \url{http://www.ec.gc.ca/default.asp?lang=En&n=714D9AAE-1&news=6BEC86EF-BD5F-4208-8E46-7D40515C91C6}
\end{vcom}

Climate change is likely responsible, at least in part, for the rising frequency and severity of extreme weather events, such as floods, storms and droughts, since warmer temperatures tend to produce more violent weather patterns.\footcite[See: ][]{IPCCHurricane} 

\begin{vcom}
The source cited here is (Environment Canada, ND). I don't know what that means.
\end{vcom}

The Fourth Assessment Report of the IPCC (2007) asserts that changes in the frequency and intensity of extreme climate events will occur going into the future and will likely challenge human and natural systems to a much greater extent than natural changes in weather conditions.
These include hurricanes\footcite[][]{Knutson2004} and other extreme events including droughts, heat waves and floods.
The IPCC describes risks of extreme weather events as one of five special `reasons for concern' about climate change, along with risks to unique and threatened systems, the distribution of impacts and vulnerabilities (``those in the weakest economic position are often the most vulnerable to climate change''), aggregate impacts, and risks of large-scale singularities.\footcite[][See: "The long-term perspective" \url{https://www.ipcc.ch/publications_and_data/ar4/syr/en/spms5.html}"]{IPCC2007}
On hurricanes, the IPCC explains: ``Globally, estimates of the potential destructiveness of hurricanes show a substantial upward trend since the mid-1970s, with a trend towards longer storm duration and greater storm intensity, and the activity is strongly correlated with tropical sea surface temperature''.\footcite[][]{IPCCHurricane}
This accords with the basic science of hurricanes, which are driven by the latent heat in water vapour and gain strength from travelling over warmer water.

\begin{vcom}
Graphic could be included: Global trend for great natural catastrophes (as defined by Munich Re) since 1980.
\end{vcom}



In Canada, temperatures have warmed by an average of 0.24°C per decade, as indicated by data dating from the first official records of temperature conditions in 1948 through to 2010. 
This figure represents twice the global average, with temperature rises in the far north occurring at rates three times faster. 
The average national temperature in 2010 reached 3.0°C above normal, making it the hottest year on nationwide records.

\begin{vcom}
Sources would be welcome here
\end{vcom}


Precipitation levels in Canada have risen during the past half-century, with mean national levels increasing by about 12\%. 
This averages to about 20 more days of rain nation-wide compared with the 1950s. 
As climate change accelerates, and the rate of warming increases, the conditions for more volatile weather patterns become more common. 
Trends consistent with projections of climate models show increasing occurrence of extreme weather in Canada that can be traced back into the early 20th century. 
For instance, Figure X shows the increase in weather-related disasters in Canada over 100 years. 

\begin{vcom}
Do you have such a figure?
\end{vcom}

These are contrasted to the number of geophysical disasters (earthquakes and landslides) that took place over the same time period, which has remained fairly consistent.

\begin{vcom}
Citation needed
\end{vcom}



With an influx of extreme weather comes mounting costs for dealing with such events. 
The NRTEE projected that total costs associated with climate change could reach between \$21 billion and \$43 billion a year by the 2050s.\footcite[][p.15]{NRTEEPrice}
The range of estimates reflects uncertainty about the extent of action taken to reduce GHG emissions as well as other economic and population growth factors. 
Similarly, a report by the Institute for Catastrophic Loss Reduction (ICLR) for the Insurance Bureau of Canada (IBC) outlines trends of insured losses from severe weather and natural catastrophes both internationally and within Canada. 
The report reveals that financial impacts have ranged from between \$10 and \$50 billion dollars a year internationally since 2002, and with levels exceeding \$100 billion in 2011.\footcite[][p. 5]{TellingWeatherStory}
Within Canada, property insurance claims resulting from severe weather-related events from 2010-2012 have cost roughly \$1B a year.
The report outlines a number of specific examples of such claims, including:
\begin{itemize}
	\item A severe wind and thunderstorm that took place on in June of 2010 in and around Leamington in Southern Ontario caused approximately \$120 million worth of insured losses to both business and residential properties.
	\item Areas in Southern Alberta experienced a similar storm that resulted in excessive damage to private and commercial properties as well as automobiles that totalled over \$500 million in losses.
\end{itemize}
As the report details, claims resulting in both severe and smaller-impact weather events represent significant property damage for consumers, with losses driven in large part from aging sewage and water infrastructure that cannot handle the new higher precipitation levels; in fact, a rise in water levels for water-related insurance claims now ``surpass[es] fire as the number one cause of home insurance losses in many parts of the country''.\footcite[][p. 7]{TellingWeatherStory}
The report also details projections running through the 2050s of extreme weather events in Canada, including hot days per year, wildfires, hail and ice storms, tornadoes, and heavy rainfall events, and includes recommendations for dealing with the expansion of insurance-related losses nationwide.


In a report for Ceres --- a  network of investors, companies, and public interest groups seeking to accelerate and expand the adoption of sustainable business practices --- Sharlene Leurig evaluated the threat of climate change to insurers.
She concluded that: ``This changing climate will profoundly alter insurers' business landscape, affecting the industry's ability to price physical perils, creating potentially vast new liabilities and threatening the performance of insurers' vast investment portfolios''.\footcite[][p. 9]{ClimateRiskInsurers}
A climate that is changing increasingly rapidly is associated with severe weather, damage to infrastructure, and soaring costs.
This corresponds with the finding of the NRTEE that ``[g]lobal mitigation leading to a low climate change future reduces costs to Canada in the long term''.\footcite[][p. 16]{NRTEEPrice}


	\subsubsection{Increased risks to human health}



The impact of climate change on human health is no longer a contested issue, with major national and international organizations like the World Health Organization (WHO), Health Canada, the Centres for Disease Control and Prevention (CDC) and others recognizing both its existing impacts and its ongoing risks. 
The WHO, for example, asserts that ``the health effects of a rapidly changing climate are likely to be overwhelmingly negative, particularly in the poorest communities, which have contributed least to greenhouse gas emissions'' and acknowledges the increasingly damaging impact of an ever-warmer climate on numerous social and environmental health determinants, including clean air, water, food and shelter.\footcite[][]{WHOClimateHealth}



The negative effects of climate change on human health can be traced back almost forty years. 
For example, a 2009 WHO report entitled \emph{Global health risks: Mortality and Burden of Disease Attributable to Selected Major Risks} found that the modest increase in global temperature between 1970-2004 was the cause of over 140,000 deaths per year.

\begin{vcom}
The report is here, but I cannot find this figure: \url{http://www.who.int/healthinfo/global_burden_disease/GlobalHealthRisks_report_full.pdf}
Please provide a page number or corrected text.
\end{vcom}

The report explains that:
\begin{quote}
Potential risks to health include deaths from thermal extremes and weather disasters, vector-borne diseases, a higher incidence of food-related and waterborne infections, photochemical air pollutants and conflict over depleted natural resources. Climate change will have the greatest effect on health in societies with scarce resources, little technology and frail infrastructure. Only some of the many potential effects were fully quantifiable; for example, the effects of more frequent and extreme storms were excluded. Climate change was estimated to be already responsible for 3\% of diarrhoea, 3\% of malaria and 3.8\% of dengue fever deaths worldwide in 2004. Total attributable mortality was about 0.2\% of deaths in 2004; of these, 85\% were child deaths. In addition, increased temperatures hastened as many as 12 000 additional deaths; however these deaths were not included in the totals because the years of life lost by these individuals were uncertain, and possibly brief.\footcite[][p. 24]{WHOGlobalHealthRisks}
\end{quote}
A more recent study commissioned by 20 governments around the world estimates that this number has grown to approximately 400,000 climate-related deaths per year.
The report finds that ``Climate change has already held back global development; it is already a significant cost to the global economy.''\footcite[][p. 16]{DARACVM}
The report also explains that: ``Continuing today's patterns of carbon-intensive energy use is estimated, together with climate 
change, to cause 6 million deaths per year by 2030, close to 700,000 of which would be due to climate change. This implies that a combined climate-carbon crisis is estimated to claim 100 million lives between now and the end of the next decade.''\footcite[][p. 17]{DARACVM}
According to a Health Canada assessment, the most significant impacts to human health driven by changes in climate are linked to temperature stress, extreme weather, rodent and water-borne diseases, ultraviolet radiation, and air pollution.\footcite[][]{HHInACC} \footnote{Notably, this is one of many climate science reports produced by Canadian civil servants and essentially `buried' by the government of Stephen Harper. Planned coast-to-coast press conferences were cancelled, the report was released without publicity, and the report is not available through the Health Canada website.}
The report describes how ``the economic costs of extreme events in this country are rapidly increasing, as is the number of people affected by natural disasters'' and that ``[s]uch events and other climate-related hazards (e.g. smog, food-, water-, vector- and rodentborne diseases) continue to pose significant short- and long-term risks to the health and well-being of Canadians and their communities''.\footcite[][p. 432]{HHInACC}



It is generally accepted that the greatest impacts of ongoing climate change will be felt by people in low-income countries, as regions with weak health or governmental infrastructure will not have the capacity to respond to consequences of climate change appropriately. 
Particularly hard hit will be children, the elderly, people with illnesses or infirmities, and people with pre-existing medical conditions. 
As the WHO report details, a number of the fatal diseases already affecting these populations, such as diarrhea and other digestive ailments, malnutrition, and malaria, are ``highly climate-sensitive and are expected to worsen as the climate changes''.

\begin{vcom}
I cannot find this quote. Please provide a page number or change it. \url{http://www.who.int/healthinfo/global_burden_disease/GlobalHealthRisks_report_full.pdf}
\end{vcom}

Indeed, a growing body of literature is drawing attention to the incommensurate impacts of climate change on vulnerable and marginalized populations. 

\begin{vcom}
These citations are not provided with sufficient details to actually locate them. Please provide titles and ideally URLs: (Global Forum for Health Research 2010; Costello et al. 2009; Commission on Social Determinants of Health 2008).
\end{vcom}


In Canada, the relationship of health disparities to climate change impacts and adaptation is a newly emerging area of study. 
Recent reports predict that hotter city temperatures will lead to between five and 10 additional deaths per 100,000 people per year by 2050 as well as contribute to increasing pressure on Toronto hospitals due to sickness and other heat-related conditions that could swell associated costs to between \$3 million to \$8 million annually by the 2050s.\footcite[][p. 87]{NRTEEPrice}
The NRTEE concluded that climate change ``will lead to warmer summers and poorer air quality, resulting in increased deaths and illnesses in the four cities studied — Montréal, Toronto, Calgary, and Vancouver'' and that this will impose costs on the health care system of between \$3 million and \$11 million per year by the 2050s.\footcite[][p. 16]{NRTEEPrice}



	\subsubsection{Ecosystem collapse}
	
	
	\subsubsection{Threats to First Nations groups and indigenous cultures}
	
	
	\subsubsection{Threats to the infrastructure of cities, including Toronto}
	
	
	\subsubsection{Abrupt and non-linear adverse climate impacts}



	\subsubsection{Security implications}
	
	
	
A number of major analyses have been focused on the likely global security implications of climate change.
In 2008, a National Intelligence Assessment was assembled by 16 U.S. intelligence agencies.
While the report is classified, the chairman stated publicly that climate change could disrupt US access to raw materials, create millions of refugees, and cause water shortages and damage from melting permafrost.\footcite[][]{Craven}
A 2003 report commissioned by the Pentagon considered some of the more dramatic possible warming scenarios and concluded that:
\begin{quote}
In short, while the US itself will be relatively better off and with more adaptive capacity, it will find itself in a world where Europe will be struggling internally, large number so [sic] refugees washing up on its shores and Asia in serious crisis over food and water. Disruption and conflict will be endemic features of life.\footcite[][p. 22]{AbruptCCScenario}
\end{quote}
It also argues that: ``with inadequate preparation, the result [of abrupt climate change] could be a significant drop in the human carrying capacity of the Earth’s environment''.\footcite[][p. 1]{AbruptCCScenario}
A report prepared for the Center for Naval Analysis --- produced by a ``blue-ribbon panel of retired admirals and generals from the Army, Navy, Air Force, and Marines'' --- calls climate change ``potentially devastating''.\footcite[][p. 3]{NationalSecurityCC}
A joint report from the Center for Strategic and International Studies and the Center for a New American Security describes how current projections from climate models are ``too conservative'' and that ``at higher ranges of the [warming] spectrum, chaos awaits''.\footcite[][p. 78]{AgeOfConsequences}
The report also highlights the need for urgent action to reduce emissions: ``An effective response to the challenge of global warming cannot be spread out across the next century, but rather must be set in place in the next decade, in order to have any chance to meaningfully alter the slope of the curves one sees in the IPCC report''.\footcite[][p. 78]{AgeOfConsequences}



In 2012, the U.S. National Academy of Sciences published a report on: ``Climate and Social Stress: Implications for Security Analysis''.\footcite[][]{SocialStress}
The report concludes that:
\begin{quote}
Anthropogenic climate change can reasonably be expected to increase the frequency and intensity of a variety of potentially disruptive environmental events— slowly at first, but then more quickly. Some of this change is already discernible. Many of these events will stress communities, societies, governments, and the globally integrated systems that support human well-being.\footcite[][p. S-2]{SocialStress}
\end{quote}
And that:
\begin{quote}
It is prudent to expect that over the course of a decade some climate events—including single events, conjunctions of events occurring simultaneously or in sequence in particular locations, and events affecting globally integrated systems that provide for human well-being—will produce consequences that exceed the capacity of the affected societies or global systems to manage and that have global security implications serious enough to compel international response. It is also prudent to expect that such consequences will become more common further in the future.\footcite[][p. S-4]{SocialStress}
\end{quote}
All told, the report describes in great detail the ways in which climate change is a national security issue for the United States, as well as a threat to international peace and security.



	\subsection{The harm caused is inherent to the primary business of fossil fuel companies}



All the social injuries described above are imposed on innocent parties by fossil fuel companies in the course of their fundamental business activity of digging up coal, oil, and gas.
These harms are inseparable from the continuation and expansion of these core business activities.
As a result, shareholder voice is not an effective strategy for mitigating these harms. 
The value of these companies also reflects the assumption that these reserves will be extracted and burned.
The University of Toronto's investments in these companies increase the amount of harm that will arise as a result of climate change.



Divestment is the only way for the University of Toronto to avoid contributing financially to the fossil fuel industry, and by extension, to the socially injurious impacts delineated above.
Besides divestment, another approach to socially responsible investment is to try to alter a firm’s behaviour by applying pressure through shareholder voice. 
However, the harmful activities (extracting and selling fossil fuels) are inherent to the primary business of fossil fuels companies in which the university is invested.  
In this sense, investments in fossil fuel companies closely parallel investments in tobacco companies; in both cases, the problem is the primary product being produced by the industry.



For example, Shell Canada lists its business activities as follows: ``Shell Canada's Upstream businesses explore for and extract natural gas, and market and trade natural gas and power. Our Downstream business refines, supplies, trades and ships crude oil worldwide and manufactures and markets a range of products, including fuels, lubricants, bitumen and liquified petroleum gas (LPG) for home, transport and industrial use.''\footcite[][]{ShellAtAGlance}
ExxonMobil describes its upstream and downstream activities similarly.\footcite[][]{ExxonWhatWeDo}



Given the centrality of oil and natural gas extraction, as well as the refinement and sale of these resources to the business models of these companies, shareholder voice would not be an effective method to address social injury since the companies could not abandon the socially injurious activity without dissolving their existing business models.  
Moreover, the market value of these companies reflects an assumption that their reserves will be extracted and burned.  
Therefore, it would be unreasonable for the University of Toronto to expect to be able to alter the socially injurious activities of these companies while holding onto its investments in the fossil fuel industry.  
Thus, divestment is the only appropriate response for the University of Toronto to adopt in order to dissolve any financial complicity in the fossil fuels industry’s socially injurious activities.  



	\subsection{The business activities of these companies frustrate the enforcement of the rules of domestic and international law intended to protect individuals against deprivation of health, safety, and basic freedoms}



The socially injurious activities of fossil fuel companies frustrate the enforcement of rules of domestic and international law intended to protect individuals against deprivation of health, safety and basic freedoms.  



\begin{vcom}
	The section below needs a fair amount of work. We need to go beyond pointing to very general rights in the constitution and international legal documents and work toward a convincing legal case.
\end{vcom}



First, these activities undermine the \emph{Canadian Charter of Rights and Freedoms}.  
Section 7 states ``the right to life, liberty and security of the person and the right not to be deprived thereof except in accordance with the principles of fundamental justice.''   
Since life and security of the person depend on a healthy environment, implicit in this statement is the right to a healthy environment.  
By contributing to increasingly dangerous global climate change, the activities of companies in the fossil fuels industry undermine the right to life by depriving people of the benefits of a healthy environment.  



In addition, numerous pieces of Canadian environmental legislation explicitly recognize and seek to protect the right to a healthful environment.
The \emph{Ontario Environmental Bill of Rights} (1993) recognizes the ``inherent value of the natural environment'' and states that ``the people of Ontario have the right to a healthful environment'' and ``have as a common goal the protection, conservation and restoration of the natural environment for the benefit of present and future generations.''  
The purposes of the act are:
\begin{enumerate}
	\item to protect, conserve and, where reasonable, restore the integrity of the environment by the means provided in this Act;
	\item to provide sustainability of the environment by the means provided in this Act; and
	\item to protect the right to a healthful environment by the means provided in this Act.  1993, c. 28, s. 2 (1).
\end{enumerate}

\begin{vcom}
	Legislation cited here should be cited using the same \LaTeX footnote system employed through the rest of the brief.
\end{vcom}


The above purposes include the following:
\begin{enumerate}
	\item The prevention, reduction and elimination of the use, generation and release of pollutants that are an unreasonable threat to the integrity of the environment.
	\item The protection and conservation of biological, ecological and genetic diversity.
	\item The protection and conservation of natural resources, including plant life, animal life and ecological systems.
	\item The encouragement of the wise management of our natural resources, including plant life, animal life and ecological systems.
	\item The identification, protection and conservation of ecologically sensitive areas or processes.  1993, c. 28, s. 2 (2). 
\end{enumerate}



The activities of fossil fuel companies frustrate all of the above purposes by contributing to climate change, thereby undermining the right to a healthy environment of the people of Ontario.

\begin{vcom}
	More specific examples of how each of these purposes are challenged by climate change should be added.
\end{vcom}


Environmental laws for other provinces of Canada recognize and seek to protect the same right to a healthy environment.  



For example, Part 1, section 6 of the \emph{Yukon Environment Act} states that: ``The people of the Yukon have the right to a healthful natural environment.''   
In accordance with this right, the Act seeks to protect the environment of the Yukon by providing an appropriate process to assess the environmental effects of projects and activities in the Yukon or that may have effects in the Yukon. Similarly, the \emph{Northwest Territories Environmental Rights Act} recognizes that ``the people of the Northwest Territories have the right to a healthy environment and a right to protect the integrity, biological diversity and productivity of the ecosystems in the Northwest Territories'' and establishes the means by which individuals can act to protect the environment from harm. By pursuing the extraction of fossil fuels, the companies in question undermine the right to a healthy environment that these acts articulate and protect. Finally, Quebec’s \emph{Environmental Quality Act} states that, ``Every person has a right to a healthy environment and to its protection, and to the protection of the living species inhabiting it, to the extent provided for by this Act and the regulations, orders, approvals and authorizations issued under any section of this Act and, as regards odours resulting from agricultural activities, to the extent prescribed by any standard originating from the exercise of the powers provided for in subparagraph 4 of the second paragraph of section 113 of the Act respecting land use planning and development'' (chapter A-19.1).  

\begin{vcom}
Should we be including legislation from provinces other than Ontario? If so, it should be cited in the standard way for the brief.
\end{vcom}

The activities of the fossil fuels industry in Canada also violate the constitutional and treaty rights of Canada's First Nations.
These violations arise both from the specific impact of fossil fuel development projects --- such as the oil sands --- and from the inevitable consequences of burning fossil fuels.
Rights that are being violated include the right to consultation and accommodation; the right to waters and land and to fish, hunt and trap; and the aboriginal rights affirmed in Canada's constitution.



Keepers of the Athabasca member Vivienne Beisel explains how the oil sands development has violated Treaty 8 and the Constitution: ``The cumulative impacts of oil sands development has all but destroyed the traditional livelihood of First Nations in northern Athabasca watershed. 
The law is clear that First Nations must be consulted whenever the province contemplates action that may negatively affect Aboriginal and treaty rights...
The province has continued to issue approvals for new developments without obtaining their consent or consulting with First Nations in a meaningful and substantial way. 
This is in direct breach of Treaty 8 First Nations' treaty-protected Aboriginal rights to livelihood, and thus a violation of s.35(1) of the Constitution'''.


\begin{vcom}
	MIE HAD THIS TEXT, BUT I DON'T THINK IT ADDS TO OUR CASE ``and Articles 26 and 27 of the United Nations Declaration on the Rights of Indigenous Peoples, an international agreement which Canada, along with three other nations, has refused to sign.'' We need to add information on how the rights of aboriginals in Ontario specifically risk being violated by fossil fuel companies.
\end{vcom}



Finally, the activities of the fossil fuels companies in which the University of Toronto is invested frustrate international law.  
First, Article 3 of the \emph{Universal Declaration of Human Rights} states that ``Everyone has the right to life, liberty and security of person.''   
The right to life is a precondition to all other fundamental human rights.  
The activities of companies in the fossil fuels industry undermine the right to life by depriving people of the benefits of a healthy environment.  
In addition, the \emph{Hague Declaration on the Environment} (1989), to which Canada is a signatory, makes the link between the right to life and the harmful change effects of climate change explicit: ``The right to live is the right from which all other rights stem.  
Guaranteeing this right is the paramount duty of those in charge of all States throughout the world.  
Today, the very conditions of life on our planet are threatened by the severe attacks to which the earth’s atmosphere is subjected.''   
In signing onto this Declaration, Canada recognized the reality of the threat to human life posed by climate change and pledged to take measures to address that threat.  
The University of Toronto’s investment in fossil fuels frustrates any efforts Canada has taken or may take in the future to address the problem of climate change by supporting the companies that most significantly contribute to the problem.	

\begin{vcom}
	Again, this needs to be made more specific both in terms of impacts and in terms of laws. Where possible, we want to be able to point to specific documented effects that contravene specific pieces of applicable legislation.
\end{vcom}


The activities of fossil fuel companies are also at odds with the fundamental objective of the \emph{United Nations Framework Convention on Climate Change} (UNFCCC), which was ratified by Canada and which entered into force on March 21st 1994.
The UNFCCC affirms the intention of signatories to achieve ``stabilization of greenhouse gas concentrations in the atmosphere at a level that would prevent dangerous anthropogenic interference with the climate system''.
Countries including Canada have since adopted a threshold of 2˚C of global temperature increase above pre-industrial levels as constituting `dangerous' climate change.
Achieving this objective requires that most of the reserves of fossil fuel companies be left unburned underground.
It also requires the abandonment of projects intended to extract unconventional reserves of fossil fuels, through activities including oil and gas drilling in the arctic, exploitation of the oil sands, and extraction of previously inaccessible oil and gas reserves through hydraulic fracturing.



	\subsection{Why fossil fuels are like tobacco}


\begin{vcom}
		Get a working anchor for the link in part 3
\end{vcom}

% \hypertarget{TobaccoPrecedentInjury}



In 2007, the University of Toronto decided to divest from tobacco companies, after determining that the case to do so was consistent with university policies.
There are several important ways in which the tobacco precedent is relevant to fossil fuel divestment.



Firstly, the scientific case demonstrating the harm caused by tobacco strengthened progressively over the span of decades.
Companies were initially willing to challenge these claims, but the weight of evidence eventually made their case untenable.
Similarly, the evidence demonstrating the seriousness of anthropogenic climate change has now progressed beyond the point where it can be considered a subject of ongoing academic inquiry and debate.


Secondly, in both the cases of tobacco and fossil fuels the problem is the primary product being produced by the industry.
Just as it would be ineffective to use shareholder voice to try to convince a tobacco company to stop producing and selling tobacco, it is implausible that the university could use shareholder activism to convince fossil fuel companies to desist from activities that create and facilitate major greenhouse gas pollution.



Thirdly, both investments in tobacco and fossil fuels challenge the core values of the university.



\begin{vcom}
	Get a working link to the tobacco precedent section in part 6 here
\end{vcom}



% The tobacco precedent also demonstrates that \hyperlink{TobaccoPrecedentFinance}{the university can divest from Shell without suffering financial harm}.




\begin{vcom}
Elaborate, and add citations related to the tobacco precedent. Also, consider whether any other previous U of T divestment campaigns can be cited as useful precedents.
\end{vcom}



% END SECTION 3 - MIE