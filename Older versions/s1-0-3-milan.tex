% BEGIN SECTION 1 - TBA
% Last updated by Milan Ilnyckyj 2013-APR-24

		\section{Executive summary}

\begin{vcom}
This will be filled in once we are reasonably happy with the rest of the brief. It will certainly be the section that the most people read and which gets the most attention, so we should make a special effort to make the case compelling here.
\end{vcom}


\begin{vcom}
	Citations (and probably internal links to later parts of the brief) should be added to this section.
\end{vcom}


The governments of the world, including the government of Canada, have agreed that raising global temperatures to more than 2˚C above where they were before we started burning fossil fuels would be ``dangerous''. [CITE COPENHAGEN ACCORD]
If we are to achieve that objective, we cannot burn most of the fossil fuels that remain on the planet.
Based on hundreds of thousands of years of evidence on how the climate responds to greenhouse gasses (GHGs), we can calculate that avoiding a 2˚C increase means we must keep future GHG pollution to no more than 565 billion tonnes (gigatonnes) of carbon dioxide (CO2). [CITE]
At the same time, we know that burning the world's proven reserves of coal, oil, and natural gas would produce 2,795 gigatonnes of CO2 --- nearly five times as much as it would be safe to burn. [CITE]
That means we need to find a way to keep 80\% of the world's fossil fuel reserves unburned.



The business plans of fossil fuel companies do not take this objective into account.
They assume they can burn all of their proven reserves, along with any additional reserves they discover in unconventional areas like the arctic, the deep ocean, and Canada's bituminous sands.
Right now, we are adding about 30 gigatonnes of CO2 to the atmosphere each year, and the amount we add is increasing at a rate of about 3\%.
That means that we are on track to exceed the 565 gigatonne limit within 15 years.



Two big implications arise from this. First --- we need to find a way to meet the world's energy needs without burning most of the Earth's remaining fossil fuels. 
This requires a massive redirection of investment from fossil fuel energy sources to different energy sources that do not alter the climate.
Second --- the stockmarket value of fossil fuel companies is based on the assumption that they will be able to dig up and burn what they own.
If they are allowed to do this, the global effects will be catastrophic.
As such, much of the value of these companies is an illusion, based on an out-dated assumption that we can use the atmosphere forever as a free dumping ground for CO2.



The University of Toronto has an opportunity to do two things: to take part in the redirection of investment that is necessary to prevent climatic catastrophe, and to sell its shares in fossil fuel companies before the general public accepts that most of their reserves are unburnable.



This brief will explain in detail why divestment from fossil fuel companies is in keeping with the values of the university and why it is feasible and financially prudent.
Campaigns to divest from fossil fuel companies are ongoing at more than 250 universities across North America, including Harvard, Yale, McGill, and the University of British Columbia.
[NUMBER] schools have already committed to divest.



By making the same choice, the University of Toronto can improve its future financial prospects, uphold its values, and take part in a necessary global transition away from CO2-intensive forms of energy.



% END SECTION 1 - TBA