	\subsubsection{United Kingdom}

The Climate Change Act 2008 made the UK the first country in the world to have a legally binding long-term framework to cut carbon emissions.\footcite[][]{ClimateConvention2009}  It introduced a long-term framework for managing emissions through a system of national carbon budgets: caps on the total quantity of greenhouse gases permitted in the UK over a specified time.\footcite[][]{ClimateConvention2009} Each carbon budget covers a five year period, with the first three carbon budgets running from 2008 to 2012, 2013-2017 and 2018-2022.\footcite[][]{ClimateConvention2009} During these periods, emissions must be reduced (from 1990 levels) by 22\%, 28\% and 34\% respectively.\footcite[][]{ClimateConvention2009} 

The Act also created a framework for building the UK’s ability to adapt to climate change, including:
\begin{itemize}
	\item a UK-wide Climate Change Risk Assessment that must take place every five years
	\item a National Adaptation Programme which must be put in place and reviewed every five years to address the most pressing climate change risks to the UK
	\item a mandate giving the government the power to require ‘bodies with functions of a public nature’ and ‘statutory undertakers’ (eg water and energy utilities) to report on what they are doing to address the risks posed by climate change to their work.\footcite[][]{ClimateConvention2009} 
\end{itemize}


The UK Department of Energy \& Climate Change has set the following national policies and strategies for combating climate change:
\begin{itemize}
	\item setting carbon budgets to limit the amount of greenhouse gases the UK is allowed to emit over a specified time
	\item using statistics on greenhouse gas emissions and further evidence, analysis and research to inform energy and climate change policy
	\item using the EU Emissions Trading Scheme (EU ETS) to meet over 50\% of the UK’s carbon emissions reduction target between now and 2020
	\item using a set of values for carbon to make sure project and policy appraisals account for their climate change impacts
	\item using the 2050 Calculator to let policy makers and the public explore the different options for meeting the 2050 emissions reduction targets.\footcite[][]{UKgovnt} \footcite[][]{EUETS} 
 
\end{itemize}



The UK Department of Energy \& Climate Change is taking action to support businesses and the public to use energy more efficiently by:
\begin{itemize}
	\item reducing demand for energy with smart meters and other energy-efficient measures for industry, businesses and the public sector
	\item reducing emissions by improving the energy efficiency of properties through the Green Deal
	\item providing incentives for public and private sector organizations to take up more energy-efficient technologies and practices through the CRC Energy Efficiency Scheme 
	\item reducing greenhouse gases and other emissions from transport
	\item reducing greenhouse gas emissions from agriculture. \footcite[][]{UKgovnt}
\end{itemize}



The UK Department of Energy \& Climate Change is investing in low-carbon technologies by:
\begin{itemize}
	\item taking action to increase the use of low-carbon technologies and creating an industry for carbon capture and storage
	\item reducing emissions from the power sector and encouraging investment in low-carbon technologies by reforming the UK’s electricity market
	\item providing over £200 million of funding for innovation in low-carbon technologies from 2011 to 2015.\footcite[][]{UKgovnt}
\end{itemize}



The UK Department of Energy \& Climate Change is publicly reporting carbon emissions from businesses and the public sector by:
\begin{itemize}
	\item encouraging corporate reporting of greenhouse gas emissions
	\item asking English local authorities to measure and report their greenhouse gas emissions. \footcite[][]{UKgovnt}
\end{itemize}
