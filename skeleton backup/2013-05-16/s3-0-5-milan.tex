% BEGIN SECTION 3 - MIE
% Last updated by Milan Ilnyckyj 2013-APR-24

		\section{The activities of fossil fuel companies are socially injurious, and this social injury cannot be reasonably remedied through shareholder voice}

\begin{vcom}		
We may want to come up with a snappier title for this section. Also, since this whole section is largely about law, we need to (a) be completely clear and correct in what we say about the contents and interpretation of these laws (b) make completely defensible claims about the impacts of climate change and fossil fuel extraction generally (c) make sure to run this by some lawyers and people familiar with the U of T administration before we submit it
\end{vcom}

	\subsection{From the U of T divestment policy}

\begin{itquote}
Social injury is the injurious impact which the activities of a company are found to have on consumers, employees, or other persons, particularly including activities which violate, or frustrate the enforcement of, rules of domestic or international law intended to protect individuals against deprivation or health, safety, or basic freedoms
\end{itquote}



	\subsection{Social injury}


The primary activities of fossil fuel companies impose social injury on consumers, employees, or other persons.
The burning of a large portion of the world's reserves of fossil fuels would inflict great social injury through:
\begin{enumerate}
\item The impact of climate change on agriculture
\item The inundation of coastal areas
\item Storms, droughts, other extreme weather
\item The spread of tropical disease
\item Ecosystem collapse 
\item Threats to First Nations groups and indigenous cultures
\item Threats to the infrastructure of cities, including Toronto
\item The threat of abrupt and non-linear adverse climate impacts, arising from positive feedback effects and important thresholds in the climate system
\end{enumerate}

\begin{vcom}
	Monica has prepared additional text to add to each of these points. As discussed at the 23 April 2013 planning meeting, she and Jon will revise that text until they are happy with it and then send it to Milan to incorporate into the \LaTeX file
\end{vcom}

	\subsection{The harm caused is inherent to the primary business of fossil fuel companies}
	
The harms from fossil fuel extraction and burning arise directly and inescapably from the core business activities of fossil fuel companies.
As a result, shareholder voice is not an effective strategy for mitigating these harms. 
The value of these companies reflects the assumption that these reserves will be extracted and burned.
The University of Toronto's investments in these companies increase the amount of harm that will arise as a result of climate change.



Divestment is the only way for the University of Toronto to avoid contributing financially to the fossil fuel industry, and by extension, to the socially injurious impacts delineated above.
Besides divestment, another approach to socially responsible investment is to try to alter a firm’s behaviour by applying pressure through shareholder voice. 
However, the harmful activities (extracting and selling fossil fuels) are inherent to the primary business of fossil fuels companies in which the university is invested.  
In this sense, investments in fossil fuel companies closely parallel investments in tobacco companies; in both cases, the problem is the primary product being produced by the industry.



For example, Shell Canada lists its business activities as follows: ``Shell Canada's Upstream businesses explore for and extract natural gas, and market and trade natural gas and power. Our Downstream business refines, supplies, trades and ships crude oil worldwide and manufactures and markets a range of products, including fuels, lubricants, bitumen and liquified petroleum gas (LPG) for home, transport and industrial use.'' ExxonMobil describes its upstream and downstream activities similarly.   



Given the centrality of oil and natural gas extraction, as well as the refinement and sale of these resources to the business models of these companies, shareholder voice would not be an effective method to address social injury since the companies could not abandon the socially injurious activity without dissolving their existing business models.  
Moreover, the market value of these companies reflects an assumption that their reserves will be extracted and burned.  
Therefore, it would be unreasonable for the University of Toronto to expect to be able to alter the socially injurious activities of these companies while holding onto its investments in the fossil fuel industry.  
Thus, divestment is the only appropriate response for the University of Toronto to adopt in order to dissolve any financial complicity in the fossil fuels industry’s socially injurious activities.  



	\subsection{The business activities of these companies frustrate the enforcement of the rules of domestic and international law intended to protect individuals against deprivation of health, safety, and basic freedoms}



The socially injurious activities of fossil fuel companies frustrate the enforcement of rules of domestic and international law intended to protect individuals against deprivation of health, safety and basic freedoms.  



\begin{vcom}
	The section below needs a fair amount of work. We need to go beyond pointing to very general rights in the constitution and international legal documents and work toward a convincing legal case.
\end{vcom}



First, these activities undermine the \emph{Canadian Charter of Rights and Freedoms}.  
Section 7 states ``the right to life, liberty and security of the person and the right not to be deprived thereof except in accordance with the principles of fundamental justice.''   
Since life and security of the person depend on a healthy environment, implicit in this statement is the right to a healthy environment.  
By contributing to increasingly dangerous global climate change, the activities of companies in the fossil fuels industry undermine the right to life by depriving people of the benefits of a healthy environment.  



In addition, numerous pieces of Canadian environmental legislation explicitly recognize and seek to protect the right to a healthful environment.
The \emph{Ontario Environmental Bill of Rights} (1993) recognizes the ``inherent value of the natural environment'' and states that ``the people of Ontario have the right to a healthful environment'' and ``have as a common goal the protection, conservation and restoration of the natural environment for the benefit of present and future generations.''  
The purposes of the act are:
\begin{enumerate}
	\item to protect, conserve and, where reasonable, restore the integrity of the environment by the means provided in this Act;
	\item to provide sustainability of the environment by the means provided in this Act; and
	\item to protect the right to a healthful environment by the means provided in this Act.  1993, c. 28, s. 2 (1).
\end{enumerate}

\begin{vcom}
	Legislation cited here should be cited using the same \LaTeX footnote system employed through the rest of the brief.
\end{vcom}


The above purposes include the following:
\begin{enumerate}
	\item The prevention, reduction and elimination of the use, generation and release of pollutants that are an unreasonable threat to the integrity of the environment.
	\item The protection and conservation of biological, ecological and genetic diversity.
	\item The protection and conservation of natural resources, including plant life, animal life and ecological systems.
	\item The encouragement of the wise management of our natural resources, including plant life, animal life and ecological systems.
	\item The identification, protection and conservation of ecologically sensitive areas or processes.  1993, c. 28, s. 2 (2). 
\end{enumerate}



The activities of fossil fuel companies frustrate all of the above purposes by contributing to climate change, thereby undermining the right to a healthy environment of the people of Ontario.

\begin{vcom}
	More specific examples of how each of these purposes are challenged by climate change should be added.
\end{vcom}


Environmental laws for other provinces of Canada recognize and seek to protect the same right to a healthy environment.  



For example, Part 1, section 6 of the \emph{Yukon Environment Act} states that: ``The people of the Yukon have the right to a healthful natural environment.''   
In accordance with this right, the Act seeks to protect the environment of the Yukon by providing an appropriate process to assess the environmental effects of projects and activities in the Yukon or that may have effects in the Yukon. Similarly, the \emph{Northwest Territories Environmental Rights Act} recognizes that ``the people of the Northwest Territories have the right to a healthy environment and a right to protect the integrity, biological diversity and productivity of the ecosystems in the Northwest Territories'' and establishes the means by which individuals can act to protect the environment from harm. By pursuing the extraction of fossil fuels, the companies in question undermine the right to a healthy environment that these acts articulate and protect. Finally, Quebec’s \emph{Environmental Quality Act} states that, ``Every person has a right to a healthy environment and to its protection, and to the protection of the living species inhabiting it, to the extent provided for by this Act and the regulations, orders, approvals and authorizations issued under any section of this Act and, as regards odours resulting from agricultural activities, to the extent prescribed by any standard originating from the exercise of the powers provided for in subparagraph 4 of the second paragraph of section 113 of the Act respecting land use planning and development'' (chapter A-19.1).  

\begin{vcom}
	Should we be including legislation from provinces other than Ontario? If so, it should be cited in the standard way for the brief.
\end{vcom}

The activities of the fossil fuels industry in Canada also violate the constitutional and treaty rights of Canada's First Nations.
These violations arise both from the specific impact of fossil fuel development projects --- such as the oil sands --- and from the inevitable consequences of burning fossil fuels.
Rights that are being violated include the right to consultation and accommodation; the right to waters and land and to fish, hunt and trap; and the aboriginal rights affirmed in Canada's constitution.



Keepers of the Athabasca member Vivienne Beisel explains how the oil sands development has violated Treaty 8 and the Constitution: ``The cumulative impacts of oil sands development has all but destroyed the traditional livelihood of First Nations in northern Athabasca watershed. 
The law is clear that First Nations must be consulted whenever the province contemplates action that may negatively affect Aboriginal and treaty rights...
The province has continued to issue approvals for new developments without obtaining their consent or consulting with First Nations in a meaningful and substantial way. 
This is in direct breach of Treaty 8 First Nations' treaty-protected Aboriginal rights to livelihood, and thus a violation of s.35(1) of the Constitution'''.


\begin{vcom}
	MIE HAD THIS TEXT, BUT I DON'T THINK IT ADDS TO OUR CASE ``and Articles 26 and 27 of the United Nations Declaration on the Rights of Indigenous Peoples, an international agreement which Canada, along with three other nations, has refused to sign.'' We need to add information on how the rights of aboriginals in Ontario specifically risk being violated by fossil fuel companies.
\end{vcom}



Finally, the activities of the fossil fuels companies in which the University of Toronto is invested frustrate international law.  
First, Article 3 of the \emph{Universal Declaration of Human Rights} states that ``Everyone has the right to life, liberty and security of person.''   
The right to life is a precondition to all other fundamental human rights.  
The activities of companies in the fossil fuels industry undermine the right to life by depriving people of the benefits of a healthy environment.  
In addition, the \emph{Hague Declaration on the Environment} (1989), to which Canada is a signatory, makes the link between the right to life and the harmful change effects of climate change explicit: ``The right to live is the right from which all other rights stem.  
Guaranteeing this right is the paramount duty of those in charge of all States throughout the world.  
Today, the very conditions of life on our planet are threatened by the severe attacks to which the earth’s atmosphere is subjected.''   
In signing onto this Declaration, Canada recognized the reality of the threat to human life posed by climate change and pledged to take measures to address that threat.  
The University of Toronto’s investment in fossil fuels frustrates any efforts Canada has taken or may take in the future to address the problem of climate change by supporting the companies that most significantly contribute to the problem.	

\begin{vcom}
	Again, this needs to be made more specific both in terms of impacts and in terms of laws. Where possible, we want to be able to point to specific documented effects that contravene specific pieces of applicable legislation.
\end{vcom}


The activities of fossil fuel companies are also at odds with the fundamental objective of the \emph{United Nations Framework Convention on Climate Change} (UNFCCC), which was ratified by Canada and which entered into force on March 21st 1994.
The UNFCCC affirms the intention of signatories to achieve ``stabilization of greenhouse gas concentrations in the atmosphere at a level that would prevent dangerous anthropogenic interference with the climate system''.
Countries including Canada have since adopted a threshold of 2˚C of global temperature increase above pre-industrial levels as constituting `dangerous' climate change.
Achieving this objective requires that most of the reserves of fossil fuel companies be left unburned underground.
It also requires the abandonment of projects intended to extract unconventional reserves of fossil fuels, through activities including oil and gas drilling in the arctic, exploitation of the oil sands, and extraction of previously inaccessible oil and gas reserves through hydraulic fracturing.



	\subsection{Why fossil fuels are like tobacco}


\begin{vcom}
		Get a working anchor for the link in part 3
\end{vcom}

% \hypertarget{TobaccoPrecedentInjury}



In 2007, the University of Toronto decided to divest from tobacco companies, after determining that the case to do so was consistent with university policies.
There are several important ways in which the tobacco precedent is relevant to fossil fuel divestment.



Firstly, the scientific case demonstrating the harm caused by tobacco strengthened progressively over the span of decades.
Companies were initially willing to challenge these claims, but the weight of evidence eventually made their case untenable.
Similarly, the evidence demonstrating the seriousness of anthropogenic climate change has now progressed beyond the point where it can be considered a subject of ongoing academic inquiry and debate.


Secondly, in both the cases of tobacco and fossil fuels the problem is the primary product being produced by the industry.
Just as it would be ineffective to use shareholder voice to try to convince a tobacco company to stop producing and selling tobacco, it is implausible that the university could use shareholder activism to convince fossil fuel companies to desist from activities that create and facilitate major greenhouse gas pollution.



Thirdly, both investments in tobacco and fossil fuels challenge the core values of the university.



\begin{vcom}
	Get a working link to the tobacco precedent section in part 6 here
\end{vcom}



% The tobacco precedent also demonstrates that \hyperlink{TobaccoPrecedentFinance}{the university can divest from Shell without suffering financial harm}.




\begin{vcom}
	Elaborate, and add citations related to the tobacco precedent. Also, consider whether any other previous U of T divestment campaigns can be cited as useful precedents.
\end{vcom}



% END SECTION 3 - MIE