% BEGIN SECTION 4 - STUART
% Last updated by Stu Basden 2013-MAY-12



\section {Divestment is compatible with the university's fiduciary duties}



\subsection {From the U of T divestment policy}
\begin{itquote}	
(i) prudent investment. The University has a fiduciary duty to manage investments responsibly to maximize return on its investments within a policy risk tolerance as approved by Business Board from time to time.

...

The committee will consider the following guidelines in considering the appropriate response to any request:
\begin{itemize}
  \item the extent and significance of the University's investment in a particular entity. Determination of whether investments are considered significant will depend on the committee’s judgment of the relative magnitude of the University’s holdings both as a fraction of all University investments and in relation to the market capitalization of the entity under review.
  \item the degree to which the entity itself is involved in the undesirable activity.
\end{itemize}
Normally, activity is considered significant if more than ten percent of the entity's revenues are derived from the undesirable activity.
\end{itquote}

\subsection {There is no evidence of a divestment penalty for investors}

Several studies have attempted to quantify the financial consequences of divestment from the fossil fuel industry or heavy polluters in general.
In aggregate, these studies found no significant impact on investment risk in predictive models, nor a performance penalty in tests using historical data.
\begin{description}
  \item[Historical] The UN Environment Program Finance Initiative meta-analysis found there to be no evidence of penalty to screened portfolios \footcite{UNEPFI2007}
  \begin{vcom}
    I (Stu) didn't really understand the document. This comment should be clarified.
  \end{vcom}
  \item[Risk Based Assessment] The aperio group found that divesting from the ``Filthy Fifteen'' ``increases absolute portfolio risk by only 0.0006\%, or about a half of one one-thousandth of a percent.'' Even divesting from the entire Fossil Fuel sector only results in a 0.0034\% return penalty. In other words, the portfolio does become riskier, but by such a trivial amount that the impact is statistically irrelevant. \footcite{Aperio2013}
  \item[Forward Looking] Carbon Tracker and Standard \& Poors together conducted a study on the implications of carbon constraints for credit ratings of the oil and gas sector. Their scenario assumes reducing demand for CO2-intensive fuels, in line with the internationally recognized limit of a 2ºC rise in global temperatures, and is ``not materially different from the current price deck assumptions.'' The study concludes with the statement,
  \begin{slquote} ``as the price declines persist in our stress scenario of weaker oil demand, meaningful pressure could build on ratings. First to be affected would be the relatively focused, higher cost producers, and then the more diversified integrated players. In both cases, according to our study, the causes would be a decline in operating cash flows, weakening free cash flow and credit measures, along with less certain returns on investment and less robust reserve replacement.'' \footcite{SandPConstrained}
  \end{slquote}
  So not only does divestment present little risk of missing out on improvements in ratings (and therefore, investment returns), divestment could actually be beneficial for the portfolio, in that it would remove risk of being invested in companies which only appear to decline in the long term.
  
\end{description}


\subsection {Market capitalization and value at risk}

\subsubsection {Stated policy objectives are incompatible with the current valuation of fossil fuel reserves}

Fossil fuels may provide a hedge against other asset classes, but only in scenarios where unconstrained emissions lead to accelerated and possibly catastrophic warming. 
The international community is in broad agreement that this must be prevented from happening.




\subsubsection {Regulatory risk is not adequately priced} 



As one scenario for the World Energy Outlook in 2012, the International Energy Agency assumes international cooperation to keep \ce{CO2} under 450ppm, which in their model constrains the likelihood of warming greater than 2°C to 55\%. 
This is in contrast to their baseline New Policies Scenario, which assumes modest reductions in the rate of emissions increase compared to the third scenario, Current Policies. 
Regarding the effect this policy environment would have on the price of fossil fuels they estimate:
\begin{slquote}Compared with the New Policies Scenario, the global oil price in the 450 Scenario in 2035 is \$25 per barrel lower and the coal price almost 40\% lower. The price for natural gas falls by 23\% in Europe and 4\% in North America.\footcite[][p. 257]{IEA2012}
\end{slquote}

For any scenario where emissions are constrained to keep warming under 2°C, market assumptions regarding the profitability of fossil fuel extraction are necessarily optimistic. 
Marginal projects will become unprofitable and returns to investors for even the most profitable projects will decline. 
Indeed, the study conducted by Standard \& Poors even indicates near-term threats to the stability of investing in some fossil fuel companies:
  ``Under our stressed scenario, the ratings on companies with high development and production costs, including those focused on unconventional resources, could see rating pressure build \emph{within one or two years}, especially if the companies are relatively undiversified'' (emphasis ours). \footcite{SandPConstrained}
  The study continues, ``We see a deterioration in credit measures for these smaller oil companies over 2014-2015, to a degree that could potentially lead to negative outlook revisions and downgrades over 2014-2017. ... this could result in an earlier deterioration in our business risk profile assessments.''
  Furthermore, even ``the financial risk profiles of the oil majors would weaken modestly over the next five years.''
  
  It is also worthy of note that the study claims, ``the core business model [of fossil fuel companies] could come into question,'' and that ``this could potentially result in \emph{a downgrade of more than one notch} if we were to place less reliance on undeveloped or probable reserves than at present'' (emphasis ours).
  Indeed, even for the `oil majors', ``reserve replacement ratios (RRRs) below 100\%... could in our opinion become a rating constraint.''
  
  


\subsubsection {There is a strong potential for malinvestment in capital-intensive, long-term projects}

The IEA's World Energy Outlook (2012) concluded that, ``more than two-thirds of current proven fossil-fuel reserves cannot be commercialized in a 2DegC world before 2050.'' \footcite{IEA2012}
The Standard \& Poors Carbon Tracker Initiative study raises their concerns concerning the fossil fuel sector: ``This illustrates to us the apparent divergence between the assets owned by coal, oil, and gas companies and the direction of negotiations at UNFCCC conferences.'' \footcite{SandPConstrained}
Indeed, investment in CO2-emissions-enabling infrastructure goes in the opposite direction of the international community.

The persistently high price of fuels on the world market in recent years has lead to unprecedented investment on the part of the fossil fuel industry in projects that were previously deemed too marginal to profitably develop. 
Development of unconventional hydrocarbon reserves such as tar sands, oil shale, offshore drilling in extremely deep water and the arctic, hydraulic fracturing and mountaintop removal coal mining entails extremely high capital investment. 
Scenarios in which carbon emissions are restricted sufficiently to keep global temperatures from rising more than 2°C would likely cripple the return on much of this investment.

In anticipating such a scenario, the fossil fuel industry has been pinning its hopes on the development of effective methods of carbon capture and sequestration (CCS). 
Despite tremendous investment in this technology on the part of both the private and public sectors, economically feasible sequestration of emissions at scales needed to mitigate climate change remains elusive. 
There are currently no commercial scale CCS projects in operation on the planet, and in 2008 Cambridge Energy Research Associates (CERA) predicted that it would be another two decades before CCS saw large-scale deployment. \footcite{CERACrossing}
According to the Carbon Tracker Initiative, even if CCS is deployed in line with an idealised scenario by 2050, this would only extend fossil fuel carbon budgets by 12-14\%, or just 4\% of total global reserves. \footcite{CTI2013}
It must be remembered that at the current rate of global carbon emissions, the entire budget of carbon emissions would be spent by the late 2020s, several years before large-scale CCS can be expected to come online. \footcite{CTI2012}

CCS has many other problems associated with it. For example, CCS would use extra energy, potentially as much as 40\% of the power generated by a power station. \footcite{GPCCS}
This reduces the efficiency of the power plant, both increasing financial costs, and increasing the amount of fuel needed per energy output, which in turn increases the problems associated with fossil fuel extraction.
Indeed, the increased cost of the energy provided by CCS-enabled power stations would likely be higher than the cost of energy from renewable sources, and so would almost certainly never be implemented. \footcite{SmartPlanetCCS}
Storing carbon underground is risky—safe and permanent storage of CO2 cannot be guaranteed, and even very low leakage rates could undermine any climate mitigation efforts. \footcite{GPCCS}
Finally, money spent on CCS will divert investments away from sustainable solutions to climate change, which the world will need to transfer to eventually, whether or not it burns all the available (non-renewable) fossil fuels.

Therefore, pinning our hopes on a non-existent technology, that is likely to both be more expensive and problematic than other energy sources, is a false hope.




\subsubsection {Fossil fuel reserves as stranded assets} 



Given the degree to which proven reserves of carbon exceed allowable emissions for sub-2°C warming, companies with fossil fuel reserves as their largest assets may be substantially overvalued under current market conditions. 
Stranded assets in the form of unburnable reserves and large liabilities incurred to develop those reserves combine to create a risk not only to equity, but to bondholders as well.
The Carbon Tracker Initiative reports that in 2012 the Fossil Fuel sector spent \$674 billion prospecting for new sources of carbon, sources which cannot be exploited if the 2C target is to be met. \footcite{CTI2013}

As the Carbon Tracker Initiative's 2012 report made clear, fossil fuel companies have significantly more exploitable sources of carbon available than is safe to burn. \footcite{CTI2012}
Therefore, when considering ``What A Carbon-Constrained Future Could Mean For Oil Companies' Creditworthiness,'' Standard \& Poors decided that, ``instead of considering issues of peak oil in terms of supply, this introduces a concept of peak oil demand.'' \footcite{SandPConstrained}


\subsubsection {Volatility of investor sentiment}



Current market capitalization of the fossil fuel industry rests in part on the assumption that the global investor class will continue to see the sector as a reliable investment even as damage from climate change becomes apparent. 
This assumption has been increasingly challenged from both outside and within the financial industry. 
Traditionally conservative-minded publications such as \emph{The Economist}, \footcite{EconomistUnburnable} \emph{Business Week} \footcite{BusinessWeekOvervalued} and the \emph{Financial Times} \footcite{FTOvervalued} have published articles suggesting the fossil fuel sector is overvalued. 
In recent months, other voices within the financial industry such as investors groups and hedge fund managers have been increasingly sounding the alarm over the ``Carbon Bubble''. \footcite{JeremyGrantham} 
The Guardian recently reported
\begin{slquote}
The message to all the players across the financial chain, from ratings agencies through accountants, to actuaries, investment advisors and all the rest, is also obvious. If the regulators won't do their job, do it for them. \emph{Jump, before you are pushed} (emphasis ours). \footcite{Guardian6Trillion}
\end{slquote}
The afore-mentioned Standard \& Poors study, which saw a declining trend on both the short-term and long-term outlook for fossil fuel companies (both mid-size and large), reached their conclusion without ``explicitly [factoring in] any mitigating measures such as ... material cuts in near-term capital investment.'' \footcite{SandPConstrained}
However, there is already a significant, international fossil-fuel divestment movement, which could result in such material cuts: over 300 colleges and universities, and over 100 cities and states currently have divestment campaigns, along with several religious institutions.
So far, ten cities and four colleges have pledged to divest and the movement is only just getting started.









\subsection {Fossil fuels represent a risk to the university's other investments}

Institutional investors, and universities in particular are unique in that they are often expected to plan financially on a timescale far longer than average. 
On timescales of 50 years or more, the consequences of unconstrained emissions are very likely to overshadow all other financial considerations.
According to a 2012 report by DARA, Climate change is already costing the world more than \$1.2 trillion, wiping 1.6\% annually from global GDP.
By 2030, the researchers estimate, the cost of climate change and air pollution combined will rise to 3.2\% of global GDP, with the world's least developed countries forecast to bear the brunt, suffering losses of up to 11\% of their GDP. \footcite{DARACVM}
Going further into the future is increasingly hard to predict, with estimates varying widely: the Stern Review estimates losses of between 5\%-20\%, \footcite{Stern2007} and a United Nations report asserts that climate change could cost Latin American and Carribean countries 137\% of GDP by 2100. \footcite{CCLatinAmerica}
However, regardless of the variations of predictions, the trend is crystal clear: the more the climate changes, the greater the reductions to GDP.
Therefore, reducing climate change will result in increased GDP, therefore, greater returns on the University's investments.





\subsection {Attractive substitutes exist for divested equities}

There are many attractive alternatives that could form substantial portions of the Universities' portfolio.
The renewable energy sector has enormous growth potential, and is starting to match even convention fossil-fuel energy prices (let alone unconventional).
Unsubsidised renewable energy is now cheaper than electricity from new-build coal- and gas-fired power stations in Australia, according to new analysis from research firm Bloomberg New Energy Finance. \footcite{BlombergAussieWind}
Solar power is predicted to be cheaper than fossil fuel power in the USA as soon as 2015. \footcite{GlobalDataSolar}
In March 2013, 100\% of the new energy on the U.S. grid was solar power. \footcite{SmartPlanetSolar100}

There are three broad-based mutual funds that are completely fossil free: Green Century Balanced Fund, Porfolio 21 Global Equity Mutual Fund, and Shelton Green Alpha Fund. \footcite{GFFMyMoney}



\subsection {Pensions and climate change}

Pensions are intended to allow the pensioner to enjoy a satisfactory, even comfortable, retirement.
However, the more the climate is changed, the lower the quality of life that will be enjoyed by the retirees.
A study conducted by the World Bank makes it clear that a 4ºC hotter world will be a much more hostile world than one in which there has only been a 2ºC rise, with 6ºC or greater rises being even more hostile still. \footcite{WorldBank4C}
We saw in the previous sections of this brief that even a 2ºC rise will result in a greater frequency of natural disasters than the relative climate stability of the development of human civilisation.
Indeed, the worlds' top scientists have calculated that a concentration of CO2 in the atmosphere that is higher than 350ppm is not compatible with the planet ``on which civilization developed and to which life is adapted.'' \footcite{TargetAtmosphere}

Therefore, it is in the best interests of the future pensioners of the University of Toronto (its current employees) to live in a world with a stable climate.
For example, if a 55 year old college president were to insist today that a portfolio requires fossil fuel investment, then when that president reaches the age of retirement,  only 11\% of the CO2 released during the class of 2016's education will have left the atmosphere.
The same article also highlights the perspective of students: 
\begin{slquote} ``Even if today’s college students live to be 100 years old, more than half of the CO2 released into the atmosphere during the four years they are in college will still be present there at the end of their lives – warming the planet and contributing to extreme events, like droughts, floods, and storms all the while – long after the decision makers behind those investment choices will have left office. The college students across the US who are arguing that their education should not be funded by actions that diminish the health of the world in which their future will unfold have a strong case, supported by the basic physics of the climate.'' \footcite{ClimateInteractivePersist}
\end{slquote}
This concept is further reinforced by James Powell, former-president of Oberlin, Franklin and Marshall, and Reed College, who suggests that trustees have a quasi-legal duty to do all they can about climate change.
``The board is supposed to make sure that the endowment allows for intergenerational equity, that the students who are going to Oberlin in 2075 get as much benefit from it as those there now. But with global warming, you're guaranteeing a diminution of quality of life decades out.'' \footcite{CaseForDivestment}

Therefore, not only is divestment from the fossil fuel industry a sound financial decision for meeting the financial obligations of prudent investment, the current employees of the University of Toronto will benefit from such divestment.

\begin{vcom}
  Notes on this section:

  Stuart took over the drafting of this section. This may be a place where we can make especially good use of documents already prepared by other schools, including: (a) schools U of T regards as peers, like Harvard (b) schools in a similar position to U of T, like McGill and UBC and (c) schools that have already divested.
  
  This section also needs to specifically address this section from the Procedures for Responding to Social and Political Issues with Respect to University Divestment: ``the extent and significance of the University’s investment in a particular entity.  Determination of whether investments are considered significant will depend on the committee's judgment of the relative magnitude of the University’s holdings both as a fraction of all University investments and in relation to the market capitalization of the entity under review.''
  
  It is virtually certain that the ad hoc committee will consider the special responsibility the university has toward current and future retirees. We need to make the case that (a) divestment is a sound financial decision for meeting those obligations and (b) if current employees want a decent retirement, dangerous climate change must be avoided.
\end{vcom}




% END SECTION 4 - STUART