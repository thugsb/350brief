% BEGIN SECTION 5 - MILAN
% Last updated by Milan Ilnyckyj 2013-MAY-12


	
	
		\section{Actions have been taken by the Canadian government and international bodies on this issue}



All three levels of Canada's government have taken action in response to the threat of climate change.

		\subsection{From the U of T divestment policy}


\begin{itquote}	
Responses should be based on the following principles:

...

(iii) actions taken by the Canadian government or other national or international bodies with regard to the particular issue of concern.
\end{itquote}


		
		\subsection{Federal government}
		
		
		
\begin{description}
	\item[Emission standards for passenger vehicles and light trucks] In November 2012, proposed regulations were released for vehicles beginning with the 2017 model year. 
	Average emissions from vehicles in 2025 are expected to be 50\% of those sold in 2008.\footcite[][]{ECReducing}
	\item[Heavy duty vehicles] In April 2012, the federal government released regulations for heavy duty vehicles beginning with the 2014 model year.\footcite[][]{ECReducing}
	\item[Coal-fired power plants] In September 2012, final regulations were introduced to limit emissions from the coal-fired electricity sector.\footcite[][]{ECCoalFired}
	\item[Renewable fuel requirement] As of December 2010, gasoline is required to contain an average of 5\% renewable content, with a 2\% requirement for diesel fuel.\footcite[][]{ECReducing}
	\item[Carbon capture and storage (CCS)] Canada's federal and provincial governments have committed a total of approximately \$3 billion in funding for CCS, which could lead to as many as five to six large-scale demonstration projects in Canada.\footcite[][]{ECReducing}
	\item[Agricultural greenhouse gases] Canada is contributing \$27 million toward the Global Research Alliance on Agricultural Greenhouse Gases, a group created to advance research, technology transfer, and adoption of beneficial management practices to mitigate agricultural greenhouse gases.\footcite[][]{ECReducing}
\end{description}		
		
		
		
		\subsection{Government of Ontario}
\begin{description}
	\item[Emission reduction targets] The Government of Ontario has legislated greenhouse gas emission reduction targets of 6\% below 1990 levels by 2014, 15\% below by 2020, and 80\% below by 2050.\footcite[][]{OMEGreening}
	\item[Phasing out coal] The Government of Ontario has committed to phasing out coal-fired electricity generation by 2014.\footcite[][]{EPACessation}
	\item[Public transit investments] The Ontario government is contributing over \$9 billion to the Metrolinx Regional Transportation Plan.\footcite[][]{OMEGreening}
	\item[Green Energy Act] Ontario's 2009 Green Energy Act created a system of feed-in tariffs to support the deployment of renewable energy options including solar photovoltaic, biogas, biomass, landfill gas, and wind power.\footcite[][]{OMEGreen} 
It established a right for all renewable energy projects to be connected to the grid, streamlined the approval process for green energy projects, and began the implementation of a `smart' energy grid.
	\item[Forest protection] Ontario has protected roughly half of the province's boreal forest from mining and forestry, motivated in part by the forest's importance as a carbon sink.\footcite[][]{}
	\item[Establishment of a Climate Change Secretariat] In 2008, the province created a permanent secretariat to coordinate its \emph{Climate Change Action Plan}.\footcite[][]{}
	\item[Community Go Green Fund] The province provided \$6 million to 90 community groups in order to help charitable or environmental organizations, youth or cultural associations, educational institutions and Aboriginal communities reduce their carbon footprint.\footcite[][]{OMEGreening}
\end{description}	



	\subsection{City of Toronto}



\textbf{Climate Change Action Plan}

The city's Climate Change, Clean Air and Sustainable Energy Action Plan was unanimously adopted by Toronto City Council in July 2007. The city allocated over \$1 billion over the next five years to projects to reduce greenhouse gas emissions.\footcite{TorontoEnvOff2007}

These commitments included:
\begin{itemize}
	\item \$67 million for the Better Building Partnership and the Sustainable Energy Funds, which are low interest revolving loan funds that support energy conservation and renewable energy
	\item \$136 million for energy retrofits to and installation of renewable energy systems on City owned buildings; 
	\item \$24 million for tree planting, in addition to the \$40 million a year operating budget for the city's Forestry Unit; 
	\item \$36 million to accelerate implementation of the City's Bike Plan; 
	\item \$20 million for the Live Green Toronto program which provides support for neighbourhood and community groups in taking action on Climate Change; 
	\item \$10 million for continued conversion of traffic signals to LED lights; 
	\item \$7 million for the Clean Roads to Clean Air street sweeping initiative; 
	\item \$186 million for water efficiency and improved energy efficiency in Toronto Water operations that will achieve an annual avoidance of an estimated 14,000 tonnes of greenhouse gas emissions; 
	\item \$21 million for methane gas capture and control at closed and operating landfills; 
	\item \$67 million to build anaerobic digestion facilities that will capture biogas from collected Green Bin organic materials and generate enough electricity to power an estimated 1,700 homes; 
	\item \$380 million to improve rapid transit services, such as, new light rapid transit lines, rapid transit routes for buses and an improved signalling system that will increase the capacity of the Yonge subway line; 
	\item \$400 million for the purchase of electric-hybrid buses; and 
	\item \$10 million plus for a range of initiatives including the Green Fleet Transition, the Eco-Roofs and Greenroofs Incentive programs, and support initiatives that promote production and consumption of locally grown food.
\end{itemize}

These investments are specifically justified with reference to the danger of climate change, with expected impacts on the city including rising mean temperatures, warmer winters, changes in disease vectors, changes in precipitation patterns, increased extreme weather, falling lake and stream levels, and rising sea levels.\footcite{TorontoEnvOff2007}

The City of Toronto has also committed to specific greenhouse gas reduction targets, starting with the city's 1990 baseline level of approximately 22 million tonnes per year:\footcite{TorontoAQandCC}
\begin{itemize}
	\item 6 percent by 2012 (1,320,000 tonnes per year)
	\item 30 percent by 2020 (6,600,000 tonnes per year)
	\item 80 percent by 2050 (17,600,000 tonnes per year)
\end{itemize}

Other actions taken by the city include:
\begin{description}
	\item [Adaptation] The city is making efforts to prepare for the impacts of climate change, through programs and policies including Toronto’s Heat Alert system and Hot Weather Response Plan, The Wet Weather Flow Master Plan, Green Roof Pilot Incentive Program, Deep Lake Water Cooling (Enwave), Peaksaver and Keep Cool Programs (Toronto Hydro), Green Development Standard, and Better Buildings Partnership.
	\item [Great Lakes Climate Change Policy Coordination] Along with 10 other cities in the Great Lakes region, Toronto is working to develop an international city-level policy on climate change.
	\item [Live Green Toronto] This  five-year, \$20-million dollar program is intended to promote and support actions by residents and community groups to reduce emissions, clean our air and protect our climate.
	\item [Landfill gas] The City of Toronto collects and burns landfill gases that are emitted at its three largest landfill sites: Keele Valley, Brock West and Beare. The city explains that: ``the process of collecting and incinerating landfill gases is crucial to the goal of combating the emission of greenhouse gases into the atmosphere''.\cite{TorontoAQandCC}
	\item [Greenhouse Gas and Air Emissions Inventory] In 2007, the city completed a combined greenhouse gas and air quality emissions inventory, with information about energy consumption and pollutants within the city. 
	\item [Concern about oil sands pipelines] The Toronto City Council has expressed its desire to review the application of Enbridge to reverse their Line 9 pipeline to carry diluted bitumen from the oil sands. 
	The city may apply to become an intervenor in the National Energy Board process.
\end{description}



% Beginning of section from Jon and Emily - Converted into LaTeX by Milan 2013-04-03



		\subsection{Actions taken by other national bodies}
		
		
		
The extensive actions undertaken by other countries demonstrates the seriousness of climate change.
In many cases, they have implemented significantly more ambitious policies than those enacted in Canada to date.
This action demonstrates how, in the view of the world's major governments, the need to mitigate climate change is not properly the subject of academic inquiry and debate.


		
		\subsubsection{United States}
		
		
		
\textbf{Federal government}


At the federal level, the White House under the leadership of President Obama has taken many steps towards mitigating and adapting to climate change.\footcite[][]{WHcc2013} \footcite[][]{WHadaptation2013} \footcite[][]{WHenergy}
\begin{description}
	\item [Monitoring Emissions] The United States is comprehensively cataloguing greenhouse gas emissions from its largest emitting sources.
	\item [Government Procurement and Energy Consumption] President Obama directed the Federal Government – the largest energy consumer in the U.S. economy – to reduce its greenhouse gas emissions from direct sources such as building energy use and fuel consumption by 28 percent by 2020. He also directed Federal agencies to reduce their greenhouse gas emissions from indirect sources, such as those from employee commuting, by 13 percent by 2020.
	\item [Creation of the Climate Change Adaptation Task Force (CCATF) and the U.S. Global Change Research Program (USGCRP)] The CCATF recommends how Federal agency policies and programs can better prepare the United States to address the risks associated with a changing climate. The USGCRP is a collaborative effort involving 13 Federal agencies to evaluate the current and future impacts of climate change, inform policy-makers and the public about scientific findings, and investigate effective ways to reduce greenhouse gas emissions and deploy cost-effective clean energy technology.
	\item [Investing in Clean Energy] With the support of administration policy, the U.S. has nearly doubled renewable energy generation from wind, solar, and geothermal sources since 2008. Since 2009, the Department of Interior has approved 29 onshore renewable energy projects, including 16 solar, 5 wind, and 8 geothermal projects. Moving forward, the Department of Interior is committed to issuing permits for 10,000 megawatts of renewable power on public lands and in our offshore waters by the end of 2012, enough to power 3 million homes. In 2010, President Obama also set a goal of breaking ground on at least four commercial scale cellulosic or advanced biorefineries by 2013. That goal was accomplished a year ahead of schedule.
	\item [Smart Grid] In 2011, the Administration announced that it would accelerate the permitting review of seven proposed electric transmissions lines. These infrastructure projects, when built, will increase grid capacity, facilitating better integration of renewable energy sources, avoiding blackouts, and helping to accommodate the growing number of electric vehicles on the road. The Administration also launched a Green Button initiative in 2011 to empower Americans to reduce energy use in their homes. Already, utilities across the country have committed to providing 27 million households with access to data about their own energy use with a simple click of an online ``Green Button'' that will help them reduce waste and shrink their energy bills.
	\item [Clean Energy Research \& Development] In 2009, the Administration funded the Department of Energy’s Advanced Research Project Agency-Energy (ARPA-E), which focuses on “out-of-the-box” transformational energy research that brings together the nation’s best scientists, engineers, and entrepreneurs. Building upon the initial investment, in late September 2011, the ARPA-E program announced 60 cutting-edge research projects in 25 states. In total, The ARPA-E has supported more than 120 individual projects.
	\item [Clean Energy Innovation Hubs] The Administration also launched a series of clean energy innovation hubs, which bring together teams of the best researchers and engineers in the United States to solve major energy challenges. The hubs will focus on improving batteries and energy storage, reducing constraints from critical materials, developing fuels that can be produced directly from sunlight, improving energy efficient building systems design, and using modelling and simulation for advanced nuclear reactor operations.
	\item [The President's Better Buildings Challenge] The President's Better Buildings Challenge has set a goal to improve the energy efficiency of commercial buildings by 20 percent by 2020. The Administration has also partnered with manufacturing companies, representing over 1,400 plants, to improve energy efficiency by 25 percent over 10 years.
\end{description}	


\textbf{The Environmental Protection Agency (EPA)}
	
The EPA develops standards for greenhouse gas emissions from mobile and stationary sources under the Clean Air Act. Its federal regulatory activities are in addition to its volunteer programs, international partnerships, and partnerships with states and tribes.
\begin{description}
	\item [Standards to Cut Greenhouse Gas Emissions and Fuel Use for New Motor Vehicles] The EPA is enabling the production of a new generation of clean vehicles --- from the smallest cars to the largest trucks --- through reduced greenhouse gas emissions and improved fuel use. Together, the enacted and proposed standards are expected to save more than six billion barrels of oil through 2025 and reduce more than 3,100 million metric tons of carbon dioxide emissions
	\item [Renewable Fuel Standard Program] A set of regulations to ensure that transportation fuel sold in the United States contains a minimum volume of renewable fuel. By 2022, the Renewable Fuel Standard (RFS) program will reduce greenhouse gas emissions by 138 million metric tons, about the annual emissions of 27 million passenger vehicles, replacing about seven percent of expected annual diesel consumption
	\item [Proposed Carbon Pollution Standard for New Power Plants] On March 27, 2012, EPA proposed a Carbon Pollution Standard for New Power Plants that would, for the first time, set national limits on the amount of carbon pollution that power plants can emit. The proposed rule, which applies only to new fossil-fuel-fired electric utility generating units, will help ensure that current progress continues toward a cleaner, safer, and more modern power sector.
	\item [Oil and Natural Gas Air Pollution Standards] On April 18, 2012, EPA finalized cost effective regulations to reduce harmful air pollution from the oil and natural gas industry, while allowing continued, responsible growth in U.S. oil and natural gas production. The final rules are expected to yield a nearly 95 percent reduction in VOC emissions from more than 11,000 new hydraulically fractured gas wells each year. The rules will also reduce air toxics and emissions of methane, a potent greenhouse gas.
	\item [Geologic Sequestration of Carbon Dioxide] The EPA has finalized requirements for geologic sequestration, including the development of a new class of wells, Class VI, under the authority of the Safe Drinking Water Act's Underground Injection Control Program. 
\end{description}



The EPA is also taking adaptation measures. These include:	
\begin{itemize}
	\item The Climate Ready Estuaries program works with the National Estuary Programs and the coastal management community to: (1) assess climate change vulnerabilities, (2) develop and implement adaptation strategies, and (3) engage and educate stakeholders.
	\item The EPA's Climate Ready Water Utilities (CRWU) initiative assists the water sector, which includes drinking water, wastewater, and stormwater utilities, in addressing climate change impacts.
\end{itemize}
	
	

Some federal departments are also taking actions specific to their purview. For example, the Department of Transportation’s Congestion Mitigation and Air Quality (CMAQ) Improvement Program provides over \$8.1 billion dollars in funds to State DOTs, MPOs, and transit agencies to invest in projects that reduce emissions from transportation-related sources.  Since October 2009, the Department of Energy and the Department of Housing and Urban Development have jointly completed energy upgrades in more than one million homes across the country.



\textbf{Regional/State/Local level}

\begin{vcom}
	This is the only place in the brief where we include raw URLs right in the brief text
\end{vcom}

There are a vast number of initiatives happening across the U.S. at the regional, state, and local levels. Due to the sheer volume, these initiatives cannot possibly be catalogued in this space. However, the Department of Transportation has two useful web-pages with links to various databases and initiatives across the country.
\begin{itemize}
	\item Local action: \url{http://climate.dot.gov/state-local/local-action-plans.html}
	\item Regional initiatives: \url{http://climate.dot.gov/state-local/regional-initiatives.html}
\end{itemize}
One initiative notable for its aggressiveness in tackling climate change is California Senate Bill X1-2. SBX1-2 directs California Public Utilities Commission's Renewable Energy Resources Program to increase the amount of electricity generated from eligible renewable energy resources per year to an amount that equals at least 20\% of the total electricity sold to retail customers in California per year by December 31, 2013, 25\% by December 31, 2016 and 33\% by December 31, 2020.




\textbf{Joint State-Provincial Initiatives}



There are several cross-border initiatives between U.S. states and Canadian provinces that are working to address climate change. These initiatives have been designed to reduce GHG emissions, develop clean energy sources, and achieve other environmental and economic goals.\footcite[][]{CCESinitiatives} They include:
\begin{itemize}
	\item North America 2050: \url{http://www.c2es.org/us-states-regions/regional-climate-initiatives#NA2050}
	\item Western Climate Initiative: \url{http://www.c2es.org/us-states-regions/regional-climate-initiatives#WCI}
	\item Regional Greenhouse Gas Initiative: \url{http://www.c2es.org/us-states-regions/regional-climate-initiatives#RGGI}
	\item Midwest Greenhouse Gas Reduction Accord: \url{http://www.c2es.org/us-states-regions/regional-climate-initiatives#MGGRA}
	\item Transportation and Climate Initiative \url{http://www.c2es.org/us-states-regions/regional-climate-initiatives#TCI}
\end{itemize}


	\subsubsection{United Kingdom}

The Climate Change Act 2008 made the UK the first country in the world to have a legally binding long-term framework to cut carbon emissions.  It introduced a long-term framework for managing emissions through a system of national carbon budgets: caps on the total quantity of greenhouse gases permitted in the UK over a specified time. Each carbon budget covers a five year period, with the first three carbon budgets running from 2008 to 2012, 2013-2017 and 2018-2022. During these periods, emissions must be reduced (from 1990 levels) by 22\%, 28\% and 34\% respectively.

The Act also created a framework for building the UK’s ability to adapt to climate change, including:
\begin{itemize}
	\item a UK-wide Climate Change Risk Assessment that must take place every five years
	\item a National Adaptation Programme which must be put in place and reviewed every five years to address the most pressing climate change risks to the UK
	\item a mandate giving the government the power to require ‘bodies with functions of a public nature’ and ‘statutory undertakers’ (eg water and energy utilities) to report on what they are doing to address the risks posed by climate change to their work.
\end{itemize}


The UK Department of Energy \& Climate Change has set the following national policies and strategies for combating climate change:
\begin{itemize}
	\item setting carbon budgets to limit the amount of greenhouse gases the UK is allowed to emit over a specified time
	\item using statistics on greenhouse gas emissions and further evidence, analysis and research to inform energy and climate change policy
	\item using the EU Emissions Trading Scheme (EU ETS) to meet over 50\% of the UK’s carbon emissions reduction target between now and 2020
	\item using a set of values for carbon to make sure project and policy appraisals account for their climate change impacts
	\item using the 2050 Calculator to let policy makers and the public explore the different options for meeting the 2050 emissions reduction targets 
\end{itemize}



Reducing the demand for energy and helping people and businesses to use energy more efficiently:
\begin{itemize}
	\item reducing demand for energy with smart meters and other energy-efficient measures for industry, businesses and the public sector
	\item reducing emissions by improving the energy efficiency of properties through the Green Deal
	\item providing incentives for public and private sector organisations to take up more energy-efficient technologies and practices through the CRC Energy Efficiency Scheme 
	\item reducing greenhouse gases and other emissions from transport
	\item reducing greenhouse gas emissions from agriculture
\end{itemize}



Investing in low-carbon technologies:
\begin{itemize}
	\item taking action to increase the use of low-carbon technologies and creating an industry for carbon capture and storage
	\item reducing emissions from the power sector and encouraging investment in low-carbon technologies by reforming the UK’s electricity market
	\item providing over £200 million of funding for innovation in low-carbon technologies from 2011 to 2015
\end{itemize}



Publicly reporting carbon emissions from businesses and the public sector:
\begin{itemize}
	\item encouraging corporate reporting of greenhouse gas emissions
	\item asking English local authorities to measure and report their greenhouse gas emissions
\end{itemize}



	\subsubsection{Germany}
	
	

\textbf{Targets}



In the framework of EU effort sharing under the Kyoto Protocol, Germany has committed itself to cutting its emissions of climate-damaging gases by a total of 21\% in the period 2008 to 2012 compared with 1990. 
In addition, Germany has pledged to reduce its GHG emissions by 40\% by 2020, 55\% by 2030, 70\% by 2040, and by 80-95\% by 2050 (compared with 1990 levels).  
Germany has also set ambitious targets for increasing the share of renewable energy in final energy consumption, with 18\% by 2020, 30\% by 2030 and by 60\% by 2050.



\textbf{Emissions Trading}



Emissions trading in particular makes a significant contribution to emissions reductions in Germany. 
The climate protection targets for the period 2008 to 2012 have been made significantly more stringent: from 2008, old power plants in Germany will be allocated around 30 percent fewer emission allowances than their current level of emissions. Furthermore, 10 percent of the allowances will be auctioned. 



\textbf{Feed-In Tariff}



The use of an adequate, long-term and predictable feed-in tariff encourages the construction of many renewable energy production sites. 
The differentiated feed-in tariff leads to one of the most diversified ranges of renewable energy technologies used within the European Union.



\textbf{Integrated Energy and Climate Programme (IECP)}



In order to reach the German climate protection goals the Federal Government has elaborated a comprehensive Integrated Energy and Climate Programme (IECP). 
Its goal is to ensure an ultramodern, secure and climate-friendly energy supply in Germany. 
It comprises measures for enhanced energy efficiency and expanded use of renewable energy sources.


Measures contained in the IECP include:
\begin{description}
	\item[Amendment to the Combined Heat and Power Act] The share of high-efficiency CHP plants in electricity production will be doubled by 2020 from the current level of around 12 percent to around 25 percent.
	\item[Amendment to the Energy Industry Act] Liberalising electricity metering will facilitate and promote innovative metering methods and demand-related, time-variable tariffs. This will enable consumers to reduce their energy costs and will improve the efficiency of the power generation sector.
	\item[Report and draft amendment to the Energy Saving Ordinance] Energy standards will be tightened by an average 30 percent from 2009. As a second step (planned for 2012), these efficiency standards will be tightened by a further 30 percent.
	\item[Clean power plants] Standards will be laid down for nitrogen oxide emissions from new power plants.
	\item[Guidelines on the procurement of energy-efficient products and services] Energy-efficient appliances and services will be promoted through priority procurement.
	\item[Amendment to the Renewable Energy Sources Act] The government’s goal is to increase the share of renewables in the electricity sector from the current level of over 13 percent to 25-30 percent in 2020, and then to continue increasing the level further. The amendment to the Renewable Energy Sources Act contains among other things new provisions for regulating tariffs for offshore wind farms.
	\item[Renewable Energies Heat Act] The share of renewable energies in heat provision will be increased to 14 percent by 2020. Obligations to use renewable energies in new buildings are laid down in the Renewable Energies Heat Act. Funding for the government support programme for existing buildings will increase - from 130 million euro in 2005 to up to 350 million in 2008 and up to 500 million euro from 2009.
	\item[Amendment to the Gas Grid Access Ordinance] The amendment to the Gas Grid Access Ordinance will ensure that biogas can be fed into the natural gas grid to a greater extent. A share of 10 percent biogas is possible by 2030.
	\item[Amendment to the Biofuel Quota Act] The share of biofuels will be increased and from 2015 will be geared more towards reducing greenhouse gas emissions. The amendment to the Biofuel Quota Act will lead to a rise in the biofuels’ share to around 20 percent by volume (17 percent by energy content) by the year 2020.
	\item[Sustainability Ordinance] The Sustainability Ordinance will ensure that when producing biomass for biofuels, minimum requirements for sustainable management of agricultural land and for the conservation of natural habitats are complied with.
	\item[Fuel Quality Ordinance] The amended Fuel Quality Ordinance will increase the blending limit of bioethanol in petrol fuels from 5 to 10 percent volume. For biodiesel in diesel fuels, this blending limit will increase from 5 to 7 percent volume.
	\item[Reform of vehicle tax to a pollutant and CO2 basis] For new vehicles, this tax will then be calculated on the basis of a vehicle's emissions rather than engine capacity as before.
	\item[Chemicals Climate Protection Ordinance] This Ordinance will reduce emissions of fluorinated greenhouse gases from mobile and stationary cooling installations through provisions on leakproofness and labelling of the installations and on recovery and return of the refrigerants used.
\end{description}



	\subsubsection{China}
	

China has surpassed the United States as the world’s largest greenhouse gas emitter. 
In 2011, China was the world leader in renewable energy technology investments, spending \$52 billion.



\textbf{Targets}



Under China’s 12th Five Year Plan, the government set set binding targets to reduce energy consumption per unit of GDP by 16 percent, cut CO2 emissions per unit of GDP by 17 percent, and raise the proportion of non-fossil fuels in the overall primary energy mix to 11.4 percent.  
At the Copenhagen Climate Change Summit in 2009, the Chinese government signalled its goal to reduce the carbon emissions intensity per unit of GDP by 40-45\% from 2005 levels by 2020.



\textbf{Transformation and upgrading of traditional industries}



Conserving energy and cutting emissions by optimizing and upgrading its industrial structure. 
The government has stepped up evaluation and examination of energy conservation, environmental impact assessments, and preliminary examination of land used for construction projects. 
It has raised the entry threshold for certain industries and limited new projects in industries with high energy consumption, high pollutant emissions or excess capacity. 
It has also controlled the export of products with high energy consumption and high pollutant emissions. 



\textbf{Supporting the development of strategic and newly emerging industries}



China has initiated a special fund to boost the development of strategic emerging industries, and expanded its venture capital program for emerging industries. 
So far 102 venture capital funds have been set up under the program, managing a total of 29 billion yuan. 
Among these funds, 24, with a total value of 7 billion yuan, are designed to stimulate the development of the energy-saving, environmental protection and new energy sectors.




\textbf{Carbon pricing}



Although this was not included in China’s 12th Five Year Plan, there have been reports that the Chinese government will be introducing a carbon tax on major energy consumers before the end of the plan. 
It is estimated that the tax would begin at 10 yuan (\$1.59) per ton of carbon dioxide emitted, and would increase depending on the company’s emission levels (information is not yet available on the details of the tax increases).



Five Chinese cities (Shanghai, Beijing, Shenzhen, Tianjin, and Chongqing) and two provinces (Guangdong and Hubei) are currently preparing pilot emissions trading schemes, set to begin in 2013. 
The Chinese government has ordered these areas to set greenhouse gas emissions control targets, and to implement an emissions trading scheme in order to meet these targets. 
This pilot project is considered to be an important learning step, leading up to the implementation of a national emissions trading scheme by 2015.



	\subsubsection{France}
	

\textbf{Targets}



France continues to support the targets stipulated in the Kyoto Protocol and sees the UNFCCC as a primary body through which climate change negotiations will move forward. 
France has already made progress in reducing its greenhouse gas emissions; in 2010, France had reached a 6.6\% reduction in emissions (compared to 1990 levels). 
France is committed to meeting the EU target of a 20\% reduction in emissions by 2020 (1990 levels) and has also set a goal of a 75\% reduction in emissions by 2050 (1990 levels), with intermediary targets of 40\% reduction by 2030 and 60\% reduction by 2040.  
Currently, France estimates that it will exceed its targets and achieve a 22.8\% reduction in greenhouse gas emissions by 2020 (compared to 1990 levels).



\textbf{L’Observatoire National sur les Effets du Réchauffement Climatique (ONERC - National Observatory on the Effects of Climate Change)}



ONERC was created by legislation passed on February 19th 2001. The ONERC has three main purposes:
\begin{itemize}
	\item To collect and spread information on risks related to global warming
	\item To formulate recommendations on adaptation measures to mitigate the effects of climate change.
	\item To be the focal point of the IPCC in France.
\end{itemize}




\textbf{Stratégie nationale d’adaptation au changement climatique (SNACC - National Strategy from Climate Change Adaptation)}



France’s national adaptation strategy was adopted on the 13th of November 2006, based on recommendations from the ONERC. 



It outlines four priority areas for adaptation
\begin{itemize}
	\item Acting to ensure public security and health
	\item Addressing social aspects and inequalities of climate-change risk
	\item Limiting costs and taking advantage of the change
	\item Protecting cultural heritage
\end{itemize}


There are eight strategic action steps developed in the strategy:
\begin{itemize}
	\item Developing scientific knowledge
	\item Consolidating observation systems
	\item Informing and educating all actors
	\item Promoting a regional and community-oriented approach
	\item Financing adaptation actions
	\item Utilizing legislative and regulatory instruments
	\item Taking into consideration the special status of overseas territories
	\item Contributing to international cooperation
\end{itemize}



\textbf{Grenelle Environment}



The Grenelle Environment was a series of political talks initiated by Nicolas Sarkozy in September and October 2007 that brought together representatives of all levels of government, civil society and industry to develop public policy on environmental and sustainable development issues. 
It has led to the following policies and actions in these areas:


Residential Sector
\begin{itemize}
	\item After 2012, all new buildings must have a primary energy consumption of less than 50kWh/m2/year on average. This standard was implemented in after 2010 for all public buildings and for construction under the National Urban Renovation Program. By 2020, all new buildings must have a primary energy consumption that is less than the amount of renewable energy produced in the buildings (energy positive buildings).
	\item Eco-loans at 0\% interest: allow owners to take 10-15 year loan of up to 30,000 euros towards improving the energy efficiency of their property. This program can be combined with other financial support tools.
	\item All public and state owned buildings will undergo an energy performance assessment by 2010,  and renovations will begin on these buildings in 2012 that should result in a 40\% reduction in energy consumption and a 50\% reduction of greenhouse gas emissions within a period of 8 years.
	\item The most energy intensive 800,000 social housing units will be renovated prior to 2020. Loans will be made available at a 1.9\% interest rate between 2009-2010 to allow for the immediate renovation of 100,000 units, and upgrades will continue at a rate of 70,000 units per year.
\end{itemize}



Transportation
\begin{itemize}
	\item 2,000 km of high-speed rail will be built by 2020 and an additional 2,500km will be planned. 
	\item France will meet the EU objective of reducing vehicle emission to 120g CO2/km.
	\item The “bonus-malus” program in place since January 2008 provides a credit for the purchase of low-emitting vehicles  (less than 130g CO2/km) and imposes a tax on purchase of vehicles emitting more than 160g CO2/km.
	\item France had the objective of a 5.75\% biofuel mix between 2001-2008, and increased the target to 7\% in 2010 and 10\% by 2015. To reach these objectives, a general tax on polluting activities (TGAP) will be imposed on operators not respecting this fuel mix ratio and an exemption program on the domestic tax for petroleum products (TIPP) for biofuels will be implemented.
\end{itemize}



Industry
\begin{itemize}
	\item The 2005 directive creating a cap and trade system will be reviewed. This review was adopted by the European Parliament and Council in December 2008. It will allow the implementation period to be extended, to harmonize the system of quota allocation and to reinforce the objectives of reducing greenhouse gas emissions in this sector. This measure will achieve, at the European level a 21\% reduction of emissions in this sector between 2005-2020 (1990 levels).
\end{itemize}



Energy
\begin{itemize}
	\item Certificates of Energy Efficiency, in place since 2006, will be expanded.
	\item The “Ecoconception” Directive will be implemented:
	\item Completely retiring incandescent light bulbs by 2012.
	\item Limiting the consumption of single digital decoders to 1W by 2010 and 0.5W by 2012.
	\item Improving the performance of electric chargers and external power supplies.
	\item Developing renewable energy to achieve 23\% mix in energy consumption by 2020 by increasing the annual production of renewable energy by 20 million tons of oil equivalent.
	\item Renewable Heat Fund (Fonds chaleur renouvelable): this program created a fund of 1 billion euros for 2009-2011 to develop renewable sources such a wood, geothermal, and solar to be used for heating in the tertiary sector and in industry,
	\item Tax credit for sustainable development that promotes the purchase of solar water heaters and solar panels was extended until 2012.
	\item The construction of new biomass plants with a capacity of 250MW.
	\item Increasing the capacity of geothermal energy sixfold by 2020, by providing 2 million homes with heat pumps.
	\item A fixed tariff for wind energy and improving the planning and consultation process for new wind turbines; simplification of the process for developing off-shore wind energy.
	\item 1 billion euros will be devoted to research into sustainable development. 
\end{itemize}



For solar energy
\begin{itemize}
	\item Building a solar plant in each French region for a cumulative power of 300 MW, supporter by simplified tariffs to secure long term investment.
	\item Creating a 45 euro cent/kWH tariff to facilitate the installation of solar panels on private buildings.
	\item Reducing the administrative and financial steps when panels do not exceed 30 m2.
	\item Increasing the scope of the public buildings that are eligible for the reduced tariff for purchasing electricity produced from renewable sources.
	\item Construction permits cannot oppose the installation of renewable energy production systems on buildings.
\end{itemize}

% End of section from Jon and Emily - Converted into LaTeX by Milan 2013-04-03


		\subsection{Actions taken by international bodies}
		
\begin{vcom}
	Substantially more information and examples could be added here.
\end{vcom}

International efforts to address climate change have often been centred around the United Nations Framework Convention on Climate Change (UNFCCC), though many other international forums and organizations have also made efforts to address the issue.

United Nations Framework Convention on Climate Change (UNFCCC)
\begin{itemize}
	\item Signed in 1992, came into force in 1994 with 50 ratifications
	\item Objective: ``stabilize greenhouse gas concentrations in the atmosphere at a level that would prevent dangerous anthropogenic interference with the climate system''
	\item This has subsequently come to be understood to mean limiting warming to less than 2˚C
\end{itemize}
	

\begin{vcom}
	Other sub-components of UNFCCC – adaptation funding, etc
\end{vcom}	






Canada has repeatedly endorsed the 2˚C limit for warming
\begin{itemize}
	\item The 2009 Copenhagen Accord - signed by Canada - recognizes ``the scientific view that the increase in global temperature should be below 2 degrees Celsius''
	\item \textsf{Other references to the limit}
	\item For the world to reach this goal, fossil fuels need to be phased out aggressively
\end{itemize}
	
% END SECTION 5 - MILAN