% BEGIN SECTION 6 - MONICA
% Last updated by Milan 2013-06-14


		\singlespacing
		\section{Why start with Royal Dutch Shell?}
		\label{sec:Shell}
		\doublespacing



\begin{vcom}
This is a section where we need to be very careful to maintain a dispassionate tone and to include strong evidence supporting our factual and legal claims. Ideally, we want impeccably-documented examples of wrongdoing by Shell, along with information on which laws they have violated.
\end{vcom}



The preceding sections of this brief show how climate change is causing of increasing instability and harm to the planet.
Consequences of climate change have the potential to cause unprecedented levels of social and environmental damage, unless dramatic policy changes are put into place rapidly. 
One important dimension of that response is the redirection of large amounts of investment away from new fossil fuel developments.
The extraction and burning of fossil fuels are the direct causes of climate change, and of all the adverse consequences that follow from it. 
Investing in the fossil fuel industry fuels the continuation of these activities and thus constitutes deriving financial benefit from harmful activity.
This is not in accordance with socially or ethically responsible investment practices.\footcite[][]{Richardson_2008}



As a first step toward divestment from the fossil fuel industry, we ask that the University of Toronto divest 100\% of its holdings from Royal Dutch Shell by the end of 2013. 
As one of the largest fossil fuel companies in the world, as well as the university's largest single holding, Shell represents an ideal starting point for the university's move to divest from the fossil fuel industry.
		

			
In addition to the broader implication of Shell's activities as directly contributing to the harmful affects of climate change, Shell represents an ideal case for divestment due to three main reasons  related to the operations of the company: 



\begin{enumerate}
	\item Shell has repeatedly carried out actions resulting in social injury, including conduct in Nigeria and Alberta that conflicts with domestic and international law.
	\item Shell represents a financial risk to investors, with even greater shareholder uncertainty in the medium- and long-term due to proposed projects that are costly and high-risk.
	\item Divestment from Shell will not adversely affect the university's portfolio.
\end{enumerate}



	\subsection{Shell's ongoing history of social injury}



Royal Dutch Shell has been found to repeatedly and willfully cause social injury as a result of activities that:
\begin{enumerate}
	\item Directly conflicted with national and international law, and
	\item Infringed on governmental regulations or on international health and safety or environmental standards. 
\end{enumerate}
The following list of legal actions taken against Shell demonstrates that the company has consistently and knowingly inflicted social harm as a consequence of a number of its global operations.



	\subsubsection{Legal offences in Nigeria}



Shell has a long history of human rights and environmental abuses in the Niger Delta region, where it has operated since 1958.
In 2012 alone, 198 oil spills took place at Shell facilities in Nigeria, releasing 26,000 barrels of oil.\footcite{DutchResponsible}
A United Nations report in 2011 determined that cleaning up mangroves contaminated by Shell would take 30 years and at least \$1 billion.\footcite{Ogoniland}
In the same report, it was identified that some families in the area were consuming water with benzene levels 900 times higher than those considered safe by the World Health Organization.\footcite{OilSpoils}



Currently, Shell is the midst of a number of litigation processes at various stages, as documented in Shell's 2011 Annual Report:
\begin{quote}
Shell subsidiaries and associates operating in Nigeria are parties to various environmental and contractual disputes.
These disputes are at different stages in litigation, including at the appellate stage, where judgments have been rendered against Shell. 
If taken at face value, the aggregate amount of these judgments could be seen as material.\footcite[][P.138]{Shell_2011}
\end{quote}
Since the publication of the report, Shell has been found responsible for oil pollution and ordered to pay compensation by a Dutch court in \emph{Niger Delta Farmers vs. Shell} (detailed below).\footcite[See also:][]{DutchResponsible}
The parties are currently in the process of negotiating compensation.
The full financial impact of the company's ``environmental and contractual disputes'' in Nigeria is not yet known and could have a material effect on the company.\footcite[][p. 139]{Shell_2011}
In 2011, \emph{The Economist} argued that ``[l]egal pressures on the company are increasing'' and that Shell has ``paid out \$1.7m in compensation to groups in the delta affected by spills''.\footcite{OilSpoils}


The following is a partial list of legal challenges to Shell's activities in Nigeria over the span of approximately fifteen years:



\textbf{Bodo vs. Shell:}



This case is ongoing. 
It was first brought before the High Court in London on June 18th 2012, and will be heard sometime this year. 
Shell is challenged by 11,000 members of the Niger Delta Bodo community, who say the company is responsible for spilling approximately 500,000 barrels of oil in 2008. 
Shell has admitted liability for two spills in the Bodo region.\footcite{DutchResponsible}



\textbf{Kiobel v. Royal Dutch Petroleum Co.:} 



This case is ongoing. 
It was first brought before the United States Supreme Court and granted petition on October 17th 2011. 
Nigerian plaintiffs accuse Royal Dutch Shell and its affiliate Shell Transport and Trading Company PLC of providing transportation and payments to government forces who committed crimes against humanity in the Ogoni region, including the arrest, torture and murder of protestors challenging Shell operations.\footcite{KiobelShell} \footcite{TortureComplicity}
This case involves activities included in the \emph{Wiwa v. Royal Dutch Shell Co.} cases described below.



\textbf{Niger Delta Farmers vs. Shell Oil Company:} 



A verdict for this case was reached in January 2013. 
Shell Nigerian subsidiary, Shell Petroleum Development Company of Nigeria Ltd. (SPDC), was sued in a Dutch court by four farmers and the environmental organization Friends of the Earth on October 10th 2012. 
SPDC was found responsible for oil spills in Niger Delta on one of four counts and ordered to pay compensation to Nigerian farmer Friday Akpan for incidents occurring in 2004, 2005, 2007.\footcite{OrderedFarmer}
Compensation is being negotiated. 
\emph{The Economist} argued that ``[t]he ruling could open a flood-gate to legal complaints against oil companies''.\footcite{MixedVerdict}



\textbf{Wiwa v. Royal Dutch Shell Co.:}



In 1993, Ken Saro-Wiwa took part in a march by 300,000 Ogoni people, demanding a share in oil revenues and increased political autonomy.
Following a trial by a military tribunal, Saro-Wiwa and eight other Ogoni leaders were hanged in 1995.
According to the United Nations Environment Programme: ``Continued social upheaval in the area further alienated the Ogoni community from SPDC [Shell Petroleum Development Company (Nigeria) Ltd], and MOSOP [Movement for the Survival of the Ogoni People] has since been campaigning for the total expulsion of Shell from Ogoniland.''\footcite[][p.27]{Ogoniland}
Saro-Wiwa was identified as a ``prisoner of conscience'' by PEN Canada and Amnesty International, and international appeals and global outrage took place in response to the tribunal's decision.



Under the \emph{Alien Tort Statute}, the \emph{Torture Victim Protection Act} of 1992, and \emph{Racketeer Influenced and Corrupt Organizations Act} (RICO), the Wiwa family has brought three lawsuits against Royal Dutch Shell, its Nigerian subsidiary, and the CEO of that subsidiary in the United States District Court for the Southern District of New York.
These cases involved the hanging of Saro-Wiwa and eight others, the detention and torture of Owens Wiwa and Michael Tema Vizor, and the shooting of Karololo Kogbara while she peacefully protested the bulldozing of her crops to permit the construction of a Shell pipeline.
The plaintiffs alleged that the executions were carried out with the ``knowledge, consent and/or support'' of Shell.\footcite{ShellTrialDelayed}
They also alleged that Shell was paying soldiers involved in human rights abuses in the region.\footcite{Shell15Million}
Shell settled legal action out of court with a payout of \$15.5 million dollars.\footcite{Shell15Million}
The settlement is one of the largest payouts by a multinational corporation charged with human rights violations to date and speaks to the company's complicity in these activities.



Notably, four years after the execution of his father, Ken Saro-Wiwa Jr. was a Visiting Fellow and Senior Resident at the University of Toronto's Massey College.
The college possesses the final handwritten letter Saro-Wiwa Sr. wrote to his son, prior to his execution, along with a handmade chair crafted by Saro-Wiwa Sr. and inscribed with a poem criticizing Shell.\footnote{See: \url{http://www.flickr.com/photos/sindark/7947170092/in/set-72157631444407428}}



\textbf{US Dept. of Justice vs. Panalpina, Shell, et al.}



In 2010, Shell was implicated in a case brought against Panalpina, a Swiss-based company that provides international air and ocean freight, by the U.S. Department of Justice. 
Panalpina was implicated in foreign bribery charges by US regulatory bodies and settled on a total of \$85 million over these allegations. 
Royal Dutch Shell and five other oil companies were also implicated and charged along with Panalpina, paying a total of \$246 million in penalties altogether. 
As stated by Robert Khuzami, the Director of Enforcement for the US Securities and Exchange Commission (SEC), ``These companies resorted to lucrative arrangements behind the scenes to obtain phoney paperwork and special favors, and they landed themselves squarely in investigators’ crosshairs.'' \footcite[][p. 119]{KochanGoodYear_2011} 
The case is significant as setting potential precedents of vigilance for global companies that utilize external contractors in parts of the world ``where resources are plentiful but the rule of law is shaky.''\footcite[][]{Bribery_2010}



With respect to Shell's role, the company was implicated in corrupt activities that took place in Nigeria and included the expedition of services such as clearing drilling rigs and other equipment through customs (more specifically, using a customs broker to pay officials to acquire special treatment for a project conducted in Nigeria).\footcite[][]{ShellBribes_2010}
Shell consented to pay a disgorgement of \$18.15 million and a criminal fine of \$30 million.\footcite[][]{SullivanCromwell_2010}
Shell subjected to a Deferred Prosecution Agreement (DPA) with the U.S. Department of Justice (DOJ) for violations of bribery and books provisions of the Foreign Corrupt Practices Act (FCPA).\footcite[][p.17]{Shell_2011}
Shell also consented to a Cease and Desist Order from the U.S. Securities and Exchange Commission (SEC) on account of record keeping violations and internal control provisions of the FCPA. 
As a result, the DPA outlined an ethics program designed to prevent and identify any breach of the FCPA as well as any other applicable anti-corruption laws corresponding to all aspects of Shell’s operations. 
The program also calls for Shell to immediately report any evidence of questionable activity to the DOJ. 
As stated in Shell's 2011 annual report, such activity could have a significant impact on the company: ``Any violations of the DPA, or of the SEC's Cease and Desist Order, could have a material adverse effect on the Company.''\footcite[][]{Shell_2011}



\textbf{Gas flaring}



Since 2005, Shell has refused to comply with a Federal High Court order to end gas flaring in the Iwherekan community in Nigeria. 
Shell is also avoiding payment of \$1.5 billion in compensation to the Delta’s Ijaw ethnic group for decades of pollution.\footcite[][]{Ukala_2011}



\textbf{Oil spills}



Going forward, Shell faces thousands of claims related to oil spills in Nigeria, and charges in the most recent case (\emph{Niger Delta Farmers vs. Shell Oil Company}) opens doors for further legal actions.\footcite[][]{MixedVerdict}



	\subsubsection{Infringements on governmental regulations and international health and environmental standards with respect to operations in Nigeria}
	
	
	
The release of the Assessment of the Environment of Ogoniland by the United Nations Environment Programme (UNEP) on August 4th 2011 confirmed the devastating extent of pollution in the minority Ogoni region. 
The estimated time required for clean-up is between 25 to 30 years. 
The UN condemned Shell for failing to comply to its own operating standards and for under-reporting pollution.\footcite[][]{Ogoniland}



The same UN report also confirms that all water bodies in Ogoniland are polluted with hydrocarbons and reveals that benzene, a known carcinogen, is concentrated at a level 900 times above World Health Organization standards for safe drinking water.



Shell has repeatedly ignored Nigerian federal law (and its own internal policies) calling for regular inspection and maintenance and upgrading of pipelines and production facilities, as well as and prompt and effective response to oil spills.\footcite[][]{Steiner_2008}  \footcite[][]{Steiner_2010}



	\subsubsection{Legal offences in Alberta}
	


Shell is one of the biggest players in developing the Athabasca oil sands, with 249,000 barrels per day of production from its Scotford upgrader.\footcite{AlbertaEnergy}
Shell Canada currently operates the Alberta Oil Sands Project (AOSP), which consists of the Albian Sands Mine, Muskeg River Mine, Jackpine Mine, and the Scotford Upgrader. 



The AOSP is close to a number of First Nations communities who claim that the project adversely affects their health, livelihood, and lands.\footcite[][]{RiskingRuin_2012}  
Under Canadian constitutional law, there is a duty to consult and accomodate aboriginal people on development projects that affect them.
There are currently a series of legal proceedings related to tar sands developments launched by First Nations that could impact the viability of Shell's current and future operation plans, as outlined below:
\begin{description}
	\item [Ongoing - ACFN vs. Shell Canada] In continuing legal battles throughout 2011-2012, the Athabasca Chipewyan First Nations (ACFN) sued Shell Canada for breach of terms of agreements made in 2003 and 2006 regarding the company's existing tar sands mines.\footcite{FNSues} The ACFN alleges that Shell has not honoured these agreements and that these breaches have allowed Shell's operations to continue damaging the surrounding environment and the infringing upon the rights of ACFN peoples. Affected First Nations communities continue to seek legal options to delay or halt Shell's operations in the AOSP.
	\item [2009 - Ecojustice vs. Shell Canada] Ecojustice, an environmental organization, took Shell to the Alberta Court of Appeal after Shell breached signed commitments with the government of Alberta to reduce carbon emissions for the Jackpine and Muskeg River mines. Alberta courts instructed regulators to ignore the breach. However, the ruling has prompted both residents and elected officials in Alberta to demand an overhaul of regulatory approval processes in the province.\footcite[][]{RiskingRuin_2012}
	\item [Cases related to groundwater contamination] As described in their 2011 Annual Report, Royal Dutch Shell (including subsidiaries), has been sued repeatedly by public and semi-private water purveyors, as well as governmental bodies, who insist that Shell take responsibility for groundwater contamination in various instances.\footnote{One barrel of surfaced-mined oil from tar sands extraction requires 2-4 barrels of freshwater and creates about 1.5 barrels of toxic waste. This waste is held in `tailings ponds', which covered $176 km^2$ in 2010 and contained 830 billion litres of toxic waste. Shell's tailings ponds cover $23 km^2$ and contain millions of litres of toxic waste. Each day, 11 million litres of waste leaks into the Athabasca River from tar sands operations. These toxins are known carcinogens and leaks have had devastating impacts on human and ecological health.} 
As outlined in the Annual Report, at the end of 2011, fewer than 10 of these cases remained open, with the remaining cases in various stages of litigation. The number of allegations made by numerous public and private entities, including governmental agencies, speaks to Shell's consistent negligence in ensuring environmental safety. While groundwater cases remain ongoing, a study published by Alberta Health in 2008 confirmed a 30\% rise in the number of cancers between 1995 and 2006 in the community of Fort Chipewyan,\footcite[][]{RiskingRuin_2012} providing scientific evidence supporting the appeals of First Nations residents that AOSP activities were polluting the surrounding environment.\footnote{This study, however, lacks appropriate data and is considered a conservative estimate by many residents.} An internal government memo, obtained by journalist Mike De Souza by virtue of Access to Information legislation, confirms groundwater toxins related to bitumen mining and upgrading are seeping from tailings ponds and contaminating groundwater. These toxins are not naturally occurring, contrary to statements made by government and industry.\footcite[][]{Memorandum_2012} \footcite[See also: ][]{TailingsLeaking}
\end{description}


Alongside the promise of future legal conflicts as a result of the company's activities in Nigeria, more legal challenges are almost certain to arise as First Nations communities continue to oppose Shell's operations. 



The International Finance Corporation's (IFC) implementation of a new Sustainability Framework, which requires clients of Equator Principle banks to obtain the free, prior and informed consent of indigenous communities impacted by mining projects, poses a significant obstacle to the company going forward where they conflict with the interests of various First Nations communities.\footcite[][]{Sosa_2011}
In addition, the International Covenant on Civil and Political Rights states that: ``In no case may a people be deprived of its own means of subsistence.''\footcite[Part I, Article I (2)][]{CivilAndPolitical}
Shell violates this obligation by making the waters of the Niger Delta unsafe to drink, and threatens to do so further by contaminating the food and water of aboriginal communities in Alberta and the arctic.


	\subsubsection{Continued threats to human rights, environmental well-being and international law}
	

	
Court rulings in cases brought against Shell over the past fifteen years have resulted in determinations of guilt, out of court settlements, and case dismissals. 
In the case of Shell, one can reasonably argue that where there is smoke there is fire --- that is, regardless of the outcomes of individual legal challenges, the sheer volume of allegations against the company stand as evidence of Shell's willful and repeated acts of social injury.
Shell's record of being the target of lawsuits raises the question of whether this investment represents the values of the university of Toronto, in addition to being a material risk to the company's profitability going forward.
Moreover, a review of Shell's most recent activities and the projects it has slated for the immediate future indicates that Shell will continue to engage in activities that constitute human rights abuses and environmental degradation.
For instance:
\begin{enumerate}
	\item In the summer of 2011, Shell supported Syrian President Bashar al-Assad's regime by contributing over \$55 million during government crackdowns.\footcite[][]{Syria_2011} Moreover, Shell continued drilling and exporting crude oil from Syria throughout the first year of the popular revolt and did not halt operations until Western imposed oil sanctions and global outrage forced them to withdraw from the country on the December 2nd 2011.
	\item Both Shell's current activities and its proposed projects in the arctic will threaten local First Nations communities such as the Inupiat who live around the Beaufort and Chukchi Sea and who practice a subsistence culture, both by tradition and by necessity.\footcite[][p. 13]{RiskingRuin_2012}
\end{enumerate}
In May 2013, members of the Native Village of Port Hope, Alaska and the Athabasca Chipewyan First Nation (ACFN) participated in Shell's Annual General Meeting and confronted Shell's chairman about the risks of drilling in the arctic.\footcite{ShellArcticOilRush}
In addition to the direct threat posed to people living in the region, arctic drilling risks adding to the already dangerously large reserves of fossil fuels being exploited globally, contributing further to the universal threat of climate change.


	\subsection{Shell represents financial risk}
	


Royal Dutch Shell or any of its subsidiaries are a risky investment for two main reasons:
\begin{enumerate}
	\item Previous violations of human rights and environmental regulations may ultimately have a material affect on the company; these same activities can also manifest in decreased shareholder confidence.
	\item High-risk ventures going into the medium and long term introduce uncertainty on a number of levels.
\end{enumerate}

	
	
	\subsubsection{Poor reputation for social responsibility lowers shareholder confidence}
	

	
Shell's reputation for complicity in human rights and environmental degradation has resulted in lowered shareholder confidence and has prompted socially conscious investors to avoid holding Shell stock.
For instance, the Dow Jones Sustainability Index, which integrates assessment of economic, environmental and social criteria with emphasis on long-term shareholder value, excluded Shell from the Index in both 2010 and 2011 following concerns about the company's activities in Nigeria (which include both human rights and environmental abuses).\footcite{ShellIndex}
Shell's European Universe was included in the 2012 Index, but all others remain excluded (including the North American, Asia Pacific, Aussie, Emerging Markets, Korean Universes).


In February of 2012, 28 Right Livelihood Award Laureates including conservation scientists and professionals petitioned the Norway Government Pension Fund to divest all its holdings in Royal Dutch Shell. 
The petition was made after this group, in collaboration with numerous other Nigerian scientists and communities, found the Delta to be ``one of the most severely oil-impacted ecosystems in the world.''\footcite[][]{NigerDeltaReport_2006}
This collaborative team of scientists and Nigerian residents that led to the 2011 UNEP assessment discussed above.
As stated on their petition, the argument for divestment is based on the company's ``willful negligence'' which resulted in the extensive environmental harm found in the Niger Delta region.\footcite[][]{NorwayPetition_2012}



	\subsubsection{Fossil fuel extraction in the arctic represent particularly high-risk and unpredictable endeavours}



The arctic is experiencing some of the most profound and rapid effects of climate change.
World renowned physicist and oceans expert Peter Wadhams calls the situation in the arctic a ``global disaster,'' observing that ice is disappearing at a faster rate than previously predicted.\footcite{Wadhams2012}
The IPCC has observed that: ``Average Arctic temperatures have increased at almost twice the global average rate in the past 100 years.'' \footcite[][Climate Change 2007: Synthesis Report - Observed changes in climate and their effects]{IPCC2007}
The rapid warming of the arctic has global consequences, as vanishing sea ice is replaced with darker water and more energy and heat are absorbed by the Earth from the sun as a consequence.


Despite the growing body of accepted scientific facts that point towards the significant and unpredictable consequences of a melting arctic, Shell has spent over \$4.5 billion on operations and lease purchases in the far north, taking advantage of the climate impacts in the arctic to secure further exploration and drilling.\footcite{ShellSuspends}
Because Shell's production has been decreasing for the past 10 years --- with the exception of a 5\% increase in 2010  --- booking new reserves is of primary importance for the company.
This is driving Shell to invest in more pollution-intensive forms of oil, such as those in Canada's oil sands, as well as oil reserves that are riskier to extract, such as those in the arctic.
Shell's Alaskan project alone accounted for about one-seventh of Shell's total exploration spending in 2011. 
While arctic extraction projects represent a new branch of growth for the company, these projects are also risky for shareholders for four reasons:



\textbf{High costs} 



Unconventional methods of extracting oil, especially in harsh and isolated regions such as the arctic, are extremely costly due to technological requirements, human resources, costs of spill cleanups, and other related expenses. 
For example, recent projects such as Shell's Sakhalin-2 project in Russia saw an unexpected cost overrun from \$6 - \$22 billion dollars in 2006.\footcite[For more on these four central risk factors see: ][]{OutInTheCold_2012}



Moreover, recent incidents have occurred casting doubt on Shell's capacity to safely undertake arctic operations. 
Sixteen distinct and serious safety and environmental violations were discovered on the Noble Explorer --- a Shell drilling rig anchored in the arctic waters off Alaska. 
The UK Coast Guard inspected the rig and reported findings of ``systematic failure and lack of main engine preventative maintenance.''\footcite{RigFindings} \footcite[See also: ][]{ArcticUnsafe}
These findings have been turned over to the U.S. Department of Justice and U.S. federal prosecutors have been asked to take legal action over these violations as of late February, 2013. 



\textbf{Arctic projects are dependent on a favourable political climate}



An interaction of soaring costs, uncertainty related to project completion, and popular resistance against drilling in sensitive regions such as the arctic may lead to difficulties securing subsidies or tax breaks from governments.
Shell has ``spent several years on an intensive lobbying campaign to persuade federal officials that it could drill safely'' in the arctic.\footcite{ShellSuspends}
The credibility of this claim is diminished by Shell's inability to operate drilling rigs safely in the region.
Furthermore, the Deepwater Horizon disaster in the Gulf of Mexico has raised awareness among policy-makers and the general public about the risks involved in oil extraction within extreme environments, as well as the inability of major oil firms to rapidly contain serious spills when they occur.



\textbf{Lack of oil spill plan}



There is currently no proven method to clean up an oil spill in the remote and extreme arctic landscape, nor are there many resources available for such an event. 
A 2011 report from top scientists at the U.S. Geological Survey confirm that not enough is known about the arctic's unique marine environment to ensure an adequate or sufficient clean up plan in the case of an oil spill. 
As asserted in this survey, this lack of knowledge presents a ``major constraint to a defensible scientific framework for critical Arctic decision making.''\footcite[][]{Holland-BartelsPierce_2011}



Shell's inability to operate drilling equipment safely in the arctic is highlighted by the damage suffered by the company's two drilling ships --- Kulluk and Noble Discoverer --- which were attempting to operate in the region.\footcite{ShellSuspends}
As a result of these incidents, Shell has canceled its arctic drilling plans outright through 2013, though it has stated its intention to resume arctic drilling in future seasons.



\textbf{Funding challenges}



The social and environmental responsibility guidelines of international financial institutions (IFIs) and signatories to the Equator Principles --- the voluntary set of standards for assessing and managing social and environmental risk --- have delayed or halted funding for frontier extraction projects in the past. 
For example, the European Bank did not solicit funds in 2003--2006 for the Reconstruction and Development (EBRD) of Shell's Sakhalin-2 due to serious breaches of their environmental and sustainability guidelines. 
Growing frustration and resistance on behalf of First Nations communities and the implementation of new IFC guidelines promise more delays on account of legal challenges posed by affected communities.\footcite[][]{Mathiason_2005}



\textbf{Supporting industry-based evidence of risks to Shell's ongoing and proposed projects}



\begin{enumerate}
	\item On March 8th, 2013, Norwegian state-owned oil company, Statoil, announced that it is slowing plans to drill for oil in U.S. arctic waters after Shell's most recent string of incidents in arctic.
	\item German bank WestLB announced it would not invest in any company drilling in the arctic because the ``risks and costs are simply too high.''\footcite[][]{Naidoo_2012}
	\item Total, the French oil company, has disavowed drilling in the arctic; CEO Christophe de Margerie claims that ``Oil on Greenland would be a disaster ... A leak would do too much damage to the image of the company.''\footcite[][]{NewsWire_2012}
	\item Growing resistance around production of tar sands puts operations there at risk, as Shell has already faced shareholder resolutions demanding greater clarity over the risk of tar sands investments.
	\item The Carbon Bubble: Latest climate science tells us that approximately 80\% of reserves owned by fossil fuel companies cannot be burned. As policy-makers and the financial industry absorb this information, the business case for drilling in extreme environments and exploiting unconventional reserves may be undermined.\footcite[][]{carbontracker}\footcite[][]{UNEP_2009}\footcite[][]{HSBC_2013}
\end{enumerate}


	\subsection{Divestment from Shell would not hurt the university financially}

\begin{vcom}
		Get a working anchor for the link in part 3
\end{vcom}

% \hypertarget{TobaccoPrecedentFinance}

Shell constitutes the university's largest single holding at approximately \$9.48 million dollars as of March 2012.\footcite[][]{UTAM_2012}  
At the same time, it represents about 1\% of the institution's total known endowment.
When the university divested from the tobacco industry its holdings in the industry were calculated at 1.6\% of US equities and .98\% of international equities, for a total weight in tobacco stocks of 2.28\%.\footcite[][]{TobaccoReport_2007}
In the case of tobacco, the \emph{ad hoc} committee assembled by the President to consider divestment concluded that selling tobacco stocks  would not adversely affect the university's portfolio. 
Given the lesser size of the university's investment in Shell, it is likely that divestment in this case will also not damage the university financially.



\begin{vcom}
CAN SOMEONE PLEASE READ THE BLURB BELOW (taken from the Advisory Board Report explaining their decision for tobacco divestment) AND MAKE SURE I AM RIGHT IN THIS LAST PARAGRAPH? I’M REALLY SLOW WHEN IT COMES TO THIS KIND OF STUFF AND I WANT TO MAKE SURE I’M NOT JUST MAKING THIS UP!

	Milan - I need to get this document from Monica before I can assess the following paragraphs. It doesn't seem to be available via Google. Also, everything about why shareholder activism is inadequate should probably go into part 3, possibly with a hyperlink here. 
\end{vcom}



In its assessment of the impact of prohibiting University funds on tobacco investments, UTAM indicated that it could implement the divestment of tobacco stocks only for Separate Accounts with minimal management and monitoring costs expected. 
While Separate Accounts constitute 54\% of total investments, as of the end of July 2006, the amount of equities in Separate Accounts as a proportion of total investments is 28\%. 
The amount of tobacco stocks that could be divested would be a very small portion of this 28\%.



The case for Pooled Accounts which make up 46\% of total investments is a different matter. 
Divestiture of all Pooled Account holdings would involve significant time and costs as it would require a fundamental change in the UTAM business model for managing the university's investments. 
It is not feasible for the university to change its investment structure to move away from pooled funds, given that it is substantially less expensive to manage, and so provide a higher expected return, net of costs. 
Thus the ``prudent investor'' rule would only permit divestment from tobacco stocks held in separate accounts.

\begin{vcom}
	Get a working link to the tobacco precedent section in part 6 here
\end{vcom}

% For information on why fossil fuel divestment is similar to tobacco divestment in terms of social injury, see: \hyperlink{TobaccoPrecedentInjury}{Why fossil fuels are like tobacco}






The university's Policy on Social and Political Issues With Respect to University Divestment contemplates the possibility of the university voting proxies and attempting to influence the conduct of companies in which it invests (``shareholder activism'').
Some other institutions, including Yale, do so. 
The University of Toronto has neither the resources nor the jurisdiction to monitor the business practices of tobacco companies, or any other enterprises. 
The university must devote its resources to its mission of research and teaching. 
Nor is the university a large enough investor to shape the conduct of companies in which it invests. 

\begin{vcom}
	This seems like a weak ending for this section
\end{vcom}


% END SECTION 6 - MONICA