		
		\subsubsection{United States}
		
		
		
\textbf{Federal government}


At the federal level, the White House under the leadership of President Obama has taken many steps towards mitigating and adapting to climate change.\footcite[][]{WHcc2013} \footcite[][]{WHadaptation2013} \footcite[][]{WHenergy}
\begin{description}
	\item [Monitoring Emissions] The United States is comprehensively cataloguing greenhouse gas emissions from its largest emitting sources.\footcite[][]{GHGReporting}
	\item [Government Procurement and Energy Consumption] President Obama directed the Federal Government – the largest energy consumer in the U.S. economy – to reduce its greenhouse gas emissions from direct sources such as building energy use and fuel consumption by 28 percent by 2020.\footcite[][]{FedEmissionReduction2010} He also directed Federal agencies to reduce their greenhouse gas emissions from indirect sources, such as those from employee commuting, by 13 percent by 2020.\footcite[][]{FedEmissionReduction2010}
	\item [Creation of the Climate Change Adaptation Task Force (CCATF) and the U.S. Global Change Research Program (USGCRP)] The CCATF recommends how federal agency policies and programs can better prepare the United States to address the risks associated with a changing climate.\footcite[][]{WHadaptation2013} The USGCRP is a collaborative effort involving 13 federal agencies to evaluate the current and future impacts of climate change, inform policy-makers and the public about scientific findings, and investigate effective ways to reduce greenhouse gas emissions and deploy cost-effective clean energy technology.\footcite[][]{USGCRP}
	\item [Investing in Clean Energy] With the support of administration policy, the U.S. has nearly doubled renewable energy generation from wind, solar, and geothermal sources since 2008.\footcite[][]{WHEnergyStrategy} Since 2009, the Department of Interior has approved 29 onshore renewable energy projects, including 16 solar, 5 wind, and 8 geothermal projects.\footcite[][]{WHEnergyStrategy} Moving forward, the Department of Interior is committed to issuing permits for 10,000 megawatts of renewable power on public lands and in our offshore waters by the end of 2012, enough to power 3 million homes.\footcite[][]{DOIWindPower} In 2010, President Obama also set a goal of breaking ground on at least four commercial scale cellulosic or advanced biorefineries by 2013.\footcite[][]{WHEnergyStrategy} That goal was accomplished a year ahead of schedule.\footcite[][]{WHEnergyStrategy}
	\item [Smart Grid] In 2011, the Administration announced that it would accelerate the permitting review of seven proposed electric transmissions lines.\footcite[][]{GridModernization} These infrastructure projects, when built, will increase grid capacity, facilitating better integration of renewable energy sources, avoiding blackouts, and helping to accommodate the growing number of electric vehicles on the road.\footcite[][]{GridModernization} The Administration also launched a Green Button initiative in 2011 to empower Americans to reduce energy use in their homes.\footcite[][]{GreenButton} Already, utilities across the country have committed to providing 27 million households with access to data about their own energy use with a simple click of an online ``Green Button'' that will help them reduce waste and shrink their energy bills.\footcite[][]{GreenButton}
	\item [Clean Energy Research \& Development] In 2009, the Administration funded the Department of Energy’s Advanced Research Project Agency-Energy (ARPA-E), which focuses on “out-of-the-box” transformational energy research that brings together the nation’s best scientists, engineers, and entrepreneurs.\footcite[][]{ARPAE} Building upon the initial investment, in late September 2011, the ARPA-E program announced 60 cutting-edge research projects in 25 states.\footcite[][]{ARPAEProjects} In total, The ARPA-E has supported more than 120 individual projects.\footcite[][]{ARPAEProjects}
	\item [Clean Energy Innovation Hubs] The Administration also launched a series of clean energy innovation hubs, which bring together teams of the best researchers and engineers in the United States to solve major energy challenges.\footcite[][]{EnergyHubs} The hubs will focus on improving batteries and energy storage, reducing constraints from critical materials, developing fuels that can be produced directly from sunlight, improving energy efficient building systems design, and using modelling and simulation for advanced nuclear reactor operations.\footcite[][]{Hubs}
	\item [The President's Better Buildings Challenge] The President's Better Buildings Challenge has set a goal to improve the energy efficiency of commercial buildings by 20 percent by 2020.\footcite[][]{BetterBuildings} The Administration has also partnered with manufacturing companies, representing over 1,400 plants, to improve energy efficiency by 25 percent over 10 years.\footcite[][]{BetterBuildings}
\end{description}	


\textbf{The Environmental Protection Agency (EPA)}
	
The EPA develops standards for greenhouse gas emissions from mobile and stationary sources under the Clean Air Act.\footcite[][]{EPAAirEnforcement} Its federal regulatory activities are in addition to its volunteer programs, international partnerships, and partnerships with states and tribes.\footcite[][]{EPAVolunteer}\footcite[][]{EPAInternational}
\begin{description}
	\item [Standards to Cut Greenhouse Gas Emissions and Fuel Use for New Motor Vehicles] The EPA is enabling the production of a new generation of clean vehicles --- from the smallest cars to the largest trucks --- through reduced greenhouse gas emissions and improved fuel use.\footcite[][]{EPAinitiatives} Together, the enacted and proposed standards are expected to save more than six billion barrels of oil through 2025 and reduce more than 3,100 million metric tons of carbon dioxide emissions.\footcite[][]{EPAinitiatives}
	\item [Renewable Fuel Standard Program] A set of regulations to ensure that transportation fuel sold in the United States contains a minimum volume of renewable fuel.\footcite[][]{EPAinitiatives} By 2022, the Renewable Fuel Standard (RFS) program will reduce greenhouse gas emissions by 138 million metric tons, about the annual emissions of 27 million passenger vehicles, replacing about seven percent of expected annual diesel consumption.\footcite[][]{EPAinitiatives}
	\item [Proposed Carbon Pollution Standard for New Power Plants] On March 27, 2012, EPA proposed a Carbon Pollution Standard for New Power Plants that would, for the first time, set national limits on the amount of carbon pollution that power plants can emit.\footcite[][]{EPAinitiatives} The proposed rule, which applies only to new fossil-fuel-fired electric utility generating units, will help ensure that current progress continues toward a cleaner, safer, and more modern power sector.\footcite[][]{EPAinitiatives}
	\item [Oil and Natural Gas Air Pollution Standards] On April 18, 2012, EPA finalized cost effective regulations to reduce harmful air pollution from the oil and natural gas industry, while allowing continued, responsible growth in U.S. oil and natural gas production.\footcite[][]{EPAinitiatives} The final rules are expected to yield a nearly 95 percent reduction in VOC emissions from more than 11,000 new hydraulically fractured gas wells each year.\footcite[][]{EPAinitiatives} The rules will also reduce air toxics and emissions of methane, a potent greenhouse gas.\footcite[][]{EPAinitiatives}
	\item [Geologic Sequestration of Carbon Dioxide] The EPA has finalized requirements for geologic sequestration, including the development of a new class of wells, Class VI, under the authority of the Safe Drinking Water Act's Underground Injection Control Program.\footcite[][]{EPAinitiatives} 
\end{description}



The EPA is also taking adaptation measures. These include:	
\begin{itemize}
	\item The Climate Ready Estuaries program works with the National Estuary Programs and the coastal management community to: (1) assess climate change vulnerabilities, (2) develop and implement adaptation strategies, and (3) engage and educate stakeholders.\footcite[][]{EPAwater} 
	\item The EPA's Climate Ready Water Utilities (CRWU) initiative assists the water sector, which includes drinking water, wastewater, and stormwater utilities, in addressing climate change impacts.\footcite[][]{EPAwaterutilities}
\end{itemize}
	
	

Some federal departments are also taking actions specific to their purview. For example, the Department of Transportation’s Congestion Mitigation and Air Quality (CMAQ) Improvement Program has provided over \$13 billion dollars in funding since its inception to State DOTs, Metropolitan Planning Organizations, and other sponsors to invest in projects that reduce emissions from transportation-related sources.\footcite[][]{CMAQ}\footcite[][]{CMAQProgram}  Since October 2009, the Department of Energy and the Department of Housing and Urban Development have jointly completed energy upgrades in more than one million homes across the country.\footcite[][]{WHenergy}



\textbf{Regional, state, and local level}



There are a vast number of initiatives happening across the U.S. at the regional, state, and local levels. 
Due to the sheer volume, these initiatives cannot possibly be catalogued in this space. 
However, the website of the Department of Transportation includes useful links to local action plans and databases of regional initiatives across the country.\footcite[][]{USDTaction} \footcite[][]{USDTinitiatives}



One notably aggressive regional initiative is California Senate Bill X1-2, signed into law as the \emph{California Renewable Energy Resources Act} by California Governor Jerry Brown in 2011.\footcite[][]{CaliSBX12}
The act directs the California Public Utilities Commission's Renewable Energy Resources Program to increase the amount of electricity generated from eligible renewable energy resources to an amount that equals at least 20\% of the total electricity sold to retail customers in California per year by December 31st 2013, 25\% by December 31st 2016 and 33\% by December 31st 2020.\footcite[][]{CaliforniaRenewableOverview}


\begin{vcom}
It would be good to find a few more examples here.
\end{vcom}


\textbf{Joint State-Provincial Initiatives}



There are several cross-border initiatives between U.S. states and Canadian provinces that are working to address climate change. These initiatives have been designed to reduce GHG emissions, develop clean energy sources, and achieve other environmental and economic goals.\footcite[][]{CCESinitiatives} They include:
\begin{description}
	\item[North America 2050] ``[A] multi-state, multi-regional collaborative working constructively on climate change and clean energy'', ``participants are committed to policies that move their jurisdictions toward a low carbon economy while creating jobs, enhancing energy independence and security, protecting public health and the environment, and demonstrating climate leadership''.\footcite[][]{CCESinitiatives}
	\item[Western Climate Initiative] A coalition consisting of California, British Columbia, Manitoba, Ontario, and Quebec which aims to ``reduce regional GHG emissions to 15 percent below 2005 levels by 2020 and spur investment in and development of clean-energy technologies, create green jobs, and protect public health'' in part by implementing a ``flexible, market-based, regional cap-and-trade program that caps greenhouse gas emissions and uses tradable permits to incent development of renewable and lower-polluting energy sources''.\footcite[][]{WCIPartners} \footcite[][]{WCIProgram} \footcite[][]{CCESinitiatives}
	\item[Regional Greenhouse Gas Initiative] A cooperative effort involving Connecticut, Delaware, Maine, Maryland, Massachusetts, New Hampshire, New York, Rhode Island, and Vermont, the RGGI is the first market-based regulatory program in the United States to reduce greenhouse gas emissions.\footcite[][]{RGGIWelcome} The initiative aims to cap and reduce CO2 emissions from the power sector.\footcite[][]{RGGIWelcome} \footcite[][]{CCESinitiatives}
	\item[Midwest Greenhouse Gas Reduction Accord] This regional agreement between the six governors in the Midwestern Governors Association (Illinois, Indiana, Iowa, Kansas, Minnesota, Michigan, Missouri, Ohio, and Wisconsin) and the premier of Manitoba aims to establish GHG reduction targets, ``[d]evelop a market-based and multi-sector cap-and-trade mechanism to help achieve those reduction targets'', establish a system to track GHG reductions, and take additional steps in areas like low-carbon fuel standards.\footcite[][]{GovsClimatePlatform} \footcite[][]{CCESinitiatives}
	\item[Transportation and Climate Initiative] This regional collaboration ``seeks to develop the clean energy economy and reduce oil dependence and greenhouse gas emissions from the transportation sector'', with a focus in four main areas: ``clean vehicles and fuels, sustainable communities, freight efficiency, and information and communication technologies''.\footcite[][]{TranspoClimate} Participating jurisdictions include Connecticut, Delaware, the District of Columbia, Maryland, Maine, Massachusetts, New Hampshire, New Jersey, New York, Pennsylvania, Rhode Island, and Vermont.\footcite[][]{GeorgetownOnTC}
\end{description}